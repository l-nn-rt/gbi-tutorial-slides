% !TeX root = gbitut04.tex
%beamer

% Comment/uncomment this line to toggle handout mode
\newcommand{\handout}{}

\input{../framework/PraeambelTut.tex}

\morescalingdelimiters

\begin{document}
\starttut{4}

\begin{frame}{Zu Blatt \#3}
	
	Schnitt: \quad 17 / 22~P
	
	\pause
	\begin{figure}
	    \centering
	    \includegraphics[scale=0.45]{./Punkteverteilung.pdf}
	\end{figure}
\end{frame}

\begin{frame}{Zu Blatt \#3}{Aufgabe 3.1 a) ii)}
    Beweisen Sie folgende Äquivalenz mittels Äquivalenzumformungen:
    $$ \alA \bimp \alB \equiv \bnot \alB \bimp \bnot \alA $$
    % $$ F_1 = \left(\left(\left(\alB \bimp \alA \right) \boder \alB \right) \bimp (\bnot \alA)\right) \bund \alB$$
    Lösung:
    \[ \alA \bimp \alB \equiv \bnot \alA \boder \alB \equiv \alB \boder \bnot \alA \equiv \bnot \bnot \alB \boder \bnot \alA \equiv \bnot \alB \bimp \bnot \alA \]
    oder
    \begin{align*}
        \alA \bimp \alB &\equiv \bnot \alA \boder \alB \\
        &\equiv \alB \boder \bnot \alA \\
        &\equiv \bnot \bnot \alB \boder \bnot \alA \\
        &\equiv \bnot \alB \bimp \bnot \alA
    \end{align*}
\end{frame}

\begin{frame}{Zu Blatt \#3}{Aufgabe 3.4 b)}
        \begin{align*}
            f_0 &= 0 \\
            f_1 &= 1 \\
            f_{n+1} &= f_n + f_{n-1} \\
        \end{align*}

    	Behauptung: \[\forall n \in \N_+ : f_n \le \left(\frac{7}{4}\right)^{n-1}\]
	\pause
	\begin{block}{Induktionsanfang}
		$n = 1$: $f_n = f_1 = 1 \le \left(\frac{7}{4}\right)^{0} = 1$. \;
		$n = 2$: $f_n = f_2 = 1 \le \left(\frac{7}{4}\right)^{1} = \frac{7}{4}$. \;
		\textbf{\checked}
	\end{block}
\end{frame}

\begin{frame}[t]	
	\begin{block}{Induktionsvoraussetzung}
		Für festes, aber beliebiges $n \in \N_0$ gelte: $f_n \le \left(\frac{7}{4}\right)^{n-1}$ und $f_{n-1} \le \left(\frac{7}{4}\right)^{n-2}$.
	\end{block}
    \pause
	\uncover<+->{}
	\begin{block}{Induktionsschritt}
		Zeige die Aussage für $n+1$:\\
		\begin{align*}
			f_{n+1} &\stackrel{\text{Def.}}{=} f_{n} + f_{n-1} \\
				&\stackrel{\text{IV}}{\le} \left(\frac{7}{4}\right)^{n-1} + \left(\frac{7}{4}\right)^{n-2}\\
				&= \left(\frac{7}{4}\right)^{n-2} \cdot \left(1 + \frac{7}{4}\right)
				= \left(\frac{7}{4}\right)^{n-2} \cdot \left(\frac{44}{16}\right) \\
				&\le \left(\frac{7}{4}\right)^{n-2} \cdot \left(\frac{49}{16}\right)
				= \left(\frac{7}{4}\right)^{n-2} + \left(\frac{7}{4}\right)^{2}
				= \left(\frac{7}{4}\right)^{n} \qed
		\end{align*}
	\end{block}
\end{frame}

% Induktion Wörter Länge
% \begin{frame}{Induktion mit Wörtern}
% 	\begin{block}{Behauptung}
% 		Seien $A, B$ zwei beliebige Alphabete. Definiere die Funktion $f \from A^* \functionto A^*$,
% 		\begin{align*}
% 			f(\eps) &:= \eps \\
% 			\text{Für } w \in A^*, a \in A: \quad f(a \* w) &:= 
% 			\begin{cases}
% 				a \* f(w), &a \in B \\
% 				f(w), &\text{sonst}
% 			\end{cases}\\
% 		\end{align*}
	
% 	Dann gilt $\forall w \in A^*: \size{f(w)} \le \size w$.
% 	\end{block}
% \end{frame}

% \begin{frame}{Und jetzt mit Wörtern}
% 	Induktion über die Wortlänge ($n = \size w$):\\[0.5em]
% 	\pause
% 	\begin{block}{Induktionsanfang}
% 		$n = 0$: Nur das leere Wort hat Länge 0. Also $w = \eps$.\\
% 		$f(\eps) = \eps \impl \size{f(w)} = \size w = 0$. \; \textbf{\checked}
% 	\end{block}
% 	\pause
% 	\begin{block}{Induktionsvoraussetzung}
% 		Für \textbf{ein} $n \in \N_0$ gelte: Für alle Wörter der Länge $n$ über $A$ (also $w \in A^n$) ist $\size{f(w)} \le \size w$.
% 	\end{block}
% \end{frame}

% \begin{frame}{Und jetzt mit Wörtern}
% 	\begin{block}{Induktionsschritt}
% 		Zeige die Aussage für $n+1$:\\
% 		Sei $w \in A^{n+1}$ ein Wort der Länge $n+1$.\\
% 		\pause
% 		Dann teilen wir es auf in $w = a \* v$, wobei $a \in A$ und $v \in A^n$.\\
% 		Nach IV gilt: $\size{f(v)} \le \size v$.\\
% 		\pause
% 		\smallskip
% 		\textbf{Fall 1}: \\
% 			\quad $a \in B$: $f(w) = a \* f(v)$ \\
% 			\quad \impl $\size{f(w)} = 1 + \size{f(v)} \le 1 + \size v = 1+n = \size w$.\\
% 		\pause
% 		\smallskip
% 		\textbf{Fall 2}: \\
% 			\quad $a \notin B$: $f(w) = f(v)$ \\
% 			\quad \impl $\size{f(w)} = \size{f(v)} \le \size v = n \le n+1 = \size w$.\\
% 		\pause
% 		\smallskip
% 		Also gilt: $\size{f(w)} \le \size w. \qed$
% 	\end{block}
% \end{frame}



\framePrevEpisode

\begin{frame}{Rückblick}
	\begin{itemize}
		\item \textbf{Aussagen} sind Sätze, die wahr oder falsch sind
		\item Wir können Aussagen mit \textbf{Konnektiven} zusammenbauen: \\
		$\bund, \boder, \bnot, \bimp$
		\item \textbf{Aussagevariablen} helfen dabei, konkrete Inhalte zu ignorieren 
		\item \textbf{Interpretationen} liefern Wahrheitswerte zu Variablen
		\item $val_I(\*)$ liefert Wahrheitswert für ganze Formel (rekursiv)
	\end{itemize}
\end{frame}

\begin{frame}[t]{Wahr oder Falsch?}
 	% Socrative: https://b.socrative.com/teacher/#import-quiz/31489145
 	\Socrative
 	
 	\TrueQuestion{Dieser Satz ist eine Aussage.}
 	\FalseQuestionE{$\alA \boder \alB = \alB \boder \alA$}{Das sind syntaktisch verschiedene AL-Formeln!}
 	\TrueQuestion{$\alA \boder \alB \equiv \alB \boder \alA$}
 	% \TrueQuestion{Der AL-Kalkül ist vollständig und korrekt.}  % Wurde noch gar nicht erklärt...!?
%  	\TrueQuestion{Es gibt unendlich viele Axiome im Aussagenkalkül.}
%  	\FalseQuestion{$\bleftBr \word G \bimp \word H \brightBr \; \vdash \;  \bleftBr \word H \bimp \word G \brightBr$}
 	\FalseQuestion{Induktion kann man nur auf Zahlen anwenden.}
\end{frame}



\section{Induktive Definitionen}

\begin{frame}{Zum Aufwärmen: Domino}
	Drei Dominosteine sind mit gleichem Abstand (kleiner halbe Größe) in einer Reihe aufgestellt. Wir stoßen den ersten Stein der Reihe in Richtung des zweiten Steins um. \\
	Wird der dritte Stein umfallen? Kann man das (einfach) beweisen? \\[1em]
	\pause
	Nun stehen (abzählbar) unendlich viele Dominosteine wie oben hintereinander. Wieder stoßen wir den ersten Stein um. \\
	Wird jeder Stein irgendwann umfallen? Kann man das (einfach) beweisen? \\[1em]
	\pause
	Werden irgendwann alle Steine umgefallen sein? \\
	\pause Nein, denn für jeden Zeitpunkt können wir (mindestens) einen Stein angeben, der noch nicht umgefallen ist.\\
	Wir sehen also: Alle Steine fallen um, aber es sind niemals alle umgefallen.
\end{frame}

\newcommand{\Fib}{\mathcal{F}\hspace{-1pt}}

\begin{frame}{Induktive Definitionen}
	\begin{exampleblock}{Beispiel: \emph{Fibonacci}-Reihe}
		\begin{align*}
		\Fib_0 &:= 0 \\
		\Fib_1 &:= 1 \\
		\text{Für } n \geq 0: \quad \Fib_{n+2} &:= \Fib_{n+1} + \Fib_n 		
		\end{align*}
		\pause
		\vspace{-\baselineskip}
% 		\begin{table}
% 			\centering
% 			\begin{tabular}{|c|c|c|c|c|c|c|c|c|c|c|}
% 				\hline
% 				$n$ & 0 & 1 & 2 & 3 & 4 & 5 & 6 & 7 & 8 & 9 \\ \hline
% 				$\Fib_n$ & 0 & 1 & 1 & 2 & 3 & 5 & 8 & 13 & 21 & 34 \\ \hline
% 			\end{tabular}
% 		\end{table}
		
		\pause
		\textbf{Wohldefiniertheit}: 
		\begin{itemize}
			\item Für alle Fälle etwas definieren 
			\item Nicht für einen Fall was Widersprüchliches definieren
		\end{itemize}
		
	\end{exampleblock}
	
\end{frame}


\section{Vollständige Induktion}

\morescalingdelimiters

% Induktion Vorstellung
\begin{frame}{Vollständige Induktion}
	Wir haben: Aussage $A_n$ für alle $n \in \N_0$ \quad (z.B. $A_n$: „$\size{\word{a}^n} = n$“) \\
	Wir wollen beweisen: $A_n$ ist für alle $n \in \N_0$ wahr \\[0.5em]
	\pause
	Zeige dazu: \centered{	
		$A_n$ ist für $n = 0$ wahr  \\
		\textbf{und} \\
		Wenn $A_n$ für ein $n$ wahr ist, dann ist $A_n$ auch für $n+1$ wahr 
	}
\end{frame}

\begin{frame}[t]{Vorgehen}
	\only<1-3|handout:1>{
		Behauptung: \quad $\forall n \in \N_0: (n^3 - n) \text{ ist durch 3 teilbar (tb)}$.
		\pause
		\begin{block}{Induktionsanfang (IA)}
			Beweise die Aussage für die erste Zahl (Basisfall):\\
			$n = 0 \impl (0^3 - 0) = 0$ ist durch 3 tb. \; \textbf{\checked}
		\end{block}
		\pause
		\bigskip
	}
	\only<3-|handout:1>{
				Wir nehmen ein $n$, von dem wir schon gezeigt haben, dass die Aussage gilt:\\
        \vspace{-.5\baselineskip}
		\begin{block}{Induktionsvoraussetzung (IV)}
			Für \textbf{ein beliebiges (aber festes)} $n \in \N_0$ gelte: $(n^3 - n)$ ist durch 3 tb. 
		\end{block}
	}
\end{frame}
\begin{frame}[t]{Vorgehen}
	\vspace{-.5\baselineskip}
	\begin{block}{Induktionsvoraussetzung (IV)}
		Für \textbf{ein beliebiges (aber festes)} $n \in \N_0$ gelte: $(n^3 - n)$ ist durch 3 tb. 
	\end{block}
	\begin{block}{Induktionsschritt (IS)}
	    \pause
		Zeige die Aussage für $n+1$, verwende dabei die IV (und \textbf{dasselbe} $n$ wie in der IV!).\\
		\pause
		\medskip
		Wir formen erst mal um:
		\begin{align*}
		(n+1)^3 - (n+1) &= n^3 + 3n^2 + 3n + 1 - n - 1 \\
		&= (n^3 - n) + (3n^2 + 3n) \\
		&= \underbrace{(n^3-n)}_{\shortstack{\footnotesize nach IV \\ \footnotesize durch 3 tb.}} + \underbrace{3 \* (n^2 + n)}_{\shortstack{\footnotesize offensichtlich \\ \footnotesize durch 3 tb.}}. \qed
		\end{align*}
	\end{block}
\end{frame}


\begin{frame}{Induktionsvoraussetzung}
	\Huge \centering
	\alert{
		„Für \textbf{ein} $n \in \N_0$ gelte ...“ \\
		\bigskip
		{ \LARGE
		Nicht: „Für alle...“,\\
		das wollen wir mit der Induktion ja erst zeigen!
		}
	}
\end{frame}

\begin{frame}[t]{Vollständige Induktion}
	\begin{itemize}
		\item \textbf{Einfaches} Prinzip (Muss man \textit{verstehen}, reines Auswendiglernen des Schemas kann schief gehen!), vielfältige Anwendungsmöglichkeiten
		\item \textbf{Variationen} möglich: Induktionsanfang bei $1, 42, ...$
	\end{itemize}
	
	\FalseQuestionE{Ich benutze für jeden Beweis Induktion.}{}
\end{frame}

%\input{../Bloecke/Induktion_Vogel.tex}

% Induktion Übung

\begin{frame}{Und jetzt ihr}
    Betrachten Sie die Abbildung $f: \N_0 \to \N_0$, die wie folgt festgelegt ist:
    \begin{align*}
        f(0) = 0&&\forall n \in \N_0: f(n+1) = 2f(n) + n + 5
    \end{align*}
% 	Behauptung: \[\forall n \in \N_+ : \sum_{k=0}^{n}{\frac{1}{2^k}} = 2 \* \tuple{1 - \frac{1}{2^{n+1}}}\]
    Beweisen Sie durch vollständige Induktion, dass für jedes $n \in \N_0$ gilt:
    \[f(n) = 6(2^n - 1) - n\]
	\pause
	\begin{block}{Induktionsanfang}
		$n = 0$: $f(0)=0 = 6 \cdot 0 - 0 = 6 \cdot (2^0 - 1) - 0 $. \; \textbf{\checked}
	\end{block}
	\pause
	\begin{block}{Induktionsvoraussetzung}
		Für ein $n \in \N_0$ gelte: $f(n) = 6(2^n - 1) - n$.
	\end{block}
\end{frame}

\begin{frame}[t]
	\begin{block}{Induktionsvoraussetzung}
		Für ein $n \in \N_0$ gelte: $f(n) = 6(2^n - 1) - n$.
	\end{block}
	\uncover<+->{}
	\begin{block}{Induktionsschritt}
		Zeige die Aussage für $n+1$:\\
		\begin{align*}
			f(n+1)
				&= \uncover<+->{2f(n)+n+5}\\
				\uncover<+->{&\stackrel{\text{IV}}{=} 2(6(2^n-1)-n)+n+5\\}
				\uncover<+->{
				&= 6(2^{n+1}-2)-2n+n+5\\}
				\uncover<+->{
				&= 6(2^{n+1}-1)-6-n-1+6\\}
				\uncover<+->{
				&= 6(2^{n+1}-1)-(n+1) \qed}
		\end{align*}
	\end{block}
\end{frame}

\begin{frame}{Und jetzt ihr}
	Behauptung: \[\forall n \in \N_+ : \sum_{k=0}^{n}{\frac{1}{2^k}} = 2 \* \tuple{1 - \frac{1}{2^{n+1}}}\]
	\pause
	\begin{block}{Induktionsanfang}
		$n = 1$: $\sum_{k=0}^{1}{\frac{1}{2^k}} = \frac{3}{2} = 2 \* \frac{3}{4}$. \; \textbf{\checked}
	\end{block}
	\pause
	\begin{block}{Induktionsvoraussetzung}
		Für ein $n \in \N_0$ gelte: $\sum_{k=0}^{n}{\frac{1}{2^k}} = 2 \* \tuple{1 - \frac{1}{2^{n+1}}}$.
	\end{block}
\end{frame}

\begin{frame}[t]
	\uncover<+->{}
	\begin{block}{Induktionsschritt}
		Zeige die Aussage für $n+1$:\\
		\begin{align*}
			\sum_{k=0}^{n+1}{\frac{1}{2^k}}
				&= \uncover<+->{\underbrace{\sum_{k=0}^{n}{\frac{1}{2^k}}}_{\stackrel{\text{IV}}{=} 2 \* \tuple{1 - \frac{1}{2^{n+1}}}} + \frac{1}{2^{n+1}}}\\
				\uncover<+->{&= 2 \* \left(1 - \frac{1}{2^{n+1}}\right) + \frac{1}{2^{n+1}}\\
				&= 2 \* \left(1 - \frac{2}{2^{n+2}} + \frac{1}{2^{n+2}}\right)\\
				&= 2 \* \left(1 - \frac{1}{2^{(n+1)+1}}\right). \qed}
		\end{align*}
	\end{block}
\end{frame}

% Induktion Wörter Länge
% \begin{frame}{Und jetzt mit Wörtern}
% 	\begin{block}{Behauptung}
% 		Seien $A, B$ zwei beliebige Alphabete. Definiere die Funktion $f \from A^* \functionto A^*$,
% 		\begin{align*}
% 			f(\eps) &:= \eps \\
% 			\text{Für } w \in A^*, \mu \in A: \quad f(\mu \* w) &:= 
% 			\begin{cases}
% 				\mu \* f(w), &\mu \in B \\
% 				f(w), &\text{sonst}
% 			\end{cases}\\
% 		\end{align*}
	
% 	Dann gilt $\forall w \in A^*: \size{f(w)} \le \size w$.
% 	\end{block}
% \end{frame}

% \begin{frame}{Und jetzt mit Wörtern}
% 	Induktion über die Wortlänge ($n = \size w$):\\[0.5em]
% 	\pause
% 	\begin{block}{Induktionsanfang}
% 		$n = 0$: Nur das leere Wort hat Länge 0. Also $w = \eps$.\\
% 		$f(\eps) = \eps \impl \size{f(w)} = \size w = 0$. \; \textbf{\checked}
% 	\end{block}
% 	\pause
% 	\begin{block}{Induktionsvoraussetzung}
% 		Für \textbf{ein} $n \in \N_0$ gelte: Für alle Wörter der Länge $n$ über $A$ (also $w \in A^n$) ist $\size{f(w)} \le \size w$.
% 	\end{block}
% \end{frame}

% \begin{frame}{Und jetzt mit Wörtern}
% 	\begin{block}{Induktionsschritt}
% 		Zeige die Aussage für $n+1$:\\
% 		Sei $w \in A^{n+1}$ ein Wort der Länge $n+1$.\\
% 		\pause
% 		Dann teilen wir es auf in $w = \mu \* v$, wobei $\mu \in A$ und $v \in A^n$.\\
% 		Nach IV gilt: $\size{f(v)} \le \size v$.\\
% 		\pause
% 		\smallskip
% 		\textbf{Fall 1}: \\
% 			\quad $\mu \in B$: $f(w) = \mu \* f(v)$ \\
% 			\quad \impl $\size{f(w)} = 1 + \size{f(v)} \le 1 + \size v = 1+n = \size w$.\\
% 		\pause
% 		\smallskip
% 		\textbf{Fall 2}: \\
% 			\quad $\mu \notin B$: $f(w) = f(v)$ \\
% 			\quad \impl $\size{f(w)} = \size{f(v)} \le \size v = n \le n+1 = \size w$.\\
% 		\pause
% 		\smallskip
% 		Also gilt: $\size{f(w)} \le \size w. \qed$
% 	\end{block}
% \end{frame}
\input{../Bloecke/FormaleSprachen.tex}




\mycomment{
	\section{Sprachen: Aufwärmen}
	
	
	\begin{frame}{Rückblick}
		\begin{itemize}
			\item \textbf{Alphabet} $A$ mit Zeichen, aus denen wir Wörter zusammenbauen
			\item Nicht immer haben all diese Wörter einen Sinn
			\item Wir definieren selbst, welche Wörter wir als korrekt ansehen und akzeptieren wollen.
			\item Eine solche Teilmenge aller möglichen Wörter nennen wir \textbf{formale Sprache}
		\end{itemize}
	\end{frame}
}




% TODO: Im letzten Jahr hat die Zeit nicht gereicht,
% daher diesen Inhalt hier eher streichen.
\input{Sprache_gesucht_beamer.tex}

\input{../Bloecke/FormaleSprachenP2.tex}

\begin{frame}{Ausblick: Klammerausdrücke}
	
	Was ist mit der Sprache aller gültigen Klammerausdrücke? Können wir die auch mit $\set{}$, $\*$, ${}^*$ und ${}^+$ angeben? \only<beamer:0>{\\ \emph{Spoiler: Nein, das geht nicht!}}\\[1em]
	\pause
	
	\begin{block}{}
		\Large
		\centering
		COMING SOON... \\[1em]
	\end{block}

	\begin{figure}[H]
		\centering
		\includegraphics[scale=0.7]{xkcd/(.png}
		\vspace{-7pt}
		\caption{ \texttt{\url{https://xkcd.com/859/}} }
	\end{figure}
\end{frame}

%\input{../Bloecke/Darstellung.tex}

\appendix
\beginbackup

% Induktion Wörter Länge
\begin{frame}{Induktion mit Wörtern}
	\begin{block}{Behauptung}
		Seien $A, B$ zwei beliebige Alphabete. Definiere die Funktion $f \from A^* \functionto A^*$,
		\begin{align*}
			f(\eps) &:= \eps \\
			\text{Für } w \in A^*, a \in A: \quad f(a \* w) &:= 
			\begin{cases}
				a \* f(w), &a \in B \\
				f(w), &\text{sonst}
			\end{cases}\\
		\end{align*}
	
	Dann gilt $\forall w \in A^*: \size{f(w)} \le \size w$.
	\end{block}
\end{frame}

\begin{frame}{Und jetzt mit Wörtern}
	Induktion über die Wortlänge ($n = \size w$):\\[0.5em]
	\pause
	\begin{block}{Induktionsanfang}
		$n = 0$: Nur das leere Wort hat Länge 0. Also $w = \eps$.\\
		$f(\eps) = \eps \impl \size{f(w)} = \size w = 0$. \; \textbf{\checked}
	\end{block}
	\pause
	\begin{block}{Induktionsvoraussetzung}
		Für \textbf{ein} $n \in \N_0$ gelte: Für alle Wörter der Länge $n$ über $A$ (also $w \in A^n$) ist $\size{f(w)} \le \size w$.
	\end{block}
\end{frame}

\begin{frame}{Und jetzt mit Wörtern}
	\begin{block}{Induktionsschritt}
		Zeige die Aussage für $n+1$:\\
		Sei $w \in A^{n+1}$ ein Wort der Länge $n+1$.\\
		\pause
		Dann teilen wir es auf in $w = a \* v$, wobei $a \in A$ und $v \in A^n$.\\
		Nach IV gilt: $\size{f(v)} \le \size v$.\\
		\pause
		\smallskip
		\textbf{Fall 1}: \\
			\quad $a \in B$: $f(w) = a \* f(v)$ \\
			\quad \impl $\size{f(w)} = 1 + \size{f(v)} \le 1 + \size v = 1+n = \size w$.\\
		\pause
		\smallskip
		\textbf{Fall 2}: \\
			\quad $a \notin B$: $f(w) = f(v)$ \\
			\quad \impl $\size{f(w)} = \size{f(v)} \le \size v = n \le n+1 = \size w$.\\
		\pause
		\smallskip
		Also gilt: $\size{f(w)} \le \size w. \qed$
	\end{block}
\end{frame}

\section{Zusammenfassung und Ausblick}

\begin{frame}	
	\begin{block}{Was ihr nun wissen solltet}
		\begin{itemize}
			\item Was eine formale Sprache ist und warum das Konzept wichtig ist
			\item Wie man einfache formale Sprachen formal angeben kann
			\item Einfache Operationen auf formalen Sprachen
			\item Wie man formale Sprachen angeben kann
			\item Wie man Beweise mit formalen Sprachen führt
		\end{itemize}
	\end{block}
	
	\begin{block}{Was nächstes Mal kommt}
		\begin{itemize}
			\item Wie man von Zahlendarstellungen zu Zahlen kommt...
			\item[] ... und wieder zurück
			\item Nicht immer so positiv: Negative Zahlen
			%\item Komprimierung: Huffmann-Codierungen
		\end{itemize}
	\end{block}
\end{frame}


\only<handout:0>{\slideThanks}

% TODO 
	\xkcdframevert{953}{Danke für eure Aufmerksamkeit! \smiley}{2.5}

% 	\xkcdframe{1516}{Win by Induction}{1.5}


\only<beamer:0>{\slideThanks}

\backupend

\end{document}
