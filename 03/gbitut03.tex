%beamer

% Comment/uncomment this line to toggle handout mode
% \newcommand{\handout}{}

\input{../framework/PraeambelTut.tex}

\morescalingdelimiters

\begin{document}
\starttut{3}

\framePrevEpisode

\subsection{Wahr oder Falsch?}

\begin{frame}[t]{Wahr oder Falsch?}
	\delimitershortfall=0pt
	\FalseQuestionE{$\word{aaba} \in \{ \word a, \word b\}^2\times\{\word a,\word b\}^2$}{Aber: $(\word{aa}, \word{ba}) \in \{\word a, \word b\}^2\times\{\word a, \word b\}^2$}
	\FalseQuestionE{$\setsize{\{ \eps \}} = 0$}{$\{ \eps \} \neq \{\}$}
	\FalseQuestionE{Wenn $\eps \in A^*$, dann $\size{\eps^3} = 3$.}{$\size{\eps^3} = 0$; außerdem gilt $\eps \in A^*$  immer, also für alle Alphabete $A$.}
\end{frame}



\input{../Bloecke/Aussagenlogik.tex}

\newcommand{\alnand}{\alv{\bar \land}}

%Übungsaufgabe 3.2 WS22/23
\begin{frame}{NAND}
	Man nennt eine Menge von Konnektiven, mit denen alle boolenschen Funktonen
	als aussagenlogische Formel ausgedrückt werden können, eine Basis der Aussagenlogik.
	Das aussagenlogische Konnektiv $NAND (\alv{ \bar \land})$ ist definiert als
	\begin{align*}
		val_I (G \aland H) = 
		\begin{cases}
			\W, & \text{falls } val_I(G) = \F \text{ oder } val_I (H) = \F \\
			\F, & \text{sonst}
		\end{cases}
	\end{align*}
	für beliebige Fromeln $G, H \in \LAL$

	\visible<+(-1)>{}
	\begin{alist}
		\item Geben Sie die Wahrheitstabelle für $G \alnand H$ an.
		\item Zeigen Sie dass $\alnand$ eine Basis ist. Geben Sie dafür (ohne Beweis)
		für die folgenden Formeln je eine äquivalente aussagenlogische Formel an, die nur $\alnand$ benutzt:
		\begin{enumerate}[i]
			\item $\bnot \plfoo{P}$ 
			\visible<+-|handout:2->{
				$\equiv \plfoo{P} \alnand \plfoo{P}$
			}
			\item $\plfoo{P} \bund \plfoo{Q}$
			\visible<+-|handout:2->{
				$ \equiv \bleftBr \plfoo{P} \alnand \plfoo{Q} \brightBr \alnand \bleftBr \plfoo{P} \alnand \plfoo{Q} \brightBr$
			}
			\item $\plfoo{P} \boder \plfoo{Q}$
			\visible<+-|handout:2->{
				$ \equiv \bleftBr \plfoo{P} \alnand \plfoo{P} \brightBr \alnand \bleftBr \plfoo{Q} \alnand \plfoo{Q} \brightBr$
			}
			\item $\plfoo{P} \bimp \plfoo{Q}$
			\visible<+-|handout:2->{
				$ \equiv \plfoo{P} \alnand \bleftBr \plfoo{Q} \alnand \plfoo{Q} \brightBr$
			}
		\end{enumerate}
	\end{alist}
\end{frame}

\input{../Bloecke/Aussagenlogik2}



\section{Induktive Definitionen}

\begin{frame}{Zum Aufwärmen: Domino}
	Drei Dominosteine sind mit gleichem Abstand (kleiner halbe Größe) in einer Reihe aufgestellt. Wir stoßen den ersten Stein der Reihe in Richtung des zweiten Steins um. \\
	Wird der dritte Stein umfallen? Kann man das (einfach) beweisen? \\[1em]
	\pause
	Nun stehen (abzählbar) unendlich viele Dominosteine wie oben hintereinander. Wieder stoßen wir den ersten Stein um. \\
	Wird jeder Stein irgendwann umfallen? Kann man das (einfach) beweisen? \\[1em]
	\pause
	Werden irgendwann alle Steine umgefallen sein? \\
	\pause Nein, denn für jeden Zeitpunkt können wir (mindestens) einen Stein angeben, der noch nicht umgefallen ist.\\
	Wir sehen also: Alle Steine fallen um, aber es sind niemals alle umgefallen.
\end{frame}

\newcommand{\Fib}{\mathcal{F}\hspace{-1pt}}

\begin{frame}{Induktive Definitionen}
	\begin{exampleblock}{Beispiel: \emph{Fibonacci}-Reihe}
		\begin{align*}
		\Fib_0 &:= 0 \\
		\Fib_1 &:= 1 \\
		\text{Für } n \geq 0: \quad \Fib_{n+2} &:= \Fib_{n+1} + \Fib_n 		
		\end{align*}
		\pause
		\vspace{-\baselineskip}
% 		\begin{table}
% 			\centering
% 			\begin{tabular}{|c|c|c|c|c|c|c|c|c|c|c|}
% 				\hline
% 				$n$ & 0 & 1 & 2 & 3 & 4 & 5 & 6 & 7 & 8 & 9 \\ \hline
% 				$\Fib_n$ & 0 & 1 & 1 & 2 & 3 & 5 & 8 & 13 & 21 & 34 \\ \hline
% 			\end{tabular}
% 		\end{table}
		
		\pause
		\textbf{Wohldefiniertheit}: 
		\begin{itemize}
			\item Für alle Fälle etwas definieren 
			\item Nicht für einen Fall was Widersprüchliches definieren
		\end{itemize}
		
	\end{exampleblock}
	
\end{frame}


\section{Vollständige Induktion}

\morescalingdelimiters

% Induktion Vorstellung
\begin{frame}{Vollständige Induktion}
	Wir haben: Aussage $A_n$ für alle $n \in \N_0$ \quad (z.B. $A_n$: „$\size{\word{a}^n} = n$“) \\
	Wir wollen beweisen: $A_n$ ist für alle $n \in \N_0$ wahr \\[0.5em]
	\pause
	Zeige dazu: \centered{	
		$A_n$ ist für $n = 0$ wahr  \\
		\textbf{und} \\
		Wenn $A_n$ für ein $n$ wahr ist, dann ist $A_n$ auch für $n+1$ wahr 
	}
\end{frame}

\begin{frame}[t]{Vorgehen}
	\only<1-3|handout:1>{
		Behauptung: \quad $\forall n \in \N_0: (n^3 - n) \text{ ist durch 3 teilbar (tb)}$.
		\pause
		\begin{block}{Induktionsanfang (IA)}
			Beweise die Aussage für die erste Zahl (Basisfall):\\
			$n = 0 \impl (0^3 - 0) = 0$ ist durch 3 tb. \; \textbf{\checked}
		\end{block}
		\pause
		\bigskip
	}
	\only<3-|handout:1>{
				Wir nehmen ein $n$, von dem wir schon gezeigt haben, dass die Aussage gilt:\\
        \vspace{-.5\baselineskip}
		\begin{block}{Induktionsvoraussetzung (IV)}
			Für \textbf{ein beliebiges (aber festes)} $n \in \N_0$ gelte: $(n^3 - n)$ ist durch 3 tb. 
		\end{block}
	}
\end{frame}
\begin{frame}[t]{Vorgehen}
	\vspace{-.5\baselineskip}
	\begin{block}{Induktionsvoraussetzung (IV)}
		Für \textbf{ein beliebiges (aber festes)} $n \in \N_0$ gelte: $(n^3 - n)$ ist durch 3 tb. 
	\end{block}
	\begin{block}{Induktionsschritt (IS)}
	    \pause
		Zeige die Aussage für $n+1$, verwende dabei die IV (und \textbf{dasselbe} $n$ wie in der IV!).\\
		\pause
		\medskip
		Wir formen erst mal um:
		\begin{align*}
		(n+1)^3 - (n+1) &= n^3 + 3n^2 + 3n + 1 - n - 1 \\
		&= (n^3 - n) + (3n^2 + 3n) \\
		&= \underbrace{(n^3-n)}_{\shortstack{\footnotesize nach IV \\ \footnotesize durch 3 tb.}} + \underbrace{3 \* (n^2 + n)}_{\shortstack{\footnotesize offensichtlich \\ \footnotesize durch 3 tb.}}. \qed
		\end{align*}
	\end{block}
\end{frame}


\begin{frame}{Induktionsvoraussetzung}
	\Huge \centering
	\alert{
		„Für \textbf{ein} $n \in \N_0$ gelte ...“ \\
		\bigskip
		{ \LARGE
		Nicht: „Für alle...“,\\
		das wollen wir mit der Induktion ja erst zeigen!
		}
	}
\end{frame}

\begin{frame}[t]{Vollständige Induktion}
	\begin{itemize}
		\item \textbf{Einfaches} Prinzip (Muss man \textit{verstehen}, reines Auswendiglernen des Schemas kann schief gehen!), vielfältige Anwendungsmöglichkeiten
		\item \textbf{Variationen} möglich: Induktionsanfang bei $1, 42, ...$
	\end{itemize}
	
	\FalseQuestionE{Ich benutze für jeden Beweis Induktion.}{}
\end{frame}

%\input{../Bloecke/Induktion_Vogel.tex}

% Induktion Übung

\begin{frame}{Und jetzt ihr}
    Betrachten Sie die Abbildung $f: \N_0 \to \N_0$, die wie folgt festgelegt ist:
    \begin{align*}
        f(0) = 0&&\forall n \in \N_0: f(n+1) = 2f(n) + n + 5
    \end{align*}
% 	Behauptung: \[\forall n \in \N_+ : \sum_{k=0}^{n}{\frac{1}{2^k}} = 2 \* \tuple{1 - \frac{1}{2^{n+1}}}\]
    Beweisen Sie durch vollständige Induktion, dass für jedes $n \in \N_0$ gilt:
    \[f(n) = 6(2^n - 1) - n\]
	\pause
	\begin{block}{Induktionsanfang}
		$n = 0$: $f(0)=0 = 6 \cdot 0 - 0 = 6 \cdot (2^0 - 1) - 0 $. \; \textbf{\checked}
	\end{block}
	\pause
	\begin{block}{Induktionsvoraussetzung}
		Für ein $n \in \N_0$ gelte: $f(n) = 6(2^n - 1) - n$.
	\end{block}
\end{frame}

\begin{frame}[t]
	\begin{block}{Induktionsvoraussetzung}
		Für ein $n \in \N_0$ gelte: $f(n) = 6(2^n - 1) - n$.
	\end{block}
	\uncover<+->{}
	\begin{block}{Induktionsschritt}
		Zeige die Aussage für $n+1$:\\
		\begin{align*}
			f(n+1)
				&= \uncover<+->{2f(n)+n+5}\\
				\uncover<+->{&\stackrel{\text{IV}}{=} 2(6(2^n-1)-n)+n+5\\}
				\uncover<+->{
				&= 6(2^{n+1}-2)-2n+n+5\\}
				\uncover<+->{
				&= 6(2^{n+1}-1)-6-n-1+6\\}
				\uncover<+->{
				&= 6(2^{n+1}-1)-(n+1) \qed}
		\end{align*}
	\end{block}
\end{frame}

\begin{frame}{Und jetzt ihr}
	Behauptung: \[\forall n \in \N_+ : \sum_{k=0}^{n}{\frac{1}{2^k}} = 2 \* \tuple{1 - \frac{1}{2^{n+1}}}\]
	\pause
	\begin{block}{Induktionsanfang}
		$n = 1$: $\sum_{k=0}^{1}{\frac{1}{2^k}} = \frac{3}{2} = 2 \* \frac{3}{4}$. \; \textbf{\checked}
	\end{block}
	\pause
	\begin{block}{Induktionsvoraussetzung}
		Für ein $n \in \N_0$ gelte: $\sum_{k=0}^{n}{\frac{1}{2^k}} = 2 \* \tuple{1 - \frac{1}{2^{n+1}}}$.
	\end{block}
\end{frame}

\begin{frame}[t]
	\uncover<+->{}
	\begin{block}{Induktionsschritt}
		Zeige die Aussage für $n+1$:\\
		\begin{align*}
			\sum_{k=0}^{n+1}{\frac{1}{2^k}}
				&= \uncover<+->{\underbrace{\sum_{k=0}^{n}{\frac{1}{2^k}}}_{\stackrel{\text{IV}}{=} 2 \* \tuple{1 - \frac{1}{2^{n+1}}}} + \frac{1}{2^{n+1}}}\\
				\uncover<+->{&= 2 \* \left(1 - \frac{1}{2^{n+1}}\right) + \frac{1}{2^{n+1}}\\
				&= 2 \* \left(1 - \frac{2}{2^{n+2}} + \frac{1}{2^{n+2}}\right)\\
				&= 2 \* \left(1 - \frac{1}{2^{(n+1)+1}}\right). \qed}
		\end{align*}
	\end{block}
\end{frame}

% Induktion Wörter Länge
% \begin{frame}{Und jetzt mit Wörtern}
% 	\begin{block}{Behauptung}
% 		Seien $A, B$ zwei beliebige Alphabete. Definiere die Funktion $f \from A^* \functionto A^*$,
% 		\begin{align*}
% 			f(\eps) &:= \eps \\
% 			\text{Für } w \in A^*, \mu \in A: \quad f(\mu \* w) &:= 
% 			\begin{cases}
% 				\mu \* f(w), &\mu \in B \\
% 				f(w), &\text{sonst}
% 			\end{cases}\\
% 		\end{align*}
	
% 	Dann gilt $\forall w \in A^*: \size{f(w)} \le \size w$.
% 	\end{block}
% \end{frame}

% \begin{frame}{Und jetzt mit Wörtern}
% 	Induktion über die Wortlänge ($n = \size w$):\\[0.5em]
% 	\pause
% 	\begin{block}{Induktionsanfang}
% 		$n = 0$: Nur das leere Wort hat Länge 0. Also $w = \eps$.\\
% 		$f(\eps) = \eps \impl \size{f(w)} = \size w = 0$. \; \textbf{\checked}
% 	\end{block}
% 	\pause
% 	\begin{block}{Induktionsvoraussetzung}
% 		Für \textbf{ein} $n \in \N_0$ gelte: Für alle Wörter der Länge $n$ über $A$ (also $w \in A^n$) ist $\size{f(w)} \le \size w$.
% 	\end{block}
% \end{frame}

% \begin{frame}{Und jetzt mit Wörtern}
% 	\begin{block}{Induktionsschritt}
% 		Zeige die Aussage für $n+1$:\\
% 		Sei $w \in A^{n+1}$ ein Wort der Länge $n+1$.\\
% 		\pause
% 		Dann teilen wir es auf in $w = \mu \* v$, wobei $\mu \in A$ und $v \in A^n$.\\
% 		Nach IV gilt: $\size{f(v)} \le \size v$.\\
% 		\pause
% 		\smallskip
% 		\textbf{Fall 1}: \\
% 			\quad $\mu \in B$: $f(w) = \mu \* f(v)$ \\
% 			\quad \impl $\size{f(w)} = 1 + \size{f(v)} \le 1 + \size v = 1+n = \size w$.\\
% 		\pause
% 		\smallskip
% 		\textbf{Fall 2}: \\
% 			\quad $\mu \notin B$: $f(w) = f(v)$ \\
% 			\quad \impl $\size{f(w)} = \size{f(v)} \le \size v = n \le n+1 = \size w$.\\
% 		\pause
% 		\smallskip
% 		Also gilt: $\size{f(w)} \le \size w. \qed$
% 	\end{block}
% \end{frame}

\appendix
\beginbackup

\section{Zusammenfassung und Ausblick}

\begin{frame}
	\begin{block}{Was ihr nun wissen solltet}
		\begin{itemize}
			\item Wie vollständige Induktion geht
			\item Aussagenlogik: Syntax und Semantik
			\item Wie man wahre Aussagen konstruiert \qquad (\#NoFakeNews!)
		\end{itemize}
	\end{block}

	\begin{block}{Was nächstes Mal kommt}
		\begin{itemize}
			\item Induktion
			\item Formale Sprachen
			\item Aus \word2 mach \word{10}: Übersetzungen
		\end{itemize}
	\end{block}
\end{frame}


\only<handout:0>{\slideThanks}

\xkcdframe{589}{Danke für eure Aufmerksamkeit! \smiley}{2}

\only<beamer:0>{\slideThanks}

\backupend

\end{document}
