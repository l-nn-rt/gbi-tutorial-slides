%beamer

% Comment/uncomment this line to toggle handout mode
 \newcommand{\handout}{}

\input{../framework/PraeambelTut.tex}

\morescalingdelimiters

\begin{document}
\starttut{2}

\Comment{
\begin{frame}{Organisatorisches}
	
	
		\begin{itemize}
			\item Nächsten Donnerstag: Abgabe des ersten Übungsblattes! (Achtung: Bei nachfolgenden Blättern müsst ihr euch selber erinnern!)
		\end{itemize}
	
\end{frame}
}

\begin{frame}{Zu Blatt \#1}
	
	Schnitt: \quad 9.5 / 21~P
	
	\pause
	\begin{figure}
	    \centering
	    \includegraphics[scale=0.45]{./Punkteverteilung.pdf}
	\end{figure}
\end{frame}
\begin{frame}{Zu Blatt \#1}
	\begin{itemize}[<+->]
		\item Erstmal: Volle Punktzahl ist utopisch. Fehler macht man, um draus zu lernen – dazu sind ÜBs da
		\item Nicht mehrere Alternativen angeben, unter denen ich mir die richtige heraussuchen soll. (That's \emph{your} job!)
	\end{itemize}
\end{frame}

\begin{frame}[t]{Übungsblätter: Häufige Fehler \daniel{und anderer Kram}}
	
	\TrueQuestionE{Beweise fängt man mit der Behauptung an.}{}
	\FalseQuestionE{Wenn wir $A$ schreiben, ist das immer eine Menge, \\ und $f, g$ sind immer Funktionen.}{Das muss man \textbf{explizit} angeben.}
	\FalseQuestionE{Ein Beispiel langt als Beweis.}{...bis ein anderes Beispiel herkommt, bei dem es kaputt geht... \\
		\impl eine Behauptung \textbf{widerlegen} geht mit nem Beispiel sehr gut!}
	%Mit einem Beispiel kann man eine Behauptung widerlegen, es zeigt aber nicht die Allgemeingültigkeit}

	
	\medskip
	
	\uncover<8->{Passt auf, was eure Variablen sind und welche Operationen ihr darauf anwenden könnt. Mengen kann man nicht mit $\bund$ und $\boder$ verknüpfen und auch keine \impl dazwischen tun.}
	
\end{frame}



\framePrevEpisode

\begin{frame}{Wahr oder falsch?}
	\Socrative
	\TrueQuestionE{$ \setsize{\emptyset} = 0$.}{}
	\FalseQuestionE{$ \setsize{ \set{\set{}} } = 0$.}{$ \setsize{\set{\set{}}} = \setsize{\set{\emptyset}} = 1$.}
	\TrueQuestionE{$ \setsize{\set{ 1, \set{2, 3}, 4, \set{5,6,7}}} = 4$.}{$\set{2, 3}$ und $\set{5,6,7}$ sind zwei einzelne Objekte!}
	\TrueQuestionE{$ \{1, 2\} = \{2, 1\}$.}{}
	\FalseQuestion{$ (1, 2) = (2, 1)$.}
	\FalseQuestionE{$ \left(M\cup A\right)\setminus M = A $.}{für z.~B. $ M = A = \{1\}$}
	%TODO Relationen-Fragen noch? Wenn's im ersten Tut schon vorkam!
	\visible<12->{
	\begin{block}{Bemerkungen}
		Menge / Tupel: Klammern beachten! Einfache Beispiele helfen!
	\end{block}
	}
\end{frame}
\begin{frame}[t]{Wahr oder Falsch?}
	\FalseQuestionE{Eine Funktion muss linkseindeutig sein.}{\impl Eine Funktion muss rechtseindeutig \textbf{und linkstotal} sein.}
	\TrueQuestion{Eine injektive Funktion ist linkseindeutig.}
	\TrueQuestion{Eine surjektive Funktion ist rechtstotal.}
	\FalseQuestion{Jede Relation ist eine Funktion.}
	\TrueQuestion{Jede Funktion ist eine Relation.}
\end{frame}



\section{Relationen}
\begin{frame}{Relation}
  \begin{block}{Relation}
	Eine \emph{Relation} $R$ ist eine Teilmenge eines kartesischen Produkts, also $R \subseteq A \times B$. \\
  \end{block}
\end{frame}




\begin{frame}{Eigenschaften von Relationen}
  \begin{block}{Linkstotal}
	Eine Relation $R$ heißt \emph{linkstotal}, wenn es für jedes $a \in A$ \emph{mindestens} ein $b \in B$ gibt. \\
	\glqq \hspace{1pt}Von jedem Element in A geht mindestens ein Pfeil nach B.\grqq
  \end{block}
  \begin{figure}
    \begin{minipage}{0.44\textwidth}
      \centering
      \begin{tikzpicture}[x={(12mm,0mm)},y={(0mm,-5mm)}]
        \tikzset{
          blob/.style={circle,fill,minimum size=0.7mm,inner sep=0,outer sep=0}
        }
        \tikzset{
          rahmen/.style={draw,inner sep=3mm,rounded corners=2mm}
        }
        \tikzset{
          pfeil/.style={shorten >=0.1mm,shorten <=1mm,thick,-latex}
        }

        \node[blob]  (Al) at (0,0) {};
        \node[blob]  (Bl) at (0,1) {};
        \node[blob]  (Cl) at (0,2) {};
        \node[rahmen,fit=(Al) (Bl) (Cl)] {};
        \node at ($ (Al) + (0,-1.3) $) {$A$};

        \node[blob]  (Ar) at (1,0) {};
        \node[blob]  (Br) at (1,1) {};
        \node[blob]  (Cr) at (1,2) {};
        \node[rahmen,fit=(Ar) (Br) (Cr)] {};
        \node at ($ (Ar) + (0,-1.3) $) {$B$};

        \draw[pfeil] (Al) -- (Br);
        \draw[pfeil] (Bl) -- (Cr);

        \draw[red,line width=0.7mm,opacity=0.9] (Cl) -- +(-1.5mm,-1.5mm) (Cl)
  -- ++(1.5mm,-1.5mm)
        (Cl) -- ++(-1.5mm,1.5mm) (Cl) -- ++(1.5mm,1.5mm);
      \end{tikzpicture}
      \\ \emph{Nicht} linkstotal
    \end{minipage}%
	\begin{minipage}{0.44\textwidth}
      \centering
      \begin{tikzpicture}[x={(12mm,0mm)},y={(0mm,-5mm)}]
        \tikzset{blob/.style={circle,fill,minimum size=0.7mm,inner sep=0,outer
  sep=0}}
        \tikzset{rahmen/.style={draw,inner sep=3mm,rounded corners=2mm}}
        \tikzset{pfeil/.style={shorten >=0.1mm,shorten <=1mm,thick, -latex}}

        \node[blob]  (Al) at (0,0) {};
        \node[blob]  (Bl) at (0,1) {};
        \node[blob]  (Cl) at (0,2) {};
        \node[rahmen,fit=(Al) (Bl) (Cl)] {};
        \node at ($ (Al) + (0,-1.3) $) {$A$};

        \node[blob]  (Ar) at (1,0) {};
        \node[blob]  (Br) at (1,1) {};
        \node[blob]  (Cr) at (1,2) {};
        \node[rahmen,fit=(Ar) (Br) (Cr)] {};
        \node at ($ (Ar) + (0,-1.3) $) {$B$};

        \draw[pfeil] (Al) -- (Br);
        \draw[pfeil] (Bl) -- (Cr);
        \draw[pfeil] (Cl) -- (Cr);

        \foreach \x in {Al, Bl, Cl} {
          \node at ([xshift=-3pt, yshift=3pt]\x.west)
            {\color{green!75!black}\checkmark};
        }
      \end{tikzpicture}
      \\ Linkstotal
    \end{minipage}
  \end{figure}
\end{frame}


\begin{frame}{Eigenschaften von Relationen}
  \begin{block}{Rechtseindeutig}
	Eine Relation $R$ heißt \emph{rechtseindeutig}, wenn es für jedes $a \in A$ \emph{höchstens} ein $b \in B$ gibt. \\
	\glqq \hspace{1pt}Von einem Element in A geht höchstens ein Pfeil nach B.\grqq
  \end{block}
  \begin{figure}
    \begin{minipage}{0.44\textwidth}
      \centering
      \begin{tikzpicture}[x={(12mm,0mm)},y={(0mm,-5mm)}]
        \tikzset{
          blob/.style={circle,fill,minimum size=0.7mm,inner sep=0,outer sep=0}
        }
        \tikzset{
          rahmen/.style={draw,inner sep=3mm,rounded corners=2mm}
        }
        \tikzset{
          pfeil/.style={shorten >=0.1mm,shorten <=1mm,thick,-latex}
        }

        \node[blob]  (Al) at (0,0) {};
        \node[blob]  (Bl) at (0,1) {};
        \node[blob]  (Cl) at (0,2) {};
        \node[rahmen,fit=(Al) (Bl) (Cl)] {};
        \node at ($ (Al) + (0,-1.3) $) {$A$};

        \node[blob]  (Ar) at (1,0) {};
        \node[blob]  (Br) at (1,1) {};
        \node[blob]  (Cr) at (1,2) {};
        \node[rahmen,fit=(Ar) (Br) (Cr)] {};
        \node at ($ (Ar) + (0,-1.3) $) {$B$};

        \draw[pfeil] (Al) -- (Br);
        \draw[pfeil] (Bl) -- (Cr);
        \draw[pfeil] (Al) -- (Cr);

        \coordinate (All) at ($ (Al) + (1.5mm,-0.75mm) $);
        \draw[red,line width=0.7mm,opacity=0.9]
          (All) -- +(-1.75mm,-0.68mm)
          (All) -- ++(0.68mm,-1.75mm)
          (All) -- ++(-0.68mm,1.75mm)
          (All) -- ++(1.75mm,0.68mm);
      \end{tikzpicture}
      \\ \emph{Nicht} rechtseindeutig
    \end{minipage}%
    \begin{minipage}{0.44\textwidth}
      \centering
      \begin{tikzpicture}[x={(12mm,0mm)},y={(0mm,-5mm)}]
        \tikzset{
          blob/.style={circle,fill,minimum size=0.7mm,inner sep=0,outer sep=0}
        }
        \tikzset{
          rahmen/.style={draw,inner sep=3mm,rounded corners=2mm}
        }
        \tikzset{
          pfeil/.style={shorten >=0.1mm,shorten <=1mm,thick,-latex}
        }

        \node[blob]  (Al) at (0,0) {};
        \node[blob]  (Bl) at (0,1) {};
        \node[blob]  (Cl) at (0,2) {};
        \node[rahmen,fit=(Al) (Bl) (Cl)] {};
        \node at ($ (Al) + (0,-1.3) $) {$A$};

        \node[blob]  (Ar) at (1,0) {};
        \node[blob]  (Br) at (1,1) {};
        \node[blob]  (Cr) at (1,2) {};
        \node[rahmen,fit=(Ar) (Br) (Cr)] {};
        \node at ($ (Ar) + (0,-1.3) $) {$B$};

        \draw[pfeil] (Al) -- (Br);
        \draw[pfeil] (Bl) -- (Cr);

        \foreach \x in {Al, Bl, Cl} {
          \node at ([xshift=-3pt, yshift=3pt]\x.west)
            {\color{green!75!black}\checkmark};
        }
      \end{tikzpicture}
      \\ Rechtseindeutig
    \end{minipage}
  \end{figure}
\end{frame}


\begin{frame}{Eigenschaften von Relationen}
  \begin{block}{Abbildung}
    Eine Relation $R$, welche linkstotal und rechtseindeutig ist, heißt eine \emph{Abbildung} $f$. \\
    \vspace{6pt}
    Man schreibt: $f: A \to B$, wobei $A$ der Definitionsbereich und $B$ der Ziel-/Wertebereich ist.
  \end{block}
  \begin{exampleblock}{Abbildung}
      Abbildung die alle geraden Zahlen auf 1 und alle ungeraden Zahlen auf 0 abbildet:
      $f \from \N \functionto \N, \; x \mapsto \casesl{ 1 & \text{wenn $x$ gerade} \\ 0 & \text{sonst} }$
  \end{exampleblock}
\end{frame}


\begin{frame}{Eigenschaften von Relationen}
  \begin{block}{Linkseindeutig}
	Eine Relation $R$ heißt \emph{linkseindeutig}, wenn es für jedes $b \in B$ \emph{höchstens} ein $a \in A$ gibt. \\
	Eine linkseindeutige Abbildung heißt \emph{injektiv}. \\
	\glqq \hspace{1pt}Zu jedem Element in B kommt höchstens ein Pfeil von A.\grqq
  \end{block}
  \begin{figure}
    \centering
    \begin{minipage}{0.44\textwidth}
      \centering
      \begin{tikzpicture}[x={(12mm,0mm)},y={(0mm,-5mm)}]
        \tikzset{
          blob/.style={circle,fill,minimum size=0.7mm,inner sep=0,outer sep=0}
        }
        \tikzset{
          rahmen/.style={draw,inner sep=3mm,rounded corners=2mm}
        }
        \tikzset{
          pfeil/.style={shorten >=0.1mm,shorten <=1mm,thick,-latex}
        }

        \node[blob]  (Al) at (0,0) {};
        \node[blob]  (Bl) at (0,1) {};
        \node[blob]  (Cl) at (0,2) {};
        \node[rahmen,fit=(Al) (Bl) (Cl)] {};
        \node at ($ (Al) + (0,-1.3) $) {$A$};

        \node[blob]  (Ar) at (1,0) {};
        \node[blob]  (Br) at (1,1) {};
        \node[blob]  (Cr) at (1,2) {};
        \node[rahmen,fit=(Ar) (Br) (Cr)] {};
        \node at ($ (Ar) + (0,-1.3) $) {$B$};

        \draw[pfeil] (Al) -- (Br);
        \draw[pfeil] (Bl) -- (Cr);
        \draw[pfeil] (Al) -- (Cr);

        \coordinate (Crr) at ($ (Cr) + (-1.5mm,0.75mm) $);
        \draw[red,line width=0.7mm,opacity=0.9]
          (Crr) -- +(-1.75mm,-0.68mm)
          (Crr) -- ++(0.68mm,-1.75mm)
          (Crr) -- ++(-0.68mm,1.75mm)
          (Crr) -- ++(1.75mm,0.68mm);
      \end{tikzpicture}
      \\ \emph{Nicht} linkseindeutig
    \end{minipage}%
    \begin{minipage}{0.44\textwidth}
      \centering
      \begin{tikzpicture}[x={(12mm,0mm)},y={(0mm,-5mm)}]
        \tikzset{blob/.style={circle,fill,minimum size=0.7mm,inner sep=0,outer
  sep=0}}
        \tikzset{rahmen/.style={draw,inner sep=3mm,rounded corners=2mm}}
        \tikzset{pfeil/.style={shorten >=0.1mm,shorten <=1mm,thick,-latex }}

        \node[blob]  (Al) at (0,0) {};
        \node[blob]  (Bl) at (0,1) {};
        \node[blob]  (Cl) at (0,2) {};
        \node[rahmen,fit=(Al) (Bl) (Cl)] {};
        \node at ($ (Al) + (0,-1.3) $) {$A$};

        \node[blob]  (Ar) at (1,0) {};
        \node[blob]  (Br) at (1,1) {};
        \node[blob]  (Cr) at (1,2) {};
        \node[rahmen,fit=(Ar) (Br) (Cr)] {};
        \node at ($ (Ar) + (0,-1.3) $) {$B$};

        \draw[pfeil] (Al) -- (Br);
        \draw[pfeil] (Bl) -- (Cr);

        \foreach \x in {Ar, Br, Cr} {
          \node at ([xshift=5pt, yshift=2pt]\x.east)
            {\color{green!75!black}\checkmark};
        }
      \end{tikzpicture}
      \\ Linkseindeutig
    \end{minipage}
  \end{figure}
\end{frame}


\begin{frame}{Eigenschaften von Relationen}
  \begin{block}{Rechtstotal}
	Eine Relation $R$ heißt \emph{rechtstotal}, wenn es für jedes $b \in B$ \emph{mindestens} ein $a \in A$ gibt. \\
	Eine rechtstotale Abbildung heißt \emph{surjektiv}. \\
	\glqq \hspace{1pt}Zu jedem Element in B kommt ein Pfeil von A.\grqq
  \end{block}
  \begin{figure}
    \begin{minipage}{0.44\textwidth}
      \centering
      \begin{tikzpicture}[x={(12mm,0mm)},y={(0mm,-5mm)}]
        \tikzset{
          blob/.style={circle,fill,minimum size=0.7mm,inner sep=0,outer sep=0}
        }
        \tikzset{
          rahmen/.style={draw,inner sep=3mm,rounded corners=2mm}
        }
        \tikzset{
          pfeil/.style={shorten >=0.1mm,shorten <=1mm,thick,-latex}
        }

        \node[blob]  (Al) at (0,0) {};
        \node[blob]  (Bl) at (0,1) {};
        \node[blob]  (Cl) at (0,2) {};
        \node[rahmen,fit=(Al) (Bl) (Cl)] {};
        \node at ($ (Al) + (0,-1.3) $) {$A$};

        \node[blob]  (Ar) at (1,0) {};
        \node[blob]  (Br) at (1,1) {};
        \node[blob]  (Cr) at (1,2) {};
        \node[rahmen,fit=(Ar) (Br) (Cr)] {};
        \node at ($ (Ar) + (0,-1.3) $) {$B$};

        \draw[pfeil] (Al) -- (Br);
        \draw[pfeil] (Bl) -- (Cr);

        \draw[red,line width=0.7mm,opacity=0.9] (Ar) -- +(-1.5mm,-1.5mm) (Ar)
  -- ++(1.5mm,-1.5mm)
        (Ar) -- ++(-1.5mm,1.5mm) (Ar) -- ++(1.5mm,1.5mm);
      \end{tikzpicture}
      \\ \emph{Nicht} rechtstotal
    \end{minipage}%
    \begin{minipage}{0.44\textwidth}
      \centering
      \begin{tikzpicture}[x={(12mm,0mm)},y={(0mm,-5mm)}]
        \tikzset{
          blob/.style={circle,fill,minimum size=0.7mm,inner sep=0,outer sep=0}
        }
        \tikzset{
          rahmen/.style={draw,inner sep=3mm,rounded corners=2mm}
        }
        \tikzset{
          pfeil/.style={shorten >=0.1mm,shorten <=1mm,thick,-latex}
        }

        \node[blob]  (Al) at (0,0) {};
        \node[blob]  (Bl) at (0,1) {};
        \node[blob]  (Cl) at (0,2) {};
        \node[rahmen,fit=(Al) (Bl) (Cl)] {};
        \node at ($ (Al) + (0,-1.3) $) {$A$};

        \node[blob]  (Ar) at (1,0) {};
        \node[blob]  (Br) at (1,1) {};
        \node[blob]  (Cr) at (1,2) {};
        \node[rahmen,fit=(Ar) (Br) (Cr)] {};
        \node at ($ (Ar) + (0,-1.3) $) {$B$};

        \draw[pfeil] (Al) -- (Br);
        \draw[pfeil] (Bl) -- (Cr);
        \draw[pfeil] (Al) -- (Ar);

        \foreach \x in {Ar, Br, Cr} {
          \node at ([xshift=5pt, yshift=2pt]\x.east)
            {\color{green!75!black}\checkmark};
        }
      \end{tikzpicture}
      \\ Rechtstotal
    \end{minipage}
  \end{figure}
\end{frame}


\begin{frame}{Eigenschaften von Relationen}
  \begin{block}{Bijektiv}
    Ist eine Abbildung injektiv und surjektiv, so nennt man sie \emph{bijektiv}.
  \end{block}
  \pause
  \begin{exampleblock}{Beispiel}
	\begin{center}
			$g \colon \{1, 2\} \to \{3, 4\}$ \\
			$1 \mapsto 3$ \\
			$2 \mapsto 4$
	\end{center}		
		Ja, alle Zuordnungen einzeln auflisten geht auch (Formeln sind aber manchmal praktischer). \\
		Als Relation geschrieben ist $g = \{(1, 3), (2, 4)\}$. Diese Abbildung ist bijektiv.
	\end{exampleblock}
\end{frame}



\section{Binäre Operationen}

\begin{frame}{Definition}
    \begin{block}{Definition}
          Eine \textbf{binäre Operation} auf einer Menge $A$ ist eine Abbildung $\diamond: A \times A \to A$.
    \end{block}
\end{frame}



\begin{frame}{Eigenschaften von binären Operationen}
  \begin{block}{Assoziativ}
    Eine binäre Operation $\diamond: A \times A \to A$ auf einer Menge A heißt \textbf{assoziativ}, wenn für alle $a, b, c \in A$ gilt:
    \begin{center}
        $a \diamond (b \diamond c) = (a \diamond b) \diamond c$
    \end{center}
  \end{block}
  \begin{block}{Kommutativ}
    Eine binäre Operation $\diamond: A \times A \to A$ auf einer Menge A heißt \textbf{kommutativ}, wenn für alle $a, b \in A$ gilt:
    \begin{center}
        $a \diamond b = b \diamond a$
    \end{center}
  \end{block}
\end{frame}

\begin{frame}{Aufgabe 3}
    Auf $\Z \times \Z$ sei eine binäre Operation $\diamond: (\Z \times \Z)^2 \to \Z \times \Z$ wie folgt für alle Paare $(x_1, y_1),(x_2, y_2) \in \Z \times \Z$ definiert:
    \begin{center}
        $(x_1, y_1) \diamond (x_2, y_2) = (x_1 y_2, x_2y_1)$.
    \end{center}
    \visible<+(-1)>{}

    \begin{alist}
        \item Zeigen Sie, dass $\diamond$ nicht kommutativ ist. \\
        \visible<+-|handout:2->{
        $(1,0)\diamond(1,1)=(1,0)\neq(0,1)=(1,1)\diamond(1,0)$
        }
        \item Zeigen Sie, dass $\diamond$ nicht assoziativ ist. \\
        \visible<+-|handout:2->{
        $((1,1)\diamond(1,1))\diamond(1,0)=(1,1)\diamond(1,0)=(0,1)\neq(1,0)=(1,1)\diamond(0,1)=(1,1)\diamond((1,1,)\diamond(1,0))$
        }
        \item Geben Sie eine \textit{unendliche} Menge $K \subset \Z \times \Z$ an, für die folgendes gilt: Wenn $k_1, k_2 \in K$ sind, dann gilt $k_1 \diamond k_2 = k_2 \diamond k_1$ \\
        \visible<+-|handout:2->{
        z.B. $K = \{(x,x)\mid x \in \Z \}$
        }
        \item Beweisen Sie, dass Ihr $K$ die in Teilaufgabe c) verlangte Eigenschaft hat.\\
        \visible<+-|handout:2->{
            Für jede $(x,x),(y,y) \in K$ gilt: $(x,x)\diamond(y,y)=(xy,xy)=(y,y)\diamond(x,x)$.
        }
    \end{alist}
\end{frame}

\section{Wörter}

\begin{frame}{Wörter}
	\begin{block}{Definition}
		\begin{itemize}
			\item Ein \textbf{Alphabet} $A$ ist eine endliche, nichtleere Menge von Zeichen. \pause
			\item Ein \textbf{Wort} $w$  über einem Alphabet $A$ ist ein \textbf{endliche Folge von Zeichen} aus A \\ 
		\end{itemize}
	\end{block}	
	
	\pause
	\begin{block}{Definition}
		$A^*$ ist die \textbf{Menge aller Wörter} beliebiger Länge, die nur Zeichen aus $A$ enthalten.
		% brauchen wir das?
		\mycomment{, also:\\
		\pause 
		$A^*$ ist die Menge aller Abbildungen $w \from \Z_n \functionto B$ mit $n \in \N_0$ und $B \subseteq A$. \\}
	\end{block}

	\pause
	\begin{exampleblock}{Beispiel}
		Sei $ A = \{ \word a, \word b \} $ ein Alphabet. 
		Dann sind $ w_1 = \word{aabbabab}$ und $w_2 = \word{ab} $ zwei mögliche Wörter über $A$. \\
		\impl $ w_1 \in A^*, w_2 \in A^*$
	\end{exampleblock}

\end{frame}

\begin{frame}{Wörter -- \thassedaniel{formal betrachtet}{Formalkram}}
	\begin{block}{Formale Definition}
		Ein \textbf{Wort} $w$  über einem Alphabet $A$ ist eine \textit{surjektive} Abbildung $w \from \Z_n \functionto B$ mit $B \subseteq A$. \\ 
		\smallskip
		Zur Erinnerung: \; $ \Z_n = \{i \in \N_0 \mid 0 \leq i < n \} = \set{0, ..., n-1} $ 
	\end{block}
	
	\begin{exampleblock}{Beispiel}
		Sei $ A := \{ \word a, ..., \word z \} $ ein Alphabet und $B := \set{\word a, \word b, \word e, \word n, \word o, \word r, \word s, \word t} \subseteq A$. \\
		Dann ist ein Wort $w$ gegeben durch $ w \from \Z_{12} \functionto B$, \\ 
		\smallskip
		\begin{tabular}{c|c@{\:}c@{\:}c@{\:}c@{\:}c@{\:}c@{\:}c@{\:}c@{\:}c@{\:}c@{\:}c@{\:}c@{\:}c@{\:}c@{\:}c}
			$i$ & \small 0 & \small 1 & \small 2 & \small 3 & \small 4 & \small 5 & \small 6 & \small 7 & \small 8 & \small 9 & \small 10 & \small 11 \\
			\hline
			$w(i)$ & \word a & \word n & \word a & \word n & \word a & \word s & \word s & \word o & \word r & \word b & \word e & \word t 
		\end{tabular} \\
		\bigskip
		Oder einfach wie vorhin: \quad  $w := \word{ananassorbet}$.
	\end{exampleblock}
\end{frame}

\begin{frame}[t]{Das leere Wort}
	\begin{block}{Definition}
		Wir definieren das \textbf{leere Wort} als $$ \varepsilon \from \Z_0  \functionto \set{} \qquad\text{bzw.}\qquad \eps \from \set{} \functionto \set{} $$ \pause

	\end{block}
	
	\begin{block}{Wichtig}
		Das leere Wort ist \textbf{nicht} Nichts, sondern ein echtes Wort! \\
		($0 \in \N_0$ ist ja auch ne Zahl! $\emptyset$ ist auch ne Menge!)
	\end{block}
		%Wichtig: Das leere Wort ist auch ein \enquote{echtes, gleichberechtigtes} Wort. Die Null ist bei den natürlichen Zahlen ja auch nicht einfach \enquote{nichts}. \medskip \pause
		
		\YesQuestionE{Ist $\eps \from \set{} \functionto \set{} $ eine Relation? Und eine Funktion? Ist es surjektiv?}{
			Achtung: Wir müssen fordern, dass Wörter surjektiv sind, sonst ist das leere Wort nicht eindeutig!
		}
	
\end{frame}


\begin{frame}{Konkatenation von Wörtern}
	\begin{block}{Wörter „aneinanderkleben“}
		Seien $w_1 = \word{erd}, w_2 = \word{blumentopf}$. \\
		Dann ist $ w_1 \* w_2 = \word{erdblumentopf} \neq w_2 \* w_1 = \word{blumentopferd}$. \\ 
		\pause
		Konkatenation ist also \textbf{nicht kommutativ!} \\
		\YesQuestionE{Ist sie \textbf{assoziativ}?}{$(w_1 \* w_2) \* w_3 = w_1 \* (w_2 \* w_3)$.} \medskip 
		\only<+->{Außerdem gilt immer $ \eps \* w \* \eps = w $.}
		
	\end{block}
	
	\pause
	
	\begin{block}{Beobachtung}
		Falls $w=w_1\* w_2 $ und $w_1 \in A^* , w_2 \in B^* $, dann ist
		$ w\in (A \cup B)^* $.
	\end{block}
	
\end{frame}
\begin{frame}{Konkatenation von Wörtern}

	\begin{block}{„Wortpotenzen“}
		\begin{align*}
			w^0 &= \eps \\
			w^k &= \underbrace{w \* w  \cdots  w}_{\text{$k$-mal}}
		\end{align*}

	\end{block}

\end{frame}

\begin{frame}{Länge von Wörtern}
	\begin{block}{Definition}
		$\size w$: Die \textbf{Länge} eines Wortes $w$, also die Gesamtanzahl der Zeichen von $w$.
	\end{block}
	
	\begin{block}{Beispiel}
		$$ \size{\word{hallo}} = 5 \qqquad \size \eps = 0$$
	\end{block}

	\pause
	\begin{block}{Lemma}
		$$ \size{a \* b} = \size a + \size b $$ \\
		\pause
		$$ \size{w^k} = k \* \size w $$
	\end{block}

\end{frame}

\begin{frame}{Wörter}
	\begin{block}{Definition}
		$A^n$: \emph{Menge aller Wörter der Länge $n$} über dem Alphabet $A$.\\
		Wie kann man damit $A^*$ ausdrücken? \\
		\pause
		\[ A^* = \bigcup \limits_{i = 0}^\infty A^i \]
	\end{block}
	
	
	\pause
	\begin{block}{Erinnerung}
		$
		\bigcup \limits_{i\in I} M_i = \{ x \mid \text{es gibt ein } i\in I \text{ so, dass } x\in  M_i \}  
		$
	\end{block}
\end{frame}

\begin{frame}{Aufgabe 4}
	\visible<+(-1)>{}
	\begin{itemize}
		\item Welche Wörter lassen sich aus dem Alphabet $A = \{ \word a , \word b \}$ bilden? Was enthält die Menge $A^*$? \\
		\visible<+-|handout:2->{
			\impl $\word a, \word b, \word {aa}, \word {bb}, \word {ab}, \word {ba}, \word {aaa}, \word {bbb}, \dots$ \\
			\impl $A^*$ enthält genau diese Wörter (und auch $\eps$!).
		}
		\item Ist das Wort $w = \word{aabb} \* \word{ba}$ ein Element der Menge $A^5$? \\
		\visible<+-|handout:2->{
			\impl Nein. $w = \word{aabbba}$ ist zwar ein Wort über $A$, aber hat Länge $6 \neq 5$.
		}
		\item Was ist $A^2 \times A^2$? \\
			\visible<+-|handout:2->{
				\impl $ A^2 \times A^2 = \set{(\word{aa},\word {aa}),(\word{aa},\word{bb}),(\word {aa},\word {ab}),(\word {aa},\word {ba}),(\word {bb},\word {aa}), \dots }$
			} \\ \smallskip
			Wir definieren die Abbildung $f \from A^* \times A^* \functionto A^*, \; (w_1, w_2) \mapsto w_1 \cdot w_2$. \\
			Was ist $f(A^2 \times A^2)$? \\
			\uncover<+-|handout:2->{
				\impl $ f(A^2 \times A^2) = \set{\word{aaaa}, \word{aabb}, \word{aaab}, \word{aaba}, \word{bbaa}, \dots } = A^4 $
			}\\
		\bigskip
		Erinnerung: Für $f \from A \functionto B, M \subseteq A$ definieren wir $f(M) = \{f(a) \mid a \in M\}$
	\end{itemize}
\end{frame}

\begin{frame}{Aufgabe 5}
	Es sei $n \in \N_0$. Wir definieren eine Abbildung $\tau: \Pot (\Z_n) \to A^n$, sodass für $M \in \Pot (\Z_n)$ das Wort $w = \tau(M) \in A^n$ mit $A = \set{0, 1}$ eindeutig durch folgende Eigenschaft gegeben ist: Für jedes $i \in \Z_n$ ist $w(i)=1$ genau dann, wenn $i \in M$ ist.


	\visible<+(-1)>{}
	\begin{alist}
		\item Es sei $n=4$. Geben Sie $\tau(M)$ für jedes $M \in \set{\set{1,2,3},\set{0,3},\set{\emptyset}}$ explizit an. \\
		\visible<+-|handout:2->{
			\impl $\tau(\set{\{1,2,3}) = \word{0111}, \tau(\set{0,3}) = \word{1001},\tau(\set{\{\emptyset}) = \word{0000}$
		}
		\item Es sei $n$ wieder beliebig. Zeigen Sie, dass $\tau$ bijektiv ist. \\
		\visible<+-|handout:2->{
			Injektiv: Seien $M_1, M_2 \in \Pot (\Z_n) mit w = \tau(M_1)=\tau(M_2)$ Dann gilt:
			\begin{center}
				$i \in M_1$ gdw. $w(i) = 1$ gdw. $i \in M_2$
			\end{center}
			Also ist $M_1 = M_2$ und damit $\tau$ injektiv. \\
			Surjektiv: Wir beobachten $\size{\Pot (\Z_n)} = 2^{\size{\Z_n}} = 2^n = \size{A}^n = \size{A^n}$. Somit muss $\tau$ auch surjektiv sein.
		}
		\item Es sei $w \in A^n$. Geben Sie eine hinreichende und notwendige Bedingung für $M \in  \Pot (\Z_n)$ an, sodass $\tau(M)=w$ ist. In Ihrer Formulierung darf dabei $\tau$ nirgends vorkommen. \\
		\visible<+-|handout:2->{
			\impl $M = \set{i \in \Z_n \mid w(i) = 1}$
		}

	\end{alist}
\end{frame}


% Übungsaufgabe 4.1 WS22/23
\begin{frame}{Palindrome}
	Sei $A$ ein Alphabet. Wir betrachten die Funktion $\diamond^R: A* \to A*$,
	die jedem Wort $w \in A*$ seine Spiegelung $w^R \in A$ zuordnet.
	Dieser Operator ist festegelegt für jedes $w \in A$ durch
	\begin{align*}
		\size{w^R} &= \size{w} \\
		w^R(i) &= w(\size{w} - i - 1)	\text{für alle } 0 \leq i \leq \size{w}
	\end{align*}
	Ein Wort $w \in A*$ heißt Palindrom, wenn gilt $w^R = w$. $P_A = \set{w \in A* \mid w = w^R}$
	bezeichnet die Menge aller Palindrome über A.

	\visible<+(-1)>{}
	\begin{alist}
		\item Geben Sie alle Palidrome der Länge 4 über $\set{\word a, \word b}$ an. \\
		\visible<+-|handout:2->{
			$\set{\word{aaaa}, \word{abba}, \word{baab}, \word{bbbb}}$
		}
		\item Betrachten Sie nun die $Q_A = \set{ww^R \mid w \in A*}$ \\
			 Zeigen oder widerlegen Sie $P_A \subseteq Q_A$ \\
		\visible<+-|handout:2->{
			$\word{aba} \in P_A$, aber wegen $\size{\word{aba}} = 3$ nicht in $Q_A$.
		}
	\end{alist}
\end{frame}




\appendix
\beginbackup

\section{Zusammenfassung und Ausblick}

\begin{frame}	
	\begin{block}{Was ihr nun wissen solltet}
		\begin{itemize}
			\item Wie man mit Relationen umgeht
% 			\item Welche Eigenschaften Relationen haben können
% 			\item Was Abbildungen sind und welche Eigenschaften sie haben können
			\item Wie man mit Wörtern rechnet
		\end{itemize}
	\end{block}
	
	\begin{block}{Was nächstes Mal kommt}
		\begin{itemize}
			\item Sinnvollere Gebilde als \word{\thassedaniel{egnarts\sp si\sp efiL}{retsinnaL\sp nosrO}} mit \emph{formalen Sprachen}
			\item Aus Sage wird Logik: \emph{Aussagenlogik}
		\end{itemize}
	\end{block}
\end{frame}

\only<handout:0>{\slideThanks}

\xkcdframe{1121}{Danke für eure Aufmerksamkeit! \smiley}{1.5}

\only<beamer:0>{\slideThanks}

\backupend

\end{document}