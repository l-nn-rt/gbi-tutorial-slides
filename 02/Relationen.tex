
\section{Relationen}
\begin{frame}{Relation}
  \begin{block}{Relation}
	Eine \emph{Relation} $R$ ist eine Teilmenge eines kartesischen Produkts, also $R \subseteq A \times B$. \\
  \end{block}
\end{frame}




\begin{frame}{Eigenschaften von Relationen}
  \begin{block}{Linkstotal}
	Eine Relation $R$ heißt \emph{linkstotal}, wenn es für jedes $a \in A$ \emph{mindestens} ein $b \in B$ gibt. \\
	\glqq \hspace{1pt}Von jedem Element in A geht mindestens ein Pfeil nach B.\grqq
  \end{block}
  \begin{figure}
    \begin{minipage}{0.44\textwidth}
      \centering
      \begin{tikzpicture}[x={(12mm,0mm)},y={(0mm,-5mm)}]
        \tikzset{
          blob/.style={circle,fill,minimum size=0.7mm,inner sep=0,outer sep=0}
        }
        \tikzset{
          rahmen/.style={draw,inner sep=3mm,rounded corners=2mm}
        }
        \tikzset{
          pfeil/.style={shorten >=0.1mm,shorten <=1mm,thick,-latex}
        }

        \node[blob]  (Al) at (0,0) {};
        \node[blob]  (Bl) at (0,1) {};
        \node[blob]  (Cl) at (0,2) {};
        \node[rahmen,fit=(Al) (Bl) (Cl)] {};
        \node at ($ (Al) + (0,-1.3) $) {$A$};

        \node[blob]  (Ar) at (1,0) {};
        \node[blob]  (Br) at (1,1) {};
        \node[blob]  (Cr) at (1,2) {};
        \node[rahmen,fit=(Ar) (Br) (Cr)] {};
        \node at ($ (Ar) + (0,-1.3) $) {$B$};

        \draw[pfeil] (Al) -- (Br);
        \draw[pfeil] (Bl) -- (Cr);

        \draw[red,line width=0.7mm,opacity=0.9] (Cl) -- +(-1.5mm,-1.5mm) (Cl)
  -- ++(1.5mm,-1.5mm)
        (Cl) -- ++(-1.5mm,1.5mm) (Cl) -- ++(1.5mm,1.5mm);
      \end{tikzpicture}
      \\ \emph{Nicht} linkstotal
    \end{minipage}%
	\begin{minipage}{0.44\textwidth}
      \centering
      \begin{tikzpicture}[x={(12mm,0mm)},y={(0mm,-5mm)}]
        \tikzset{blob/.style={circle,fill,minimum size=0.7mm,inner sep=0,outer
  sep=0}}
        \tikzset{rahmen/.style={draw,inner sep=3mm,rounded corners=2mm}}
        \tikzset{pfeil/.style={shorten >=0.1mm,shorten <=1mm,thick, -latex}}

        \node[blob]  (Al) at (0,0) {};
        \node[blob]  (Bl) at (0,1) {};
        \node[blob]  (Cl) at (0,2) {};
        \node[rahmen,fit=(Al) (Bl) (Cl)] {};
        \node at ($ (Al) + (0,-1.3) $) {$A$};

        \node[blob]  (Ar) at (1,0) {};
        \node[blob]  (Br) at (1,1) {};
        \node[blob]  (Cr) at (1,2) {};
        \node[rahmen,fit=(Ar) (Br) (Cr)] {};
        \node at ($ (Ar) + (0,-1.3) $) {$B$};

        \draw[pfeil] (Al) -- (Br);
        \draw[pfeil] (Bl) -- (Cr);
        \draw[pfeil] (Cl) -- (Cr);

        \foreach \x in {Al, Bl, Cl} {
          \node at ([xshift=-3pt, yshift=3pt]\x.west)
            {\color{green!75!black}\checkmark};
        }
      \end{tikzpicture}
      \\ Linkstotal
    \end{minipage}
  \end{figure}
\end{frame}


\begin{frame}{Eigenschaften von Relationen}
  \begin{block}{Rechtseindeutig}
	Eine Relation $R$ heißt \emph{rechtseindeutig}, wenn es für jedes $a \in A$ \emph{höchstens} ein $b \in B$ gibt. \\
	\glqq \hspace{1pt}Von einem Element in A geht höchstens ein Pfeil nach B.\grqq
  \end{block}
  \begin{figure}
    \begin{minipage}{0.44\textwidth}
      \centering
      \begin{tikzpicture}[x={(12mm,0mm)},y={(0mm,-5mm)}]
        \tikzset{
          blob/.style={circle,fill,minimum size=0.7mm,inner sep=0,outer sep=0}
        }
        \tikzset{
          rahmen/.style={draw,inner sep=3mm,rounded corners=2mm}
        }
        \tikzset{
          pfeil/.style={shorten >=0.1mm,shorten <=1mm,thick,-latex}
        }

        \node[blob]  (Al) at (0,0) {};
        \node[blob]  (Bl) at (0,1) {};
        \node[blob]  (Cl) at (0,2) {};
        \node[rahmen,fit=(Al) (Bl) (Cl)] {};
        \node at ($ (Al) + (0,-1.3) $) {$A$};

        \node[blob]  (Ar) at (1,0) {};
        \node[blob]  (Br) at (1,1) {};
        \node[blob]  (Cr) at (1,2) {};
        \node[rahmen,fit=(Ar) (Br) (Cr)] {};
        \node at ($ (Ar) + (0,-1.3) $) {$B$};

        \draw[pfeil] (Al) -- (Br);
        \draw[pfeil] (Bl) -- (Cr);
        \draw[pfeil] (Al) -- (Cr);

        \coordinate (All) at ($ (Al) + (1.5mm,-0.75mm) $);
        \draw[red,line width=0.7mm,opacity=0.9]
          (All) -- +(-1.75mm,-0.68mm)
          (All) -- ++(0.68mm,-1.75mm)
          (All) -- ++(-0.68mm,1.75mm)
          (All) -- ++(1.75mm,0.68mm);
      \end{tikzpicture}
      \\ \emph{Nicht} rechtseindeutig
    \end{minipage}%
    \begin{minipage}{0.44\textwidth}
      \centering
      \begin{tikzpicture}[x={(12mm,0mm)},y={(0mm,-5mm)}]
        \tikzset{
          blob/.style={circle,fill,minimum size=0.7mm,inner sep=0,outer sep=0}
        }
        \tikzset{
          rahmen/.style={draw,inner sep=3mm,rounded corners=2mm}
        }
        \tikzset{
          pfeil/.style={shorten >=0.1mm,shorten <=1mm,thick,-latex}
        }

        \node[blob]  (Al) at (0,0) {};
        \node[blob]  (Bl) at (0,1) {};
        \node[blob]  (Cl) at (0,2) {};
        \node[rahmen,fit=(Al) (Bl) (Cl)] {};
        \node at ($ (Al) + (0,-1.3) $) {$A$};

        \node[blob]  (Ar) at (1,0) {};
        \node[blob]  (Br) at (1,1) {};
        \node[blob]  (Cr) at (1,2) {};
        \node[rahmen,fit=(Ar) (Br) (Cr)] {};
        \node at ($ (Ar) + (0,-1.3) $) {$B$};

        \draw[pfeil] (Al) -- (Br);
        \draw[pfeil] (Bl) -- (Cr);

        \foreach \x in {Al, Bl, Cl} {
          \node at ([xshift=-3pt, yshift=3pt]\x.west)
            {\color{green!75!black}\checkmark};
        }
      \end{tikzpicture}
      \\ Rechtseindeutig
    \end{minipage}
  \end{figure}
\end{frame}


\begin{frame}{Eigenschaften von Relationen}
  \begin{block}{Abbildung}
    Eine Relation $R$, welche linkstotal und rechtseindeutig ist, heißt eine \emph{Abbildung} $f$. \\
    \vspace{6pt}
    Man schreibt: $f: A \to B$, wobei $A$ der Definitionsbereich und $B$ der Ziel-/Wertebereich ist.
  \end{block}
  \begin{exampleblock}{Abbildung}
      Abbildung die alle geraden Zahlen auf 1 und alle ungeraden Zahlen auf 0 abbildet:
      $f \from \N \functionto \N, \; x \mapsto \casesl{ 1 & \text{wenn $x$ gerade} \\ 0 & \text{sonst} }$
  \end{exampleblock}
\end{frame}


\begin{frame}{Eigenschaften von Relationen}
  \begin{block}{Linkseindeutig}
	Eine Relation $R$ heißt \emph{linkseindeutig}, wenn es für jedes $b \in B$ \emph{höchstens} ein $a \in A$ gibt. \\
	Eine linkseindeutige Abbildung heißt \emph{injektiv}. \\
	\glqq \hspace{1pt}Zu jedem Element in B kommt höchstens ein Pfeil von A.\grqq
  \end{block}
  \begin{figure}
    \centering
    \begin{minipage}{0.44\textwidth}
      \centering
      \begin{tikzpicture}[x={(12mm,0mm)},y={(0mm,-5mm)}]
        \tikzset{
          blob/.style={circle,fill,minimum size=0.7mm,inner sep=0,outer sep=0}
        }
        \tikzset{
          rahmen/.style={draw,inner sep=3mm,rounded corners=2mm}
        }
        \tikzset{
          pfeil/.style={shorten >=0.1mm,shorten <=1mm,thick,-latex}
        }

        \node[blob]  (Al) at (0,0) {};
        \node[blob]  (Bl) at (0,1) {};
        \node[blob]  (Cl) at (0,2) {};
        \node[rahmen,fit=(Al) (Bl) (Cl)] {};
        \node at ($ (Al) + (0,-1.3) $) {$A$};

        \node[blob]  (Ar) at (1,0) {};
        \node[blob]  (Br) at (1,1) {};
        \node[blob]  (Cr) at (1,2) {};
        \node[rahmen,fit=(Ar) (Br) (Cr)] {};
        \node at ($ (Ar) + (0,-1.3) $) {$B$};

        \draw[pfeil] (Al) -- (Br);
        \draw[pfeil] (Bl) -- (Cr);
        \draw[pfeil] (Al) -- (Cr);

        \coordinate (Crr) at ($ (Cr) + (-1.5mm,0.75mm) $);
        \draw[red,line width=0.7mm,opacity=0.9]
          (Crr) -- +(-1.75mm,-0.68mm)
          (Crr) -- ++(0.68mm,-1.75mm)
          (Crr) -- ++(-0.68mm,1.75mm)
          (Crr) -- ++(1.75mm,0.68mm);
      \end{tikzpicture}
      \\ \emph{Nicht} linkseindeutig
    \end{minipage}%
    \begin{minipage}{0.44\textwidth}
      \centering
      \begin{tikzpicture}[x={(12mm,0mm)},y={(0mm,-5mm)}]
        \tikzset{blob/.style={circle,fill,minimum size=0.7mm,inner sep=0,outer
  sep=0}}
        \tikzset{rahmen/.style={draw,inner sep=3mm,rounded corners=2mm}}
        \tikzset{pfeil/.style={shorten >=0.1mm,shorten <=1mm,thick,-latex }}

        \node[blob]  (Al) at (0,0) {};
        \node[blob]  (Bl) at (0,1) {};
        \node[blob]  (Cl) at (0,2) {};
        \node[rahmen,fit=(Al) (Bl) (Cl)] {};
        \node at ($ (Al) + (0,-1.3) $) {$A$};

        \node[blob]  (Ar) at (1,0) {};
        \node[blob]  (Br) at (1,1) {};
        \node[blob]  (Cr) at (1,2) {};
        \node[rahmen,fit=(Ar) (Br) (Cr)] {};
        \node at ($ (Ar) + (0,-1.3) $) {$B$};

        \draw[pfeil] (Al) -- (Br);
        \draw[pfeil] (Bl) -- (Cr);

        \foreach \x in {Ar, Br, Cr} {
          \node at ([xshift=5pt, yshift=2pt]\x.east)
            {\color{green!75!black}\checkmark};
        }
      \end{tikzpicture}
      \\ Linkseindeutig
    \end{minipage}
  \end{figure}
\end{frame}


\begin{frame}{Eigenschaften von Relationen}
  \begin{block}{Rechtstotal}
	Eine Relation $R$ heißt \emph{rechtstotal}, wenn es für jedes $b \in B$ \emph{mindestens} ein $a \in A$ gibt. \\
	Eine rechtstotale Abbildung heißt \emph{surjektiv}. \\
	\glqq \hspace{1pt}Zu jedem Element in B kommt ein Pfeil von A.\grqq
  \end{block}
  \begin{figure}
    \begin{minipage}{0.44\textwidth}
      \centering
      \begin{tikzpicture}[x={(12mm,0mm)},y={(0mm,-5mm)}]
        \tikzset{
          blob/.style={circle,fill,minimum size=0.7mm,inner sep=0,outer sep=0}
        }
        \tikzset{
          rahmen/.style={draw,inner sep=3mm,rounded corners=2mm}
        }
        \tikzset{
          pfeil/.style={shorten >=0.1mm,shorten <=1mm,thick,-latex}
        }

        \node[blob]  (Al) at (0,0) {};
        \node[blob]  (Bl) at (0,1) {};
        \node[blob]  (Cl) at (0,2) {};
        \node[rahmen,fit=(Al) (Bl) (Cl)] {};
        \node at ($ (Al) + (0,-1.3) $) {$A$};

        \node[blob]  (Ar) at (1,0) {};
        \node[blob]  (Br) at (1,1) {};
        \node[blob]  (Cr) at (1,2) {};
        \node[rahmen,fit=(Ar) (Br) (Cr)] {};
        \node at ($ (Ar) + (0,-1.3) $) {$B$};

        \draw[pfeil] (Al) -- (Br);
        \draw[pfeil] (Bl) -- (Cr);

        \draw[red,line width=0.7mm,opacity=0.9] (Ar) -- +(-1.5mm,-1.5mm) (Ar)
  -- ++(1.5mm,-1.5mm)
        (Ar) -- ++(-1.5mm,1.5mm) (Ar) -- ++(1.5mm,1.5mm);
      \end{tikzpicture}
      \\ \emph{Nicht} rechtstotal
    \end{minipage}%
    \begin{minipage}{0.44\textwidth}
      \centering
      \begin{tikzpicture}[x={(12mm,0mm)},y={(0mm,-5mm)}]
        \tikzset{
          blob/.style={circle,fill,minimum size=0.7mm,inner sep=0,outer sep=0}
        }
        \tikzset{
          rahmen/.style={draw,inner sep=3mm,rounded corners=2mm}
        }
        \tikzset{
          pfeil/.style={shorten >=0.1mm,shorten <=1mm,thick,-latex}
        }

        \node[blob]  (Al) at (0,0) {};
        \node[blob]  (Bl) at (0,1) {};
        \node[blob]  (Cl) at (0,2) {};
        \node[rahmen,fit=(Al) (Bl) (Cl)] {};
        \node at ($ (Al) + (0,-1.3) $) {$A$};

        \node[blob]  (Ar) at (1,0) {};
        \node[blob]  (Br) at (1,1) {};
        \node[blob]  (Cr) at (1,2) {};
        \node[rahmen,fit=(Ar) (Br) (Cr)] {};
        \node at ($ (Ar) + (0,-1.3) $) {$B$};

        \draw[pfeil] (Al) -- (Br);
        \draw[pfeil] (Bl) -- (Cr);
        \draw[pfeil] (Al) -- (Ar);

        \foreach \x in {Ar, Br, Cr} {
          \node at ([xshift=5pt, yshift=2pt]\x.east)
            {\color{green!75!black}\checkmark};
        }
      \end{tikzpicture}
      \\ Rechtstotal
    \end{minipage}
  \end{figure}
\end{frame}


\begin{frame}{Eigenschaften von Relationen}
  \begin{block}{Bijektiv}
    Ist eine Abbildung injektiv und surjektiv, so nennt man sie \emph{bijektiv}.
  \end{block}
  \pause
  \begin{exampleblock}{Beispiel}
	\begin{center}
			$g \colon \{1, 2\} \to \{3, 4\}$ \\
			$1 \mapsto 3$ \\
			$2 \mapsto 4$
	\end{center}		
		Ja, alle Zuordnungen einzeln auflisten geht auch (Formeln sind aber manchmal praktischer). \\
		Als Relation geschrieben ist $g = \{(1, 3), (2, 4)\}$. Diese Abbildung ist bijektiv.
	\end{exampleblock}
\end{frame}

