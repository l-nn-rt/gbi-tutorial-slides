% !TeX root = gbitut14.tex
%beamer

% This video for Turing:
% https://www.youtube.com/watch?v=VxvY4rI15sM

% Comment/uncomment this line to toggle handout mode
\newcommand{\handout}{}

\input{../framework/PraeambelTut.tex}

\morescalingdelimiters

\begin{document}
\starttut{14}

%\thasse{\lastframe{0.55}{20}{xkcd/turing_test.png}{https://www.xkcd.com/}}


\framePrevEpisode

\begin{frame}[t]{Wahr oder Falsch?}
	\Socrative
	\begin{figure}
		\begin{tikzpicture}[->,>=stealth,shorten >=1pt,auto,node distance=2cm,
		semithick,initial text={}]
		\tikzstyle{every state}=[]
		
		\node[state,accepting] (B)   {$Z_1$};
		\node[state]		 (M)  [right of=B]		{$Z_2$};
		
		\path
		(B) edge [loop above]  node {\word 1} (B) 
		(B) edge 			  node {\word 0} (M) 
		(M) edge [loop above] node {\word 0, \word 1} (M);
	\end{tikzpicture}
	\end{figure}
	
	\DependsQuestionE{Dieser Automat erkennt die Sprache $ \{\word 1\}^*$.} {Die Angabe des Startzustands fehlt. Startet man in $Z_2$, akzeptiert der Automat gar nichts.}
	% TODO: W/F Fragen?
\end{frame}

\begin{frame}{MIMA Programm erkennen}
	\begin{block}{Aufgabe}
		Was macht folgendes MIMA Programm?
		\begin{tabbing}
		$start$: 	\quad \quad \quad \quad \= STV x \\
											\> JMN $is\_negative$ \\
											\> JMP $fin$ \\
		$is\_negative$:						\> ADD x \\
											\> STV x \\
											\> LDC 1 \\
		$fin$:								\> ADD x \\
		\end{tabbing}
	\end{block}
\end{frame}
\begin{frame}{MIMA Programm erkennen}
	\begin{block}{Lösung}
		Das Programm führt die Umkerfunktion zum MIMA-Befehl \textbf{RAR} aus, also den Wert des Akkumulators um eine Stelle zyklisch nach link rotiert und ih zurück in den Akkumulator schreibt.
	\end{block}
\end{frame}

\begin{frame}{MIMA Programm erkennen 2}
	\begin{block}{Aufgabe}
		Was macht folgendes MIMA Programm?
		\begin{tabbing}
		$start$: 	\quad \quad \= LDV b \\
								\> NOT \\
								\> STV x \\
								\> LDC 1 \\
								\> ADD x \\
								\> ADD a \\
								\> JMN $addr_1$ \\
								\> JMP $addr_2$ \\
		\end{tabbing}
	\end{block}
\end{frame}
\begin{frame}{MIMA Programm erkennen 2}
	\begin{block}{Lösung}
		Das Programm führt einen bedingten Sprung aus, wenn der Wert der Speicherstelle $a$ (echt) kleiner ist als der Wert an der Speicherstelle $b$ (jeweils als Zweierkomplement interpretiert).
	\end{block}
\end{frame}

\newcommand{\inferrule}[2]{\displaystyle\frac{\quad #1 \quad}{\quad #2 \quad}}
\newcommand{\rulename}[1]{{~\textsf{\scriptsize(#1)}}}
\newcommand{\schlussregeln}{
	\centering
	\begin{displaymath}
		\begin{array}
			{c@{\hspace{2cm}}c} \displaystyle
			\inferrule{A \quad B}{A \aland B} \rulename{$\aland$I}
			 & \inferrule{A \aland B}{A}\rulename{$\aland$E$_l$} \quad
			\inferrule{A \aland B}{B} \rulename{$\aland$E$_r$}                                \\[2ex]
			\inferrule{A}{A \alor B}\rulename{$\alor$I$_l$} \quad
			\inferrule{B}{A \alor B}\rulename{$\alor$I$_r$}
			 & \inferrule{A \alor B\quad A \alimpl C \quad B \alimpl C}{C}\rulename{$\alor$E} \\[2ex]
			\inferrule{A \alvdash B}{A \alimpl B}\rulename{$\alimpl$I}
			 & \inferrule{A \quad A \alimpl B}{B}\rulename{MP}\rulename{$\alimpl$E}           \\[2ex]
			\inferrule{A \alimpl \alfalse}{\alnot A}\rulename{$\alnot$I}
			 & \inferrule{A \quad \alnot A}{\alfalse}\rulename{$\alnot$E} \quad
			\inferrule{\alnot \alnot A}{A}\rulename{$\alnot\alnot$E}
			\\[2ex]
			\inferrule{\alnot A \alimpl \alfalse}{A}\rulename{RAA}
			 & \inferrule{}{A \alvdash A}\rulename{Ax}
		\end{array}
	\end{displaymath}
}

\begin{frame}{Natürliches Schließen}
	\begin{block}{Aufgabe}
		Zeigen Sie mittels natürlichem Schließen: $ A \alor (B \aland C) \alimpl (A \alor B) \aland (A \alor C) $
	\end{block}
		\schlussregeln
\end{frame}

\begin{frame}{Natürliches Schließen Lösung}
	\vspace{3cm}
	%left bottom right top
	\includepdf[pages=11,clip=true,trim=100 320 100 180, scale=1]{woche_04_korrektur.pdf}
\end{frame}

\begin{frame}{Mehr Übungsaufgaben zu natürlichem Schließen}
	\url{https://formal.kastel.kit.edu/~ulbrich/natuerlich/}
\end{frame}

\begin{frame}{Reguläre Sprachen WS 2022}
	\includepdf[pages=11,clip=true,trim=20 530 20 100, scale=1]{kl-2022-ws.pdf}
\end{frame}

\begin{frame}{Reguläre Sprachen WS 2022}
	\includepdf[pages=13,clip=true,trim=20 620 20 100, scale=1]{kl-2022-ws.pdf}
\end{frame}

\begin{frame}{Reguläre Sprachen WS 2022}
	\includepdf[pages=11,clip=true,trim=20 80 20 440, scale=0.7]{kl-2022-ws.pdf}
\end{frame}

\begin{frame}{Reguläre Sprachen WS 2022}
	\includepdf[pages=13,clip=true,trim=20 500 20 220, scale=1]{kl-2022-ws.pdf}
\end{frame}

\begin{frame}{Reguläre Sprachen WS 2022}
	\includepdf[pages=12,clip=true,trim=20 700 20 50, scale=1]{kl-2022-ws.pdf}
\end{frame}

\begin{frame}{Reguläre Sprachen WS 2022}
	\vspace{3cm}
	\includepdf[pages=13,clip=true,trim=20 150 100 350, scale=0.9]{kl-2022-ws.pdf}
\end{frame}


% \begin{frame}[t]{Wahr oder Falsch?}
% 	\FalseQuestionE{Jede kontextfreie Grammatik lässt sich als regulärer Ausdruck \\ darstellen.}{Die Grammatik muss dafür rechtslinear sein, kontextfrei ist \enquote{zu mächtig}.}
% 	\TrueQuestion{Jeder reguläre Ausdruck lässt sich durch eine kontextfreie Grammatik darstellen.}
% 	\TrueQuestion{Für jede Sprache $L$ und jedes Wort $w \in L$ gilt: Es existiert ein endlicher Automat, der $w$ erkennt.}
% 	\FalseQuestion{Die Sprache der gültigen Klammerausdrücke ist regulär.}
% 	\TrueQuestionE{Die Sprache der gültigen Klammerausdrücke ist kontextfrei.}{}
% \end{frame}



\def\abbrsize{\footnotesize}

\appendix
\beginbackup

\section{Zusammenfassung und Ausblick}

\begin{frame}
	% \begin{block}{Was ihr nun wissen solltet}
	% 	\begin{itemize}
	% 		\item Rechtslineare Grammatiken
	% 		\item Reguläre Ausdrücke
	% 	\end{itemize}
	% \end{block}
	
	\begin{block}{Und so geht es weiter...}
		\vspace{-.3\baselineskip}
		\begin{itemize}
			\item Algorithmen I -- Mehr zu Algorithmen, Laufzeiten, Datenstrukturen, Graphen
			\item \textbf{T}{\abbrsize echnische} \textbf{I}{\abbrsize nformatik} -- Realisierung von Schaltungen, Prozessoren (MIMA, ...)
			\item \textbf{T}{\abbrsize heoretische} \textbf{G}{\abbrsize rundlagen der} \textbf{I}{\abbrsize nformatik} -- Mehr zu Grammatiken, Komplexität, Entscheidbarkeit, Turingmaschinen
		\end{itemize}
	\end{block}
\end{frame}

\thassedaniel{
	\begin{frame}{Falls ihr mehr wollt...}
		\begin{block}{Persönliche Empfehlungen}
			\begin{itemize}
				\item Design and Analysis of Algorithms (für Algorithmen I)
				\item CS50x
				\item From Nand to Tetris
				\item ICPC-Basispraktikum
			\end{itemize}
		\end{block}
		
		\begin{itemize}
			\item EDX (edx.org)
			\item Coursera (coursera.org)
		\end{itemize}
	\end{frame}
}{
\begin{frame}{Falls ihr mehr wollt...} % S. o.
	\begin{block}{Persönliche Empfehlungen}
		\begin{itemize}
			\item ICPC-Basispraktikum
		\end{itemize}
	\end{block}
\end{frame}
}


\begin{frame}{Das war GBI}
	\begin{columns}
		\pause
		\column{0.4\linewidth}
		\begin{figure}[H]
			\vspace{-20pt}
			\includegraphics[scale=0.45]{xkcd/heaven}
		\end{figure}
	
		\pause
		\column{0.5\linewidth}
		\begin{figure}[H]
			\vspace{-20pt}
			\includegraphics[scale=0.45]{xkcd/hell}
		\end{figure}
	\end{columns}
\end{frame}

\begin{headframe}[ Viel Erfolg  bei \\ euren Klausuren!\rlap{ \smiley}]
	--- The End ---
\end{headframe}



\only<handout:0>{\slideThanks}



%% Letzte Seite
% Scheint leider kein vernünftiges Abschieds-XKCD zu geben
\thasse{
	\xkcdframevert{287}{}{4.0}
	%\xkcdframe{5.8}{30}{xkcd/np_complete.png}{http://www.xkcd.com}{}
}
\only<beamer:0>{\slideThanks}

\backupend
\end{document}