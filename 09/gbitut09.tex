%beamer

% Comment/uncomment this line to toggle handout mode
% \newcommand{\handout}{}

\input{../framework/PraeambelTut.tex}

\morescalingdelimiters

\begin{document}
\starttut{9}

\framePrevEpisode


	%	\begin{frame}{Rückblick: Prädikatenlogik}
	%		\begin{itemize}[<+->]
	%			\item Deutlich komplizierterer Aufbau als Aussagenlogik
	%			\item Auswertung mit Interpretation und Variablenbelegung
	%			\item Quantoren erlauben allgemeine Aussagen
	%		\end{itemize}
	%	\end{frame}
	
\begin{frame}[t]{Wahr oder Falsch?}
    Sei $\RPL = \{\word R, \word S\}$ mit $\ar(\word R) = 2$ und $\ar(\word S) = 1$, \\
    $\FPL = \{\word f, \word g\}$ mit $\ar(\word f) = 1$ und $\ar(\word g) = 2$ . \\
    \TrueQuestionE{$\word{R(y,g(x,y))}$ ist präd.log. syntaktisch korrekt.}{}
    \FalseQuestionE{$\word{f(S(x))}$ ist präd.log. syntaktisch korrekt.}{Eine Relation kann nicht innerhalb einer Funktion auftauchen: Das geht nur mit Termen, nicht mit atomaren Formeln.}
    \FalseQuestionE{\enquote{Nicht alle Kinder spielen nicht.} $\equiv \plall \word{x\,(child(x)} \alimpl \word{play(x)}\plkz $}{Der Text sagt nur, dass es mindestens ein Kind gibt, das spielt.}
    \FalseQuestionE{\enquote{Wenn Person $a$ Person $b$ liebt, liebt $b$ nicht unbedingt $a$.} \\ $\equiv \plall \word a \plall \word b \, \plka \word{liebt(a,b)} \alimpl \alnot \word{liebt(b,a)}\plkz $}{Das hieße, dass es nur unerwiderte Gefühle gäbe. \textit{*schnief*} \\ Richtig wäre $\alnot \plall \word a \plall \word b \,\plka \word{liebt(a,b)} \alimpl \word{liebt(b,a)}\plkz $. 
    %\\ Umgeformt: „Es gilt nicht unbedingt, dass wenn $a$ $b$ liebt, dann auch $b$ $a$ liebt.“
    }
\end{frame}
\begin{frame}[t]{Wahr oder Falsch?}
    
    $F = \plexist \plx \, \plE{\plka \plx \plcomma \ply \plkz}$ \\
    \TrueQuestionE{$\fv(F) = \set{y}$}{}
    \FalseQuestionE{$\bv(F) = \emptyset$}{$\bv(F) = \set{\plx}$, $\plx$ ist durch $\plexist$ gebunden}
    \FalseQuestionE{$F$ ist geschlossen}{$\fv(F) \neq \emptyset$, eine Formel ist genau dann geschlossen, wenn keine freien Variablen vorkommen}

    \FalseQuestionE{$\sigma_{\set{\plx / \plz}}(F) = \plexist \plz \, \plE{\plka \plz \plcomma \ply \plkz}$}{Es dürfen nur freie Variablen substituiert werden}
	%TODO: Eine W/F-Frage zu PL Substitution
\end{frame}	



\input{../Bloecke/Hoare.tex}
% \input{../Bloecke/DrMetaNordpol}


\appendix
\beginbackup

\section{Zusammenfassung und Ausblick}

\begin{frame}
	\begin{block}{Was ihr nun wissen solltet}
		\begin{itemize}
			\item Wie der Hoare-Kalkül funktioniert
			\item Wie man mit dem Hoare-Kalkül ein Programm beweist.
		\end{itemize}
	\end{block}
	
	\begin{block}{Was nächstes Mal kommt}
		\begin{itemize}
			\item Graphen -- Alles vernetzt
			\item Systematisches Suchen und Wandern -- Algorithmen auf Graphen
		\end{itemize}
	\end{block}
\end{frame}

\only<handout:0>{\slideThanks}



%% Letzte Seite
\xkcdframe{704}{Danke für eure Aufmerksamkeit! \smiley}{2.5}
%\xkcdframevert{835}{Frohe Weihnachten und bis nächstes Jahr! \smiley}{2.2}
%\lastframetitled{0.47}{0}{xkcd/christmastree.png}{https://www.xkcd.com/835}{\vspace{-1.66\baselineskip}\\Frohe Weihnachten und einen \\ guten Start ins neue Jahr! \smiley}

\only<beamer:0>{\slideThanks}

\backupend
\end{document}