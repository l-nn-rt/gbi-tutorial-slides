% !TeX root = gbitut10.tex
%beamer

% Comment/uncomment this line to toggle handout mode
\newcommand{\handout}{}

\input{../framework/PraeambelTut.tex}

\morescalingdelimiters

\begin{document}
\starttut{10}



%\lastframe{0.42}{35}{xkcd/dfs_761.png}{https://www.xkcd.com/761}

\framePrevEpisode



\begin{frame}[t]{Wahr oder Falsch?}
	\Socrative
	\FalseQuestionE{Die Korrektheit eines Algorithmus kann man immer durch Testen beweisen.}{{\textbf{Nein, nein, nein!}} Durch Testen sieht man höchstens, wo Fehler sind, aber nicht, ob es keine gibt.}
	\FalseQuestionE{Sinnvolle Schleifeninvarianten kann man \enquote{nach Kochrezept} aufstellen.}{Nein, das Aufstellen sinnvoller Schleifeninvarianten erfordert viel Übung und kann insbesondere \textbf{nicht} vom Rechner gemacht werden.}
	\FalseQuestionE{Ein Algorithmus ist ein (kompilierbares/ausführbares) Programm.}{Nein, der Algorithmus selbst ist nur die Beschreibung der (Rechen-)\\ Vorschriften, nicht die tatsächliche Umsetzung in Programmcode.}
	\mycomment{
		\TrueQuestion{$x = y \aland y = z \impl x = z$ ist allgemeingültig}
		\FalseQuestion{$x = z \impl x = y \aland y = z$ ist allgemeingültig}
	}
\end{frame}


\begin{frame}[t]{Wahr oder falsch?}
	\FalseQuestionE{HT-I wird auf Iterationen angewendet.}{HT-I bei \kw{if}-Verzweigungen.}
	\FalseQuestionE{Ein laut Kalkül gültiges Hoare-Tripel garantiert die Korrektheit \\ des Algorithmus.}{Es muss auch gezeigt werden, dass der Algorithmus terminiert.}
	%\TrueQuestion{Der Hoare-Kalkül hilft einem, die Korrektheit eines Algorithmus zu beweisen.}
	%\FalseQuestionE{Der Hoare-Kalkül besteht aus den Axiomen HT-A, HT-E, HT-I, \\ HT-W, HT-S.}{Nur HT-A ist Axiom, die anderen sind Schlussregeln.}
\end{frame}

\input{../Bloecke/Graphen}

\input{../Bloecke/Graphen2}


\begin{frame}{Übung: Adjazenzmatrix}
	\begin{columns}
		\column{0.4\linewidth}
		$$ A = 
		\begin{pmatrix} 
		1 & 1 & 0 & 1 & 0 & 0 \\ 
		0 & 0 & 0 & 1 & 0 & 0 \\ 
		0 & 1 & 1 & 1 & 0 & 0 \\ 
		0 & 0 & 1 & 0 & 0 & 0 \\
		1 & 0 & 0 & 0 & 0 & 0 \\
		0 & 0 & 0 & 0 & 0 & 0 \\
		\end{pmatrix} $$
		
		\column{0.4\linewidth}
		\only<7->{
			\begin{tikzpicture}[->,>=stealth,baseline=-5mm]
			\matrix[matrix of math nodes,nodes={draw,circle,minimum size=6mm,inner sep=2pt},row sep=13mm,column sep=10mm,ampersand replacement=\&]
			{
				|(0)| 0 \& |(1)| 1 \& |(2)| 2 \\
				|(4)| 4 \& |(3)| 3 \& |(5)| 5 \\
			};
			\draw  (0) -- (1);
			\draw  (0) -- (3);
			\draw  (4) -- (0);
			\draw  (1) -- (3);
			\draw  (2)  to [bend left] (3);
			\draw  (2) -- (1);
			\draw  (3) to [bend left] (2);
			\path  (2) edge [loop right] ();
			\path  (0) edge [loop left] ();
			\end{tikzpicture}
		}
	\end{columns}
	
	\bigskip
	Was kann man an der Adjazenzmatrix ablesen?
	\begin{itemize}
		\item Gerichtet oder ungerichtet? \\ \pause
		\impl gerichtet (weil $A$ nicht symm.) \pause
		\item Schlingen? \\ \pause
		\impl Auf der Diagonalen von $A$: \quad Knoten 0, 2 \pause
		\item Zusammenhängend? \\ \pause
		\impl Nein (5 ist isoliert). 
	\end{itemize}
\end{frame}

\input{../Bloecke/Graphen3Matrix}

% \input{../Bloecke/GrossO}

\appendix
\beginbackup

\section{Zusammenfassung und Ausblick}

\begin{frame}
	\begin{block}{Was ihr nun wissen solltet}
		\begin{itemize}
			\item Grundbegriffe der Graphen
			\item Zentrale Eigenschaften von Graphen
		\end{itemize}
	\end{block}
	
	\begin{block}{Was nächstes Mal kommt}
		\begin{itemize}
			\item Wie man Graphen darstellen kann
			\item Warum dauert das so lange? -- Laufzeitbetrachtungen
			%\item One to rule them all -- Das Master-Theorem
			%\item Alles nur von Hand? -- Hier kommen die Automaten!
		\end{itemize}
	\end{block}
\end{frame}


\only<handout:0>{\slideThanks}



%% Letzte Seite
\xkcdframe{974}{Danke für eure Aufmerksamkeit! \smiley}{2.5}
%\lastframe{0.65}{0}{xkcd/the_general_problem_974.png}{http://www.xkcd.com/974}
\only<beamer:0>{\slideThanks}

\backupend
\end{document}