\documentclass[12pt,twoside]{report}
\setcounter{errorcontextlines}{4}

\setlength{\textheight}{236mm}
\addtolength{\topmargin}{-20mm}
\addtolength{\textwidth}{5mm}
\addtolength{\oddsidemargin}{5mm}
\addtolength{\evensidemargin}{-10mm}

%% %%%%%%%%% PACKAGES
%\usepackage{../../skript/gbi}
\usepackage{amsmath}
\usepackage{amssymb}
\usepackage{enumerate}
\usepackage{fancyhdr}
\usepackage{german}
\usepackage{graphicx}
\usepackage[latin1]{inputenc}
%% %\usepackage{latexsym}
\usepackage{mdwlist}
%\usepackage{thwmathligs}
\usepackage{thwpalatino}
%\usepackage{mdwlist}
\usepackage{pst-all}
\usepackage{tabularx}
%% \usepackage{thwalg}
%\usepackage{thwmathabbrevs}
%\usepackage{thwtextabbrevs}
% \usepackage{url}

\pagestyle{fancy}
\addtolength{\headheight}{5pt}
\fancyhead{}
\fancyfoot{}
\fancyhead[RO]{Matr.-Nr.:\hspace*{40mm}}
\fancyhead[LO]{Name:}
\fancyfoot[CE,CO]{\thepage}

\usepackage[digits]{at}
\usepackage{cmtt}
\newatcommand N[1]{\ensuremath{\langle}\hbox{\textit{#1}}\ensuremath{\rangle}}
\newatcommand T[1]{\hbox{\upshape\mttfamily\bfseries{#1}}}
\newatcommand M[1]{\ensuremath{\big#1}}
\newatcommand .{\ensuremath{\mathop{\hbox{\textbullet}}\,}}
\let\Hash=\#
\renewcommand{\#}[1]{\textup{\texttt{#1}}}
\newcommand{\act}{\ensuremath{\mathit{act}}}
\newcommand{\goto}{\ensuremath{\mathit{goto}}}
\def\strich{\underline{\hspace*{1cm}}}
\def\Strich{\underline{\hspace*{2cm}}}

\def\JANEIN{\qquad ja: \strich\qquad nein: \strich }

\newcounter{afgnr}
\setcounter{afgnr}{1}

\parskip=12pt
\parindent=0pt

\newenvironment{aufgabe}[1][\relax]%
  {\clearpage
    \noindent\textbf{Aufgabe \arabic{afgnr}}%
    \if#1\relax\else{ (#1 Punkte)}\fi
    \\
  }%
  {\addtocounter{afgnr}{1}}%

\begin{document}

\thispagestyle{empty}

\begin{center}
  \bfseries \Large
  Klausur zur Vorlesung \\
  Grundbegriffe der Informatik \\
  10. M�rz 2010 \\[8mm]
  \begin{tabular}{l@{\quad}|c|c|c|}
    \cline{2-4}
    Klausur- &
    \hbox to 20mm{\hss \vrule width 0pt height 8mm depth 1mm}&
    \hbox to 20mm{\hss \vrule width 0pt height 8mm depth 1mm}&
    \hbox to 20mm{\hss \vrule width 0pt height 8mm depth 1mm}\\
    nummer &
    \hbox to 20mm{\hss \vrule width 0pt height 6mm depth 5mm}&
    \hbox to 20mm{\hss \vrule width 0pt height 6mm depth 5mm}&
    \hbox to 20mm{\hss \vrule width 0pt height 6mm depth 5mm}\\
    \cline{2-4}
  \end{tabular}

\end{center}

\vspace*{10mm}
\def\Q{\hspace*{12mm}}
\begin{center}
  \begin{tabular}[t]{|l|*{8}{c|}}
    \cline{1-5}
    \multicolumn{5}{|l|}{} \\
    \multicolumn{5}{|l|}{Name:} \\
    \multicolumn{5}{|l|}{} \\
    \cline{1-5}
    \multicolumn{5}{|l|}{} \\
    \multicolumn{5}{|l|}{Vorname:} \\
    \multicolumn{5}{|l|}{} \\
    \cline{1-5}
    \multicolumn{5}{|l|}{} \\
    \multicolumn{5}{|l|}{Matr.-Nr.:} \\
    \multicolumn{5}{|l|}{} \\
%     \cline{1-5}
%     \multicolumn{2}{|l|}{} & & & \\
%     \multicolumn{2}{|l|}{Klausur.-Nr.:} & & &\\
%     \multicolumn{2}{|l|}{} & & & \\
    \cline{1-5}
    \multicolumn{7}{c}{ } \\
    \multicolumn{7}{c}{ } \\
    \multicolumn{7}{c}{ } \\
    \multicolumn{7}{c}{ } \\
    \cline{1-7}
    \            &   &   &   &   &   &     \\
    Aufgabe      & 1 & 2 & 3 & 4 & 5 & 6   \\
    \            & \Q& \Q& \Q& \Q& \Q& \Q  \\
    \cline{1-7}
    \            &   &   &   &   &   &      \\
    max. Punkte  & 6 & 6 & 7 & 7 & 8 & 11    \\
    \            &   &   &   &   &   &      \\
    \cline{1-7}
    \            &   &   &   &   &   &      \\
    tats. Punkte &   &   &   &   &   &      \\
    \            &   &   &   &   &   &      \\
  \cline{1-7}
    \multicolumn{6}{c}{ } \\
    \multicolumn{6}{c}{ } \\
    \multicolumn{6}{c}{ } \\
    \multicolumn{7}{c}{ } \\
    \cline{1-4}\cline{6-7}
    \multicolumn{4}{|l|}{ } &
    \multicolumn{1}{c}{ } &
    \multicolumn{2}{|l|}{ } & \multicolumn{1}{c}{ }\\
    \multicolumn{4}{|l|}{Gesamtpunktzahl:} &
    \multicolumn{1}{c}{ } &
    \multicolumn{2}{|l|}{Note:} & \multicolumn{1}{c}{ }\\
    \multicolumn{4}{|l|}{ } &
    \multicolumn{1}{c}{ } &
    \multicolumn{2}{|l|}{ } & \multicolumn{1}{c}{ } \\
    \cline{1-4}\cline{6-7}
  \end{tabular}
\end{center}


%=======================================================================
\begin{aufgabe}[2+2+2 = 6]
  In dieser Aufgabe geht es um Abbildungen; f�r $n \in \mathbb{N}_+$ gelte 
  wie in der Vorlesung: $\mathbb{G}_n=\{0, 1, \ldots, n-1\}$.

  \begin{enumerate}[a)]
  \item Sei $n \ge 1$. Wie viele Abbildungen gibt es von einer $n$-elementigen Menge
    in eine 2-elementige Menge, die \textbf{nicht} surjektiv sind?
  \item Geben Sie zwei Zahlen $n, m \in \mathbb{N}_+$ an, f�r die gilt:
    Es gibt mehr injektive Abbildungen von $\mathbb{G}_n$ nach $\mathbb{G}_m$
    als surjektive Abbildungen von $\mathbb{G}_m$ nach $\mathbb{G}_n$.

    \textbf{Hinweis:} Achten Sie auf die Indizes!
  \item Geben Sie zwei Zahlen $n, m \in \mathbb{N}_+$ an, f�r die gilt:
    Es gibt mehr surjektive Abbildungen von $\mathbb{G}_n$ nach $\mathbb{G}_m$
    als injektive Abbildungen von $\mathbb{G}_m$ nach $\mathbb{G}_n$.
  
    \textbf{Hinweis:} Achten Sie auf die Indizes!
     
  \end{enumerate}
 
%\clearpage
%\textit{Weiterer Platz f�r Antworten zu Aufgabe \theafgnr:}
\end{aufgabe}
%=======================================================================
\begin{aufgabe}[2+2+2 = 6]
  In dieser Aufgabe geht es um Huffman-Codierungen.

  Gegeben sei ein Wort �ber dem Alphabet $A=\{\#a, \#b, \#c, \#d\}$ mit
  folgenden \textbf{relativen} H�ufigkeiten:

  \begin {tabular}{c|c|c|c}
    $\#a$&$\#b$&$\#c$&$\#d$ \\ \hline
    \hphantom{.}&&&\\[-4mm]
    $x$&$\frac{1}{4}$&$\frac{1}{4}$&$\frac{1}{2}-x$
  \end {tabular}

  wobei $0 \le x \le \frac{1}{4}$ gilt.

  \begin{enumerate}[a)]
  \item Erstellen Sie den Huffman-Baum f�r $x=\frac{1}{16}$.
  \item Welche Struktur muss der Huffman-Baum haben, damit die
    Huffman-Codierung eines Wortes $w \in A^+$ mit
    den in der Tabelle angegebenen relativen H�ufigkeiten
    echt k�rzer als $2|w|$ sein kann?
  \item F�r welche $x \in \mathbb{R}$ mit $0 \le x \le \frac{1}{4}$
    werden W�rter mit den angegebenen relativen H�ufigkeiten auf
    genau doppelt so lange W�rter �ber $\{0, 1\}$ abgebildet?
      
  \end{enumerate}
 
\clearpage
\textit{Weiterer Platz f�r Antworten zu Aufgabe \theafgnr:}
\end{aufgabe}


%=======================================================================
\begin{aufgabe}[2+2+3 = 7]
  In dieser Aufgabe geht es um Mealy-Automaten.

  \begin{enumerate}[a)]
  \item Sei $X=\{\#0, \#1, \#2, \#3\}$ und $w=\#{2103} \in X^*$.
    Geben Sie ein Wort $w' \in \{\#0, \#1\}^*$ an, so dass
    $Num_2(w')=Num_4(w)$ gilt.
  \item Geben Sie einen Mealy-Automaten $A=(Z, z_0, X, f, Y, g)$
    mit $|Z|\leq 3, X=\{\#0, \#1, \#2, \#3\}$ und $Y=\{\#0, \#1\}$ an, so dass f�r
    alle W�rter $w \in X^*$ gilt:\\
    $Num_4(w)=Num_2(g^{**}(z_0, w))$.

      
  \item Geben Sie einen Mealy-Automaten $A=(Z, z_0, X, f, Y, g)$
    mit $|Z|\leq 3, X=\{\#0, \#1\}$ und $Y=\{\#0, \#1, \#2, \#3\}$ an, so dass f�r
    alle W�rter $w \in X^*$ \textbf{mit gerader L�nge} 
    gilt: $Num_2(w)=Num_4(g^{**}(z_0, w))$.
  \end {enumerate}
  
  \textbf{Hinweis 1:} $g^{**}(z_0, w)$ ist die Konkatenation aller Ausgaben,
  die $A$ bei Eingabe von $w$ erzeugt.
  
  \textbf{Hinweis 2:} Wenn Ihre Automaten mehr als drei Zust�nde
  haben, bekommen Sie weniger Punkte.
  
\clearpage
\textit{Weiterer Platz f�r Antworten zu Aufgabe \theafgnr:}
\end{aufgabe}



%=======================================================================
\begin{aufgabe}[4+2+1 = 7]
  Es sei die kontextfreie Grammatik $G=(\{S\}, \{\#a, \#b\}, S, 
  \{S \rightarrow \#aS\#a \mid \#aS \mid \#b\})$ gegeben.
  
  \begin{enumerate}[a)]
  \item Zeigen Sie durch vollst�ndige Induktion, dass f�r alle $k \in \mathbb{N}_0$ gilt:

    Wenn $S \Rightarrow^kw$ gilt, dann auch $\exists n, m \in
    \mathbb{N}_0: n \ge m \wedge w \in \{\#a^nS\#a^m, \#a^nb\#a^m\}$.
  \item Seien $n, m \in  \mathbb{N}_0$ mit $n \ge m$ gegeben.

    Erkl�ren Sie, wie man das Wort $\#a^n\#b\#a^m$ aus $S$ ableiten kann.

  \item Geben Sie eine mathematische Beschreibung von $L(G)$ an.

    \textbf{Hinweis:} Abwandlungen von $L(G)=\{w \in X^* \mid S \Rightarrow^* w\}$
    geben \textbf{keine} Punkte!

  \end{enumerate}

\clearpage
\textit{Weiterer Platz f�r Antworten zu Aufgabe \theafgnr:}
\end{aufgabe}

%=======================================================================
\begin{aufgabe}[2+3+1+2 = 8]
  Sei $G=(V, E)$ ein gerichteter Graph.

  Die Relation $S \subseteq V \times V$ sei gegeben durch 
  $\forall x, y \in V: xSy \iff$ es gibt in $G$ einen Pfad von $x$ nach $y$
  und es gibt in $G$ einen Pfad von $y$ nach $x$.

  Die Relation $R \subseteq V \times V$ sei gegeben durch 
  $\forall x, y \in V: xRy \iff$ es gibt in $G$ einen Pfad von $x$ nach $y$.

  \begin{enumerate}[a)]
    \item Geben Sie die Relation $S$ f�r folgenden Graphen $G$ an:

      \psset{xunit=2.5cm, yunit=2.5cm, runit=2.5cm}
      \begin {center}
        \begin {pspicture}(0, 0)(4, 2)
          
          \cnodeput(1, 0.5){A}{$A$}
          \cnodeput(2, 0.5){B}{$B$}
          \cnodeput(3, 0.5){D}{$C$}
          \cnodeput(1.5, 1.5){C}{$D$}
          
          \ncline[arrowscale=2]{->}{A}{B}
          \ncline[arrowscale=2]{->}{B}{C}
          \ncline[arrowscale=2]{->}{C}{A}
          \ncline[arrowscale=2]{->}{D}{B}
          
          
        \end {pspicture}
      \end {center}
      

  \item  Zeigen Sie, dass $S$ f�r beliebige gerichtete Graphen $G$ eine �quivalenzrelation ist.

  \item F�r welche Graphen $G$ gibt es nur eine �quivalenzklasse bez�glich $S$?

  \item 
    Zeigen Sie: F�r alle $x_1, x_2, y_1, y_2 \in V$ gilt: $x_1Sx_2 \wedge y_1Sy_2 \wedge x_1Ry_1 \Rightarrow
    x_2Ry_2$.
   

  \end{enumerate}
\clearpage
\textit{Weiterer Platz f�r Antworten zu Aufgabe \theafgnr:}
\end{aufgabe}

%=======================================================================
\begin{aufgabe}[1+1+1+2+2+2+2 = 11]
  Gegeben sei die folgende Turingmaschine $T$:
  \begin{itemize*}
  \item Zustandsmenge ist $Z=\{z_0, z_1, z_2, z_3, z_4, z_5\}$.
  \item Anfangszustand ist $z_0$.
  \item Bandalphabet ist $X=\{\Box,\#a,\#b \}$.
  \item Die Arbeitsweise ist wie folgt festgelegt:

     \begin{tabular}{c|cccccc}
       &$z_0$&$z_1$&$z_2$&$z_3$&$z_4$&$z_5$\\ \hline
       \  \\
       $\#a$& $(z_1, \Box, 1)$ & $(z_1, \#a, 1)$ &$(z_3, \#a, -1)$ & - &$(z_4, \#a, -1)$&$(z_5, \#b, 1)$\\
       $\#b$& $(z_5, \#a, 1)$ & $(z_2, \#b, 1)$ &$(z_2, \#b, 1)$ &$(z_4, \#a, -1)$ &$(z_4, \#b, -1)$&$(z_5, \#a, 1)$\\
       $\Box$ & - & - &$(z_3, \Box, -1)$ & - &$(z_0, \Box, 1)$ & $(z_4, \Box, -1)$\\
    \end {tabular}

    (Darstellung als Graph auf der n�chsten Seite)

        
  \end{itemize*}
  
  Die Turingmaschine wird im folgenden f�r Eingaben $w \in \{\#a^n\#b^m \mid n, m \in \mathbb{N}_0\}$
  verwendet, wobei der Kopf der Turingmaschine anfangs auf dem ersten Zeichen von $w$
  stehe (sofern $w$ nicht das leere Wort ist).

  \begin{enumerate}[a)]
  \item Geben Sie die Endkonfiguration der Turingmaschine
    f�r die Eingabe $w=\#{aaabb}$ an.
  \item Die Eingabe sei $w=\#{aaabbbbb}$. Geben Sie die Bandbeschriftung an,
    wenn $T$ das erste Mal von Zustand $z_5$ in den Zustand $z_4$ �bergeht.
  \item Beschreiben Sie, was $T$ macht, wenn $T$ sich im Zustand $z_4$
    befindet (bis sich der Zustand von $T$ �ndert).
  \item Seien $n, m\in \mathbb{N}_0$ mit  $n \ge m$ und die Eingabe $w=\#a^n\#b^m$.
    Welches Wort steht am Ende der Berechnung auf dem Band?
  \item Seien $n, m\in \mathbb{N}_0$ mit $n<m$ und die Eingabe $w=\#a^n\#b^m$.
    Welches Wort steht auf dem Band zu dem Zeitpunkt,
    an dem $T$ zum ersten Mal von Zustand $z_5$ in den Zustand $z_4$ wechselt?
  \item Seien $n, m\in \mathbb{N}_+$ mit $n < m$.
    Geben Sie Zahlen $n', m' \in \mathbb{N}_0$ mit $n'+m'<n+m$ an,
    so dass gilt:

    Bei Eingabe von $\#a^n\#b^m$ ist am Ende der Berechnung das Band leer $\iff$
    Bei Eingabe von $\#a^{n'}\#b^{m'}$ ist am Ende der Berechnung das Band leer.

  \item Geben Sie vier verschiedene Paare $(n, m) \in \mathbb{N}_+ \times \mathbb{N}_+$ an,
    f�r die gilt: Bei Eingabe von $\#a^n\#b^m$ ist am Ende der Berechnung das Band leer.
    
  \end{enumerate}

\clearpage
\textit{Weiterer Platz f�r Antworten zu Aufgabe \theafgnr:}

Darstellung der Turingmaschine als Graph:

\vspace*{-6\baselineskip}

\psset{xunit=2.5cm, yunit=2.5cm, runit=2.5cm}
\begin{center}
      \begin {pspicture}(0, 0)(4, 6)
        \pnode(0.6, 3){X}
        
        \cnodeput(1, 3){A}{$z_0$}
        \cnodeput(1, 4){B}{$z_1$}
        \cnodeput(2, 4){C}{$z_2$}
        \cnodeput(3, 3){D}{$z_3$}
        \cnodeput(2, 3){E}{$z_4$}
        \cnodeput(1, 2){F}{$z_5$}
        
        \ncline[arrowscale=2]{->}{X}{A}
        
        \ncline[arrowscale=2]{->}{A}{B}
        \naput{$\#a | \Box R$}
        \ncline[arrowscale=2]{->}{B}{C}
        \naput{$\#b | \#b R$}
        \ncline[arrowscale=2]{->}{C}{D}
        \naput{$\#a | \#a L, \Box | \Box L$}
        \ncline[arrowscale=2]{->}{D}{E}
        \naput{$\#b | \#a L$}
        \ncline[arrowscale=2]{->}{E}{A}
        \naput{$\Box | \Box R$}
        \ncline[arrowscale=2]{->}{A}{F}
        \nbput{$\#b | \#a R$}
        \ncline[arrowscale=2]{->}{F}{E}
        \nbput{$\Box | \Box L$}

        \nccircle[arrowscale=2]{->}{B}{0.2}
        \nbput{$\#a | \#a R$} 
        \nccircle[arrowscale=2]{->}{C}{0.2}
        \nbput{$\#b | \#b R$} 
        \nccircle[arrowscale=2]{->}{E}{0.2}
        \nbput{$\#a | \#a L, \#b | \#b L$} 
        \nccircle[arrowscale=2, angle=180]{->}{F}{0.2}
        \nbput{$\#b | \#a R, \#a | \#b R$}
      \end {pspicture}
    \end {center}

\clearpage
\textit{Weiterer Platz f�r Antworten zu Aufgabe \theafgnr:}


\end{aufgabe}
%=======================================================================
\end{document}
%%% Local Variables:
%%% TeX-command-default: "XPDFLaTeX"
%%% End:
