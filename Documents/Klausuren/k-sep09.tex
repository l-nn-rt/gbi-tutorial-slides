\documentclass[12pt,twoside]{report}
\setcounter{errorcontextlines}{4}

\setlength{\textheight}{236mm}
\addtolength{\topmargin}{-20mm}
\addtolength{\textwidth}{5mm}
\addtolength{\oddsidemargin}{5mm}
\addtolength{\evensidemargin}{-10mm}

%% %%%%%%%%% PACKAGES
%\usepackage{../../skript/gbi}
\usepackage{amsmath}
\usepackage{amssymb}
\usepackage{enumerate}
\usepackage{fancyhdr}
\usepackage{german}
\usepackage{graphicx}
\usepackage[latin1]{inputenc}
%% %\usepackage{latexsym}
\usepackage{mdwlist}
%\usepackage{thwmathligs}
%\usepackage{thwpalatino}
%\usepackage{mdwlist}
\usepackage{tabularx}
%% \usepackage{thwalg}
%\usepackage{thwmathabbrevs}
%\usepackage{thwtextabbrevs}
% \usepackage{url}

\pagestyle{fancy}
\fancyhead{}
\fancyfoot{}
\fancyhead[RO]{Matr.-Nr.:\hspace*{40mm}}
\fancyhead[LO]{Name:}
\fancyfoot[CE,CO]{\thepage}

\usepackage[digits]{at}
\usepackage{cmtt}
\newatcommand N[1]{\ensuremath{\langle}\hbox{\textit{#1}}\ensuremath{\rangle}}
\newatcommand T[1]{\hbox{\upshape\mttfamily\bfseries{#1}}}
\newatcommand M[1]{\ensuremath{\big#1}}
\newatcommand .{\ensuremath{\mathop{\hbox{\textbullet}}\,}}
\let\Hash=\#
\renewcommand{\#}[1]{\textup{\texttt{#1}}}
\newcommand{\act}{\ensuremath{\mathit{act}}}
\newcommand{\goto}{\ensuremath{\mathit{goto}}}
\def\strich{\underline{\hspace*{1cm}}}
\def\Strich{\underline{\hspace*{2cm}}}

\def\JANEIN{\qquad ja: \strich\qquad nein: \strich }

\newcounter{afgnr}
\setcounter{afgnr}{1}

\parskip=12pt
\parindent=0pt

\newenvironment{aufgabe}[1][\relax]%
  {\clearpage
    \noindent\textbf{Aufgabe \arabic{afgnr}}%
    \if#1\relax\else{ (#1 Punkte)}\fi
    \\
  }%
  {\addtocounter{afgnr}{1}}%

\begin{document}

\thispagestyle{empty}

\begin{center}
  \bfseries \Large
  Klausur zur Vorlesung \\
  Grundbegriffe der Informatik \\
  31. August 2009 \\[8mm]
  \begin{tabular}{l@{\quad}|c|c|c|}
    \cline{2-4}
    Klausur- &
    \hbox to 20mm{\hss \vrule width 0pt height 8mm depth 1mm}&
    \hbox to 20mm{\hss \vrule width 0pt height 8mm depth 1mm}&
    \hbox to 20mm{\hss \vrule width 0pt height 8mm depth 1mm}\\
    nummer &
    \hbox to 20mm{\hss \vrule width 0pt height 6mm depth 5mm}&
    \hbox to 20mm{\hss \vrule width 0pt height 6mm depth 5mm}&
    \hbox to 20mm{\hss \vrule width 0pt height 6mm depth 5mm}\\
    \cline{2-4}
  \end{tabular}

\end{center}

\vspace*{10mm}
\def\Q{\hspace*{12mm}}
\begin{center}
  \begin{tabular}[t]{|l|*{8}{c|}}
    \cline{1-5}
    \multicolumn{5}{|l|}{} \\
    \multicolumn{5}{|l|}{Name:} \\
    \multicolumn{5}{|l|}{} \\
    \cline{1-5}
    \multicolumn{5}{|l|}{} \\
    \multicolumn{5}{|l|}{Vorname:} \\
    \multicolumn{5}{|l|}{} \\
    \cline{1-5}
    \multicolumn{5}{|l|}{} \\
    \multicolumn{5}{|l|}{Matr.-Nr.:} \\
    \multicolumn{5}{|l|}{} \\
%     \cline{1-5}
%     \multicolumn{2}{|l|}{} & & & \\
%     \multicolumn{2}{|l|}{Klausur.-Nr.:} & & &\\
%     \multicolumn{2}{|l|}{} & & & \\
    \cline{1-5}
    \multicolumn{8}{c}{ } \\
    \multicolumn{8}{c}{ } \\
    \multicolumn{8}{c}{ } \\
    \multicolumn{8}{c}{ } \\
    \cline{1-8}
    \            &   &   &   &   &   &   &      \\
    Aufgabe      & 1 & 2 & 3 & 4 & 5 & 6 & 7   \\
    \            & \Q& \Q& \Q& \Q& \Q& \Q& \Q  \\
    \cline{1-8}
    \            &   &   &   &   &   &   &      \\
    max. Punkte  & 4 & 4 & 6 & 8 & 6 & 8 & 8    \\
    \            &   &   &   &   &   &   &      \\
    \cline{1-8}
    \            &   &   &   &   &   &   &      \\
    tats. Punkte &   &   &   &   &   &   &      \\
    \            &   &   &   &   &   &   &      \\
  \cline{1-8}
    \multicolumn{7}{c}{ } \\
    \multicolumn{7}{c}{ } \\
    \multicolumn{7}{c}{ } \\
    \multicolumn{8}{c}{ } \\
    \cline{1-4}\cline{6-7}
    \multicolumn{4}{|l|}{ } &
    \multicolumn{1}{c}{ } &
    \multicolumn{2}{|l|}{ } & \multicolumn{1}{c}{ }\\
    \multicolumn{4}{|l|}{Gesamtpunktzahl:} &
    \multicolumn{1}{c}{ } &
    \multicolumn{2}{|l|}{Note:} & \multicolumn{1}{c}{ }\\
    \multicolumn{4}{|l|}{ } &
    \multicolumn{1}{c}{ } &
    \multicolumn{2}{|l|}{ } & \multicolumn{1}{c}{ } \\
    \cline{1-4}\cline{6-7}
  \end{tabular}
\end{center}

%=======================================================================
\begin{aufgabe}[1+1+2 = 4]
  In dieser Aufgabe geht um die formalen Sprachen
  \begin{align*}
    L_1 &= \{ \#a^k\#b^m \mid k,m\in\mathbb{N}_0 \wedge k \bmod 2=0 \wedge m \bmod 3=1\} \\
    L_2 &= \{ \#b^k\#a^m \mid k,m\in\mathbb{N}_0 \wedge k \bmod 2=1 \wedge m \bmod 3=0\} \\
  \end{align*}

  Geben Sie f�r jede der folgenden formalen Sprachen $L$ je einen
  regul�ren Ausdruck $R_L$ an mit $\langle R_L\rangle = L$.

  \begin{enumerate}[a)]
  \item $L=L_1$
  \item $L=L_1\cdot L_2$
  \item $L=L_1\cap L_2$
  \end{enumerate}
 
%\clearpage
%\textit{Weiterer Platz f�r Antworten zu Aufgabe \theafgnr:}
\end{aufgabe}

%=======================================================================
\begin{aufgabe}[1+1+1+1 = 4]
  In dieser Aufgabe geht es um Abbildungen.

  Im folgenden sei $n \ge 1$ eine positive ganze Zahl.
  \begin{enumerate}[a)]
  \item Wie viele Abbildungen gibt es von einer einelementigen Menge in eine $n$-elementige Menge?
  \item Wie viele Abbildungen gibt es von einer $n$-elementigen Menge in eine einelementige Menge?
  \item Wie viele injektive Abbildungen gibt es von einer $n$-elementigen Menge in eine einelementige Menge?
  \item Wie viele injektive Abbildungen gibt es von einer 2-elementigen Menge in eine 3-elementige Menge?
  
  \end{enumerate}
 
%\clearpage
%\textit{Weiterer Platz f�r Antworten zu Aufgabe \theafgnr:}
\end{aufgabe}

%=======================================================================
\begin{aufgabe}[3+3 = 6]
  Im folgenden sei $n\ge 1$ immer eine positive ganze Zahl.

  Gegeben sei eine nichtleere Menge $M$ mit einer Halbordnung $\sqsubseteq$ darauf.
  Eine Folge $(a_1, \ldots, a_n)$ in $M$ hei�e \textit{streng monoton fallend},
  falls gilt:

  $\forall i \in \{1, \ldots, n-1\}: a_{i+1} \sqsubseteq a_i \wedge a_{i+1} \ne a_i$.

  \begin{enumerate}[a)]
  \item\label{foo} Sei $(a_1, \ldots, a_n)$ eine streng monoton fallende Folge in $M$.
    Zeigen Sie, dass gilt:

    $\forall i \in \{1, \ldots, n\} \forall j \in \{1, \ldots, n\}:
    i<j \Rightarrow a_j \sqsubseteq a_i$.

    F�hren Sie dazu eine vollst�ndige Induktion �ber die Differenz $k=j-i$ durch!

  \item Zeigen Sie durch vollst�ndige Induktion �ber $n$: 
  
    $\forall n \in \mathbb{N}_+:$ Falls $M$ kein minimales Element enth�lt,
    gibt es eine streng monoton fallende Folge $(a_1, \ldots, a_n)$
    mit $n$ Elementen in $M$.

  \end{enumerate}

\clearpage
\textit{Weiterer Platz f�r Antworten zu Aufgabe \theafgnr:}
\end{aufgabe}

%=======================================================================
\begin{aufgabe}[3+2+3 = 8]
  In dieser Aufgabe geht es um endliche Akzeptoren mit Zustandsmenge
  $Z=\{z_0, z_1, z_2\}$ und Eingabealphabet $X=\{\#a, \#b\}$.

  \begin{enumerate}[a)]
  \item Kann die durch den regul\"aren Ausdruck $(\#{aaaa})*$ beschriebene 
    Sprache von einem Automaten mit der oben genannten Zustandsmenge $Z$ 
    und dem oben genannten Eingabealphabet $X$ erkannt werden?

    Begr�nden Sie Ihre Antwort.
  
  \item Geben Sie einen endlichen Akzeptor mit der oben genannten
    Zustandsmenge $Z$ und dem oben genannten Eingabealphabet $X$ an,
    der genau die W�rter $w\in X^*$ akzeptiert, f�r die gilt:
    
    Die Anzahl der \#a in dem Wort $w$ ist gerade und gr��er als 1.
  \item Geben Sie einen endlichen Akzeptor mit der oben genannten
    Zustandsmenge $Z$ und dem oben genannten Eingabealphabet $X$ an,
    der genau die W�rter $w\in X^*$ akzeptiert, die die folgenden drei Bedingungen
    erf�llen:
    
    \begin{itemize}
    \item $w$ beginnt mit \#a oder ist das leere Wort.
    \item In $w$ kommt nirgends das Teilwort \#{aa} vor.
    \item In $w$ kommt nirgends das Teilwort \#{bb} vor.
    \end{itemize}
      
  \end{enumerate}

\clearpage
\textit{Weiterer Platz f�r Antworten zu Aufgabe \theafgnr:}
\end{aufgabe}

%=======================================================================
\begin{aufgabe}[2+2+2 = 6]

  Gegeben sei die kontextfreie Grammatik $G=(\{S, X\}, \{\#a, \#b\}, S, 
  \{S \rightarrow \#aS\#a \mid \#bS\#b \mid \#aX\#b \mid \#bX\#a,
    X \rightarrow \#aX \mid \#bX \mid \varepsilon\}$ 
  
  \begin{enumerate}[a)]
  \item Geben Sie ein Wort der L�nge 5 an, das in $L(G)$ liegt,
    und ein Wort der L�nge 5, das nicht in $L(G)$ liegt.
  \item Geben Sie eine kontextfreie Grammatik $G'=(N, T, S', P)$ an,
    f�r die gilt: 

    $L(G')=\{\#a, \#b\}^* \setminus L(G)$
  \item Geben Sie eine Abbildung $g: \mathbb{N}_0 \rightarrow \{\#a, \#b\}^*$ an,
    so dass gilt:

    $\forall n \in \mathbb{N}_0: \forall m \in \mathbb{N}_0: g(n)\#a^m \in L(G) \iff n \ne m$
    
%  \item Geben Sie einen regul�ren Ausdruck an, der die Menge aller W�rter beschreibt,
%    die \textit{nicht} in $L(G)\cdot L(G')$ liegen.
  
  \end{enumerate}

\clearpage
\textit{Weiterer Platz f�r Antworten zu Aufgabe \theafgnr:}
\end{aufgabe}

%=======================================================================
\begin{aufgabe}[4+2+2 = 8]
  Sei $\mathcal {G}$ die Menge aller gerichteten Graphen, f�r die gilt:
  Jeder Knoten hat den Ausgangsgrad 1 und es gibt einen Knoten, von dem aus
  alle anderen Knoten �ber einen Weg erreichbar sind.

  \begin{enumerate}[a)]
  \item Zeichnen Sie alle Graphen aus $\mathcal{G}$ mit vier Knoten,
    von denen keine zwei Graphen isomorph sind.  

  \item Geben Sie f�r die H�lfte der dargestellten Graphen die Adjazenzmatrix an.
    Machen Sie deutlich, welche Adjazenzmatrix zu welchem Graphen geh�rt.

  \item Geben Sie f�r jeden dargestellten Graphen, f�r den Sie keine
    Adjazenzmatrix angegeben haben, die Wegematrix an. Machen Sie
    deutlich, welche Wegematrix zu welchem Graphen geh�rt.
  \end{enumerate}
\clearpage
\textit{Weiterer Platz f�r Antworten zu Aufgabe \theafgnr:}
\end{aufgabe}

%=======================================================================
\begin{aufgabe}[2+1+1+2+1+1 = 8]
  Gegeben sei die folgende Turingmaschine $T$:
  \begin{itemize*}
  \item Zustandsmenge ist $Z=\{z_0, z_1, z_2, z_3, z_4, z_e, z_a, z_b, z'_a, z'_b, f_1, f_2\}$.
  \item Anfangszustand ist $z_0$.
  \item Bandalphabet ist $X=\{\Box,\#a,\#b,\#{\Hash} \}$.
  \item Die Arbeitsweise ist wie folgt festgelegt:

     \begin{tabular}{c|ccccc}
       &$z_0$&$z_1$&$z_2$&$z_3$&$z_4$\\ \hline
       \#a& $(z_0, \#a, 1)$ & $(z_1, \#a, 1)$ &$(z_2, \#a, 1)$ &$(z_3, \#a, -1)$ &$(z_a, \Box, 1)$\\
       \#b& $(z_0, \#b, 1)$ & $(z_1, \#b, 1)$ &$(z_2, \#b, 1)$ &$(z_3, \#b, -1)$ &$(z_b, \Box, 1)$\\
       \#{\Hash}& $(z_1, \#{\Hash}, 1)$ & $(z_2, \#{\Hash}, 1)$ &$(z_2, \#{\Hash}, 1)$ &$(z_3, \#{\Hash}, -1)$ &$(z_e, \Box, 1)$\\
       $\Box$ & $(z_2, \Box, 1)$ & $(z_3, \Box, -1)$ &$(z_2, \Box, 1)$ &$(z_4, \Box, 1)$ &$(z_2, \Box, 1)$\\
    \end {tabular}
    
    \vspace*{\baselineskip}
 
    \begin{tabular}{c|ccccc}
       &$z_a$&$z_b$&$z'_a$ & $z'_b$ & $z_e$\\ \hline
       \#a&$(z_a, \#a, 1)$ & $(z_b, \#a, 1)$ & $(z_3, \#{\Hash}, -1)$ &$(f_1, \#a, 0)$ & $(f_1, \#a, 0)$ \\
       \#b&$(z_a, \#b, 1)$ & $(z_b, \#b, 1)$ & $(f_1, \#b, 0)$ &$(z_3, \#{\Hash}, -1)$ & $(f_1, \#b, 0)$ \\
       \#{\Hash}&$(z'_a, \#{\Hash}, 1)$ & $(z'_b, \#{\Hash}, 1)$ & $(z'_a, \#{\Hash}, 1)$ &$(z'_b, \#{\Hash}, 1)$ & $(z_e, \Box, 1)$ \\
       $\Box$&$(z_2, \Box, 1)$ & $(z_2, \Box, 1)$ & $(f_1, \Box, 0)$ &$(f_1, \Box, 0)$ & $(f_2, \Box, 0)$ \\
    \end {tabular}
    
  \end{itemize*}
  Die Turingmaschine wird im folgenden benutzt f�r Bandbeschriftungen,
  bei denen auf dem Band (von Blanksymbolen umgeben) ein Wort $w \in \{\#a, \#b, \#{\Hash}\}^*$
  steht.

  Der Kopf der
  Turingmaschine stehe auf dem ersten Symbol von $w\in
  \{\#a,\#b, \#{\Hash}\}^*$.

  Sei $\mathcal{L}$ die Menge aller W�rter $w \in \{\#a, \#b, \#{\Hash}\}^*$, f�r die gilt:
  $T$ h�lt bei Eingabe von $w$ im Zustand $f_2$.

  \begin{enumerate}[a)]
  \item Geben Sie einen regul�ren Ausdruck f�r die Menge aller W�rter $w$ an,
    f�r die $T$ bei Eingabe von $w$ irgendwann h�lt.
  \item Berechnen Sie die Endkonfiguration f�r die Eingabe $w=\#{aaab\Hash baa}$.
    Die Zwischenschritte der Berechnung m\"ussen \textit{nicht} angegeben werden.
  \item Berechnen Sie die Endkonfiguration f�r die Eingabe $w=\#{abbaa\Hash abba}$.
    Die Zwischenschritte der Berechnung m\"ussen \textit{nicht} angegeben werden.
  \item Geben Sie eine formale Beschreibung von $\mathcal{L}$ an, die nicht
    auf $T$ verweist.
  \item Sei $w \in \mathcal{L}$ die Eingabe von $T$. Welches Wort 
    $w'\in \{\#a, \#b, \#{\Hash}\}^*$
    steht auf dem
    Band, wenn sich $T$ im Zustand $f_2$ befindet?
  \item Geben Sie eine m�glichst einfache Funktion $g$ an, so dass gilt:

    Es gibt eine Funktion $f \in \Theta(g)$, so dass $T$ bei Eingabe eines Wortes $w \in \mathcal{L}$
    der L�nge $n$ genau $f(n)$ Schritte macht, bis $T$ h�lt.
  
  \end{enumerate}

\clearpage
\textit{Weiterer Platz f�r Antworten zu Aufgabe \theafgnr:}
\end{aufgabe}
%=======================================================================
\end{document}
%%% Local Variables:
%%% TeX-command-default: "XPDFLaTeX"
%%% End:
