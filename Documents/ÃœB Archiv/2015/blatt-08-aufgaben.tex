\documentclass[12pt]{article}

\input{preamble-aufgaben}
\DeclarePairedDelimiter\floor{\lfloor}{\rfloor}

% =======================================================
%\newcounter{blattnr}
\setcounter{blattnr}{8}
\newcommand{\ausgabetermin}{16.~Dezember 2015}
\newcommand{\abgabetermin}{8.~Januar 2015}
%\newcommand{\punkteblatt}{13} % Blatt 1
%\newcommand{\punkteblattphysik}{13} % Blatt 1
%\newcommand{\punkteblatt}{17} % Blatt 2
%\newcommand{\punkteblattphysik}{14} % Blatt 2
%\newcommand{\punkteblatt}{18} % Blatt 3
%\newcommand{\punkteblattphysik}{18} % Blatt 3
%\newcommand{\punkteblatt}{18} % Blatt 4
%\newcommand{\punkteblattphysik}{18} % Blatt 4
%\newcommand{\punkteblatt}{18} % Blatt 5
%\newcommand{\punkteblattphysik}{18} % Blatt 5
%\newcommand{\punkteblatt}{20} % Blatt 6
%\newcommand{\punkteblattphysik}{20} % Blatt 6
%\newcommand{\punkteblatt}{20} % Blatt 7
%\newcommand{\punkteblattphysik}{0} % Blatt 7
\newcommand{\punkteblatt}{18} % Blatt 8
\newcommand{\punktetotal}{142} %
\newcommand{\punkteblattphysik}{18} % Blatt 8
\newcommand{\punktetotalphysik}{119} %
% =======================================================

\begin{document}

%\noindent
%Mit \textbf{[nicht Physik]} gekennzeichnete Aufgaben müssen von
%Studenten der Physik nicht bearbeitet werden.\\

\noindent
\textbf{Vorbemerkung.} Für alle Aufgaben auf diesem Blatt gelten die
folgenden Annahmen, ohne dass sie jedes Mal erneut aufgeführt werden:
\begin{itemize}[nosep]
\item Die Menge der möglichen Werte für eine Variable ist $\Z$, sofern
  nicht ausdrücklich etwas anderes angegeben ist.
\item Alle Variablen sind initialisiert. Der Anfangswert ist aber
  nicht immer explizit angegeben.
\item Zu einer Anweisungsfolge $S$ und einer Nachbedingung $Q$ heißt
  $P$ eine \emph{schwächste Vorbedingung}, wenn $\htr{P}{S}{Q}$ ein
  gültiges Hoare"=Tripel ist und für jedes gültige Hoare"=Tripel
  $\htr{P'}{S}{Q}$ gilt: $P'$ impliziert $P$.
\item Zu einer Anweisungsfolge $S$ und einer Vorbedingung $P$ heißt
  $Q$ eine \emph{stärkste Nachbedingung}, wenn $\htr{P}{S}{Q}$ ein
  gültiges Hoare"=Tripel ist und für jedes gültige Hoare"=Tripel
  $\htr{P}{S}{Q'}$ gilt: $Q$ impliziert $Q'$.
\end{itemize}

\vspace*{\baselineskip}

% -----------------------------------------------------------------------------

\begin{aufgabe}[4]
  Es seien $x$ und $y$ zwei verschiedene Variablen und es seien $a$ und $b$ zwei ganze Zahlen. Bestimmen Sie anhand des Hoare-Kalküls eine schwächste Vorbedingung von
  \begin{align*}
    & x \gets x + y \\
    & x \gets x \cdot x \\
    & x \gets x - y \\
    & \{ x = a \land y = b \}
  \end{align*}
  indem Sie wiederholt das Zuweisungsaxiom verwenden.
\end{aufgabe}

\begin{loesung}
  Eine schwächste Vorbedingung ist ${((x + y) \cdot (x + y)) - y} = a \land y = b$.

  Gemäß des Zuweisungsaxioms sind die folgenden Hoare-Tripel ableitbar:
  \begin{align*}
    & \{ x - y = a \land y = b \} \\
    & x \gets x - y \\
    & \{ x = a \land y = b \}
  \end{align*}
  und
  \begin{align*}
    & \{ (x \cdot x) - y = a \land y = b \} \\
    & x \gets x \cdot x \\
    & \{ x - y = a \land y = b \}
  \end{align*}
  und
  \begin{align*}
    & \{ ((x + y) \cdot (x + y)) - y = a \land y = b \} \\
    & x \gets x + y \\
    & \{ {x \cdot x-y} = a \land y = b \}
  \end{align*}
  Gemäß der Sequenzenregel ist das folgende Hoare-Tripel ableitbar:
  \begin{align*}
    & \{ ((x + y) \cdot (x + y)) - y = a \land y = b \} \\
    & x \gets x + y \\
    & x \gets x \cdot x \\
    & x \gets x - y \\
    & \{ x = c \land y = d \}
  \end{align*}
  Somit ist $((x + y) \cdot (x + y)) - y = a \land y = b$ eine Vorbedingung. Und diese ist tatäschlich eine schwächste.

  Alternativ alles hintereinander weg:
  \begin{align*}
    & \{ ((x + y) \cdot (x + y)) - y = a \land y = b \} \\
    & x \gets x + y \\
    & \{ x  \cdot x - y = a \land y = b \} \\
    & x \gets x \cdot x \\
    & \{ x - y = a \land y = b \} \\
    & x \gets x - y \\
    & \{ x = a \land y = b \}
  \end{align*}

  Statt $\{ ((x + y) \cdot (x + y)) - y = a \land y = b \}$ kann
  natürlich auch etwas Äquivalentes stehen.

  \begin{korrektur}
    Was man \emph{nicht} möchte ist Einsetzen von oben nach unten:
    \begin{align*}
      & \{ x = -b \pm \sqrt{a+b} \land y = b \} \\
      & x \gets x + y \\
      & \{ x = \pm \sqrt{a+b} \land y = b \} \\
      & x \gets x \cdot x \\
      & \{ x =  a+b \land y = b \} \\
      & x \gets x - y \\
      & \{ x = a \land y = b \}
    \end{align*}

    Erlaubt ist, beim Einsetzen von unten nach oben die arithmetischen
    Ausdrücke ab und zu \emph{explizit} "`vereinfachen"', \dh zwei
    Zusicherungen hintereinander und die untere folgt aus der darüber.

    Vorschlag: je 1 Punkt für jede Zusicherung
  \end{korrektur}

\end{loesung}

% -----------------------------------------------------------------------------

\begin{aufgabe}[6]
  Es seien $x$, $y$ und $z$ drei verschiedene Variablen und es seien $a$ und $b$ zwei ganze Zahlen. Weiter sei
  \begin{align*}
    \min \colon \Z \times \Z &\to     \Z,\\
                      (u, v) &\mapsto \begin{dcases*}
                                        u, &falls $u < v$,\\
                                        v, &falls $u \geq v$,
                                      \end{dcases*}
  \end{align*}
  und es sei
  \begin{align*}
    \max \colon \Z \times \Z &\to     \Z,\\
                      (u, v) &\mapsto u + v - \min(u, v).
  \end{align*}
  Beweisen Sie anhand des Hoare-Kalküls, dass das Hoare-Tripel
  \begin{align*}
    & \{ x = a \land y = b \} \\
    & \kw{if\ } x > y \kw{\ then} \\
    & \qquad z \gets x \\
    & \qquad x \gets y \\
    & \qquad y \gets z \\
    & \kw{else} \\
    & \qquad x \gets x \\
    & \kw{fi} \\
    & \{ x = \min(a, b) \land y = \max(a, b) \}
  \end{align*}
  gültig ist.
\end{aufgabe}

\begin{loesung}
  Gemäß des Zuweisungsaxioms sind die folgenden Hoare-Tripel ableitbar:
  \begin{align*}
    & \{ x = \min(a, b) \land z = \max(a, b) \} \\
    & y \gets z \\
    & \{ x = \min(a, b) \land y = \max(a, b) \}
  \end{align*}
  und
  \begin{align*}
    & \{ y = \min(a, b) \land z = \max(a, b) \} \\
    & x \gets y \\
    & \{ x = \min(a, b) \land z = \max(a, b) \}
  \end{align*}
  und
  \begin{align*}
    & \{ y = \min(a, b) \land x = \max(a, b) \} \\
    & z \gets x \\
    & \{ y = \min(a, b) \land z = \max(a, b) \}
  \end{align*}
  Gemäß der Sequenzenregel ist das folgende Hoare-Tripel ableitbar:
  \begin{align*}
    & \{ y = \min(a, b) \land x = \max(a, b) \} \\
    & z \gets x \\
    & x \gets y \\
    & y \gets z \\
    & \{ x = \min(a, b) \land y = \max(a, b) \}
  \end{align*}
  Die Vorbedingung dieses Hoare-Tripels wird impliziert von $x = a \land y = b \land x > y$. Gemäß der Verstärkungs- und Abschwächungsregel ist das folgende Hoare-Tripel ableitbar:
  \begin{align*}
    & \{ x = a \land y = b \land x > y \} \\
    & z \gets x \\
    & x \gets y \\
    & y \gets z \\
    & \{ x = \min(a, b) \land y = \max(a, b) \}
  \end{align*}
  Gemäß des Zuweisungsaxioms ist das folgende Hoare-Tripel ableitbar:
  \begin{align*}
    & \{ x = \min(a, b) \land y = \max(a, b) \} \\
    & x \gets x \\
    & \{ x = \min(a, b) \land y = \max(a, b) \}
  \end{align*}
  Die Vorbedingung dieses Hoare-Tripels wird impliziert von $x = a \land y = b \land \neg (x > y)$. Gemäß der Verstärkungs- und Abschwächungsregel ist das folgende Hoare-Tripel ableitbar:
  \begin{align*}
    & \{ x = a \land y = b \land \neg (x > y) \} \\
    & x \gets x \\
    & \{ x = \min(a, b) \land y = \max(a, b) \}
  \end{align*}
  Gemäß der Verzweigungsregel ist das folgende Hoare-Tripel ableitbar:
  \begin{align*}
    & \{ x = a \land y = b \} \\
    & \kw{if\ } x > y \kw{\ then} \\
    & \qquad z \gets x \\
    & \qquad x \gets y \\
    & \qquad y \gets z \\
    & \kw{else} \\
    & \qquad x \gets x \\
    & \kw{fi} \\
    & \{ x = \min(a, b) \land y = \max(a, b) \}
  \end{align*}
  Da ausschließlich gültige Hoare-Tripel ableitbar sind, ist obiges Hoare-Tripel gültig.

  \begin{korrektur}
    Vorschlag: 4 Punkte für den \kw{then}-Teil und 2 für den
    \kw{else}-Teil.
  \end{korrektur}
\end{loesung}

% -----------------------------------------------------------------------------

\begin{aufgabe}[8]
  Für jede positive ganze Zahl $n \in \N_+$ ist der ganzzahlige binäre Logarithmus von $n$, geschrieben $\floor{\log_2 n}$, jene nicht-negative ganze Zahl $k \in \N_0$, für die $2^k \leq n < 2^{k + 1}$ gilt.

  Es seien $x$, $y$ und $z$ drei verschiedene Variablen. Beweisen Sie anhand des Hoare-Kalküls und mithilfe einer Schleifeninvariante, dass das Hoare-Tripel
  \begin{align*}
    & \{ x \geq 1 \} \\
    & z \gets 0 \\
    & y \gets 1 \\
    & \kw{while\ } 2 \cdot y \leq x \kw{\ do} \\
    & \qquad z \gets z + 1 \\
    & \qquad y \gets 2 \cdot y \\
    & \kw{od} \\
    & \{ z = \floor{\log_2 x} \}
  \end{align*}
  gültig ist.

  \emph{Hinweis:} Die Nachbedingung $z = \floor{\log_2 x}$ ist äquivalent zu $2^z \leq x < 2^{z + 1}$.
\end{aufgabe}

\begin{loesung}
  Gemäß des Zuweisungsaxioms sind die folgenden Hoare-Tripel ableitbar:
  \begin{align*}
    & \{ x \geq 1 \land 1 = 1 \land z = 0 \} \\
    & y \gets 1 \\
    & \{ x \geq 1 \land y = 1 \land z = 0 \}
  \end{align*}
  und
  \begin{align*}
    & \{ x \geq 1 \land 1 = 1 \land 0 = 0 \} \\
    & z \gets 0 \\
    & \{ x \geq 1 \land 1 = 1 \land z = 0 \}
  \end{align*}
  Gemäß der Sequenzenregel ist das folgende Hoare-Tripel ableitbar:
  \begin{align*}
    & \{ x \geq 1 \land 1 = 1 \land 0 = 0 \} \\
    & z \gets 0 \\
    & y \gets 1 \\
    & \{ x \geq 1 \land y = 1 \land z = 0 \}
  \end{align*}
  Die Vorbedingung dieses Hoare-Tripels wird impliziert von $x \geq 1$. Gemäß der Verstärkungs- und Abschwächungsregel ist das folgende Hoare-Tripel ableitbar:
  \begin{align*}
    & \{ x \geq 1 \} \\
    & z \gets 0 \\
    & y \gets 1 \\
    & \{ x \geq 1 \land y = 1 \land z = 0 \}
  \end{align*}
  Um zu beweisen, dass das Hoare-Tripel aus der Aufgabenstellung ableitbar, insbesondere gültig, ist, genügt es, gemäß der Sequenzenregel, zu beweisen, dass das folgende Hoare-Tripel ableitbar ist:
  \begin{align*}
    & \{ x \geq 1 \land y = 1 \land z = 0 \} \\
    & \kw{while\ } 2 \cdot y \leq x \kw{\ do} \\
    & \qquad z \gets z + 1 \\
    & \qquad y \gets 2 \cdot y \\
    & \kw{od} \\
    & \{ z = \floor{\log_2 x} \}
  \end{align*}
  Dazu raten wir die Schleifeninvariante $y = 2^z \land y \leq x$. Diese Formel ist tatsächlich eine Schleifeninvariante, da durch zweifache Anwendung des Zuweisungsaxioms und einfache Anwendung der Sequenzenregel folgt, dass
  \begin{align*}
    & \{ 2 \cdot y = 2^{z + 1} \land 2 \cdot y \leq x \} \\
    & z \gets z + 1 \\
%    & \{ 2 \cdot y = 2^z \land 2 \cdot y \leq x \}
    & y \gets 2 \cdot y \\
    & \{ y = 2^z \land y \leq x \}
  \end{align*}
  ableitbar ist, und durch einfach Anwendung der Verstärkungs- und Abschwächungsregel folgt, dass
  \begin{align*}
    & \{ y = 2^z \land y \leq x \land 2 \cdot y \leq x \} \\
    & z \gets z + 1 \\
    & y \gets 2 \cdot y \\
    & \{ y = 2^z \land y \leq x \}
  \end{align*}
  ableitbar ist. Gemäß der Schleifeninvariantenregel ist das folgende Hoare-Tripel ableitbar:
  \begin{align*}
    & \{ y = 2^z \land y \leq x \} \\
    & \kw{while\ } 2 \cdot y \leq x \kw{\ do} \\
    & \qquad z \gets z + 1 \\
    & \qquad y \gets 2 \cdot y \\
    & \kw{od} \\
    & \{ y = 2^z \land y \leq x \land \neg (2 \cdot y \leq x) \}
  \end{align*}
  Die Vorbedingung dieses Hoare-Tripels wird impliziert von $x \geq 1 \land y = 1 \land z = 0$ und die Nachbedingung impliziert $2^z \leq x < 2^{z + 1}$, also gemäß der Hinweises $z = \floor{\log_2 x}$. Gemäß der Verstärkungs- und Abschwächungsregel ist das folgende Hoare-Tripel ableitbar:
  \begin{align*}
    & \{ x \geq 1 \land y = 1 \land z = 0 \} \\
    & \kw{while\ } 2 \cdot y \leq x \kw{\ do} \\
    & \qquad z \gets z + 1 \\
    & \qquad y \gets 2 \cdot y \\
    & \kw{od} \\
    & \{ z = \floor{\log_2 x} \}
  \end{align*}

  Alternativ notiert alles hintereinander weg:

  \begin{align*}
    & \{ x \geq 1 \} \\
    & \{ 1=2^0 \land 1\leq x \}\\
    & z \gets 0 \\
    & \{ 1=2^z \land 1\leq x \}\\
    & y \gets 1 \\
    & \{ y=2^z \land y\leq x \}\\
    & \kw{while\ } 2 \cdot y \leq x \kw{\ do} \\
    & \qquad \{ y=2^z \land y\leq x \land 2\cdot y \leq x\}\\
    & \qquad \{ 2\cdot y=2^{z+1} \land 2\cdot y\leq x \}\\
    & \qquad z \gets z + 1 \\
    & \qquad \{ 2\cdot y=2^z \land 2\cdot y\leq x \}\\
    & \qquad y \gets 2 \cdot y \\
    & \qquad \{ y=2^z \land y\leq x \}\\
    & \kw{od} \\
    & \{ y=2^z \land y\leq x \land \lnot (2 \cdot y \leq x)\}\\
    & \{ y=2^z \land y\leq x \land x < 2 \cdot y \}\\
    & \{ 2^z \leq x < 2^{z + 1} \} \\
    & \{ z = \floor{\log_2 x} \}
  \end{align*}

  \begin{korrektur}
    Vorschlag: 
    \begin{itemize}
    \item 1 Punkt für den Teil vor dem \kw{while}
    \item 3 Punkte für vernünftige Schleifeninvariante, 
    \item 1 Punkt für richtige Benutzung der Schleifeninvariante, 
    \item 2 Punkte für den Rest des Schleifenrumpfes
    \item 1 Punkte für richtige Argumentation nach dem Schleifenrumpf
    \end{itemize}
  \end{korrektur}

\end{loesung}

% -----------------------------------------------------------------------------

\end{document}
%%%
%%% Local Variables:
%%% fill-column: 70
%%% mode: latex
%%% TeX-command-default: "XPDFLaTeX"
%%% TeX-master: "korrektur.tex"
%%% End:
