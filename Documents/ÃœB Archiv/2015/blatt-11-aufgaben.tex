\documentclass[12pt]{article}

\input{preamble-aufgaben}

\usetikzlibrary{positioning}
\usetikzlibrary{arrows,shapes}

% =======================================================
%\newcounter{blattnr}
\setcounter{blattnr}{11}
\newcommand{\ausgabetermin}{21.~Januar 2016}
\newcommand{\abgabetermin}{29.~Januar 2016}
%\newcommand{\punkteblatt}{13} % Blatt 1
%\newcommand{\punkteblattphysik}{13} % Blatt 1
%\newcommand{\punkteblatt}{17} % Blatt 2
%\newcommand{\punkteblattphysik}{14} % Blatt 2
%\newcommand{\punkteblatt}{18} % Blatt 3
%\newcommand{\punkteblattphysik}{18} % Blatt 3
%\newcommand{\punkteblatt}{18} % Blatt 4
%\newcommand{\punkteblattphysik}{18} % Blatt 4
%\newcommand{\punkteblatt}{18} % Blatt 5
%\newcommand{\punkteblattphysik}{18} % Blatt 5
%\newcommand{\punkteblatt}{20} % Blatt 6
%\newcommand{\punkteblattphysik}{20} % Blatt 6
%\newcommand{\punkteblatt}{20} % Blatt 7
%\newcommand{\punkteblattphysik}{0} % Blatt 7
%\newcommand{\punkteblatt}{18} % Blatt 8
%\newcommand{\punkteblattphysik}{18} % Blatt 8
%\newcommand{\punkteblatt}{17} % Blatt 9
%\newcommand{\punkteblattphysik}{17} % Blatt 9
%\newcommand{\punkteblatt}{16} % Blatt 10
%\newcommand{\punkteblattphysik}{16} % Blatt 10
\newcommand{\punkteblatt}{19} % Blatt 11
\newcommand{\punktetotal}{194} %
\newcommand{\punkteblattphysik}{19} % Blatt 11
\newcommand{\punktetotalphysik}{171} %
% =======================================================

\begin{document}

%\noindent
%Mit \textbf{[nicht Physik]} gekennzeichnete Aufgaben müssen von
%Studenten der Physik nicht bearbeitet werden.\\

\noindent
Im folgenden schreiben wir $[n\mapsto f(n)]$ für die Funktion
$f\from\N_0\to\R_0^+\from n \mapsto f(n)$. Statt
$\Oh{[n\mapsto f(n)]}$ schreiben wir kürzer $\Oh{n\mapsto f(n)}$ und
analog bei $\Om{\cdot}$ und $\Th{\cdot}$.
\\

% -----------------------------------------------------------------------------

\begin{aufgabe}[4]
  Beweisen Sie, dass $\Omega(n \mapsto 2^n) \cap O(n \mapsto n^2)$ die leere Menge ist.

  \emph{Hinweis:} Gemäß der Vorlesung gilt $[n \mapsto 2^n] \npreceq [n \mapsto n^2]$.
\end{aufgabe}

\begin{loesung}
  Angenommen $\Omega(n \mapsto 2^n) \cap O(n \mapsto n^2) \neq \{\}$. Dann gibt es ein $f \in \Omega(n \mapsto 2^n) \cap O(n \mapsto n^2)$. Wegen $f \in \Omega(n \mapsto 2^n)$, gibt es ein $c \in \R_+$ und ein $n_0 \in \N_0$ derart, dass
  \begin{equation*}
    \forall n \geq n_0 : f(n) \geq c \cdot 2^n.
  \end{equation*}
  Und wegen $f \in O(n \mapsto n^2)$, gibt es ein $c' \in \R_+$ und ein $n_0' \in \N_0$ derart, dass
  \begin{equation*}
    \forall n \geq n_0' : f(n) \leq c' \cdot n^2.
  \end{equation*}
  Also gilt
  \begin{equation*}
    \forall n \geq \max\{n_0, n_0'\} : c \cdot 2^n \leq f(n) \leq c' \cdot n^2.
  \end{equation*}
  Somit gilt
  \begin{equation*}
    \forall n \geq \max\{n_0, n_0'\} : 2^n \leq \frac{c'}{c} \cdot n^2.
  \end{equation*}
  Folglich gilt $[n \mapsto 2^n] \preceq [n \mapsto n^2]$, im Widerspruch zum Hinweis $[n \mapsto 2^n] \npreceq [n \mapsto n^2]$. % zu $[n \mapsto 2^n] \npreceq [n \mapsto n^2]$, was wir aus der Vorlesung wissen.
\end{loesung}

% -----------------------------------------------------------------------------

\begin{aufgabe}[4]
  Es sei $p \colon \N_0 \to \R_0^+$ eine Abbildung derart, dass eine nicht-negative ganze Zahl $k \in \N_0$ existiert und nicht-negative reelle Zahlen $a_i \in \R_0^+$, für $i \in \Z_{k + 1}$, mit $a_k \neq 0$ existieren so, dass
  \begin{equation*}
    \forall n \in \N_0 : p(n) = \sum_{i = 0}^k a_i n^i.
  \end{equation*}
  Die Abbildung $p$ ist also eine Polynomfunktion des Grades $k$ mit nicht-negativen reellwertigen Koeffizienten und Definitionsbereich $\N_0$. Beweisen Sie, dass $p \in \Theta(n \mapsto n^k)$ gilt.
\end{aufgabe}

\begin{loesung}
  Man setze $c=a_k$. Dann gilt für jedes $n \in \N_0$:
  \begin{equation*}
    p(n) =    \sum_{i = 0}^k a_i n^i
         \geq a_k n^k
         =    c n^k.
  \end{equation*}
  Somit gilt $p \in \Omega(n \mapsto n^k)$.

  Setze $c' = \sum_{i = 0}^k a_i$. Für jedes $n \in \N_0$ gilt
  \begin{equation*}
    p(n) =    \sum_{i = 0}^k a_i n^i
         \leq \sum_{i = 0}^k a_i n^k
         =    c' n^k.
  \end{equation*}
  Somit gilt $p \in O(n \mapsto n^k)$.

  Insgesamt gilt $p \in \Omega(n \mapsto n^k) \cap O(n \mapsto n^k) = \Theta(n \mapsto n^k)$.
\end{loesung}

% -----------------------------------------------------------------------------

\begin{aufgabe}[4]
  Für $n\in\N_0$ sei $G_n$ der Graph $(V_n,E_n)$ mit
  \begin{align*}
    V_n &= \{ (x,y)\in\N_0^2 \mid x+y \leq n \} \\
    E_n &= \begin{aligned}[t]
      &\mathbin{\phantom{\cup}} \{ ((x,y),(x+1,y)) \mid x+1+y\leq n \} \\
      &\cup \{ ((x,y),(x,y+1)) \mid x+y+1\leq n \}
    \end{aligned}
  \end{align*}

  \begin{enumerate}[a)]
  \item Zeichnen Sie $G_3$.
  \item Für welche $n\in \N_0$ ist $G_n$ ein Baum?
  \item Geben Sie eine Funktion $f\from \N_0\to\R_0^+$ so an, dass
    $[n\mapsto \card{V_n}] \in\Th{f}$.
  \item Geben Sie eine Funktion $g\from \N_0\to\R_0^+$ so an, dass
    $[n\mapsto \card{E_n}] \in\Th{g}$.
  \end{enumerate}
\end{aufgabe}

\begin{loesung}
  \begin{enumerate}[a)]
  \item
    \begin{tikzpicture}[every node/.style={draw,circle,inner sep=1pt}
      ,x={(20mm,0mm)},y={(0mm,20mm)},>={Latex[]},baseline=(03.center)]
      \draw 
      (0,0) node (00) {$(0,0)$}
      (0,1) node (01) {$(0,1)$}
      (0,2) node (02) {$(0,2)$}
      (0,3) node (03) {$(0,3)$}
      (1,0) node (10) {$(1,0)$}
      (1,1) node (11) {$(1,1)$}
      (1,2) node (12) {$(1,2)$}
      (2,0) node (20) {$(2,0)$}
      (2,1) node (21) {$(2,1)$}
      (3,0) node (30) {$(3,0)$}
      ;
      \path (0,-0.5) coordinate;
      \draw[->] (00) -- (01);
      \draw[->] (01) -- (02);
      \draw[->] (02) -- (03);
      \draw[->] (10) -- (11);
      \draw[->] (11) -- (12);
      \draw[->] (20) -- (21);
      ;
      \draw[->] (00) -- (10);
      \draw[->] (10) -- (20);
      \draw[->] (20) -- (30);
      \draw[->] (01) -- (11);
      \draw[->] (11) -- (21);
      \draw[->] (02) -- (12);
      ;
    \end{tikzpicture}
    \item für $n=0$ und $n=1$
    \item $f: n\mapsto n^2$  \qquad (exakt sind es $(n+1)(n+2)/2$ )
    \item $g: n\mapsto n^2$  \qquad (exakt sind es $2((n+1)(n+2)/2-(n+1)) = n(n+1)$ )
  \end{enumerate}
\end{loesung}

% -----------------------------------------------------------------------------

\begin{aufgabe}[7]
  Für jeden gerichteten Graphen $G=(V,E)$ ist der sogenannte
  \emph{Kantengraph} (engl.~\emph{line graph}) $L(G)=(V',E')$ wie
  folgt definiert: Wenn $E$ nicht leer ist, dann ist
  \begin{align*}
    V' &= E, \\
    E' &= \{ ( (x,y), (y,z) ) \mid (x,y), (y,z)\in V'  \};
  \end{align*}
  %
  wenn $E$ leer ist, dann ist $V'=\{0\}$ und $E'=\{\}$.

  Für $n\in\N_0$ sei der $n$-te iterierte Kantengraph $L^n(G)$ so definiert:
  %
  \begin{align*}
    L^0(G) &= G, \\
    \forall n\in\N_0:L^{n+1}(G) &= L(L^n(G)) \;.
  \end{align*}
  Es bezeichne im folgenden $L^n_V(G)$ die Knotenmenge von $L^n(G)$ und
  $L^n_E(G)$ die Kantenmenge von $L^n(G)$.

  \emph{Hinweis:} $|M|$ bezeichnet im folgenden stets die
  Kardinalität, also die Anzahl der Elemente, einer endlichen Menge
  $M$.

  \begin{enumerate}[a)]
  \item Zeichnen Sie zu dem Graphen $H_1$

  \qquad  \begin{tikzpicture}[x={(20mm,0mm)},y={(0mm,-20mm)},every node/.style={draw,circle,minimum size=9mm,anchor=mid}]
      \node (A) at (0,0) {$A$} ;
      \node (B) at (1,0) {$B$} ;
      \node (C) at (0,1) {$C$} ;
      \node (D) at (1,1) {$D$} ;
      \draw[-{Latex[]}] (B) edge (A)
                            edge (C)
                        (C) edge (D)
                        (D) edge (B);
    \end{tikzpicture}

    den Kantengraphen $L(H_1)$ und benennen sie dessen Knoten sinnvoll.
  \item Geben Sie eine Funktion $f\from \N_0\to\R_0^+$ so an, dass
    $[n\mapsto \card{L^n_V(H_1)}] \in\Th{f}$.
  \item Geben Sie einen Graphen $H_2$ mit $5$ Knoten und
    $5$ Kanten so an, dass für dessen iterierte Kantengraphen
    $L^n(H_2)$ gilt:
    \[
      [n\mapsto \card{L^n_E(H_2)}]\in \Th{n\mapsto 0} .
    \]
    %
    %(Dabei stehe die $0$ in $\Th{0}$ für die konstante Nullfunktion
    %$[n\mapsto 0]$.)
  \item Für $n\in\N_+$ sei $B_n$ der de Bruijn-Graph mit Knotenmenge
    $V_n=\{\#0,\#1\}^n$ und Kantenmenge
    $E_n=\{ (xw,wy) \mid x,y\in\{\#0,\#1\} \land w\in\{\#0,\#1\}^{n-1}\}$
    (siehe Kapitel 15).
    \begin{itemize}
    \item Geben Sie für jedes $n\in\N_+$ eine Bijektion
      $\phi_n\from E_n \to \{\#0,\#1\}^{n+1}$ an.
    \item Zeigen Sie, dass für jedes $n\in\N_+$ der Kantengraph
      $L(B_n)$ isomorph zu $B_{n+1}$ ist.
    \item Für welche $k\in\Z_4$ gilt
      $[n\mapsto \card{L_V^n(B_2)}] \in \Oh{n\mapsto k^n}$ ?
    \end{itemize}
  \end{enumerate}
\end{aufgabe}

\begin{loesung}
  \begin{korrektur}
    Punkteaufteilung: $1+1+2+3$
  \end{korrektur}

  \begin{enumerate}[a)]
  \item \begin{tikzpicture}[x={(20mm,0mm)},y={(0mm,-20mm)},
      every node/.style={draw,circle,minimum size=9mm,anchor=mid},baseline=(current bounding box.north)]
      \node (A) at (0,0) {$BA$} ;
      \node (B) at (1,0) {$BC$} ;
      \node (C) at (0,1) {$CD$} ;
      \node (D) at (1,1) {$DB$} ;
      \draw[-{Latex[]}] (B) edge (C)
                        (C) edge (D)
                        (D) edge (A)
                            edge (B);
    \end{tikzpicture}
    
    Dieser Graph ist isomorph zu $H_1$
  \item $n\mapsto 4$
  \item \begin{tikzpicture}[x={(20mm,0mm)},y={(0mm,-20mm)},
      every node/.style={draw,circle,minimum size=9mm,anchor=mid},baseline=(current bounding box.north)]
      \node (A) at (0,0) {$A$} ;
      \node (B) at (1,0) {$B$} ;
      \node (C) at (2,0) {$C$} ;
      \node (D) at (3,0) {$D$} ;
      \node (E) at (4,0) {$E$} ;
      \draw[-{Latex[]}] (A) edge[bend left=30] (C)
                        (B) edge (A)
                            edge (C)
                        (D) edge (C)
                            edge (E);
    \end{tikzpicture}

    Anmerkung: In $L(H_2)$ existiert nur noch eine einzige Kante,
    nämlich von Knoten $(B,A)$ zu Knoten $(A,C)$; die einzige Kante
    ist also keine Schlinge. Folglich existieren in $L^2(H_2)$ keine
    Kanten mehr und damit auch nicht in allen weiteren iterierten
    Kantengraphen.
  \item 
    \begin{korrektur}
      $0.5 + 2 + 0.5$ Punkte
    \end{korrektur}
    \mbox{ }
    \begin{itemize}
    \item $\phi_n\from E_n \to \{\#0,\#1\}^{n+1} \from (xw,wy) \mapsto xwy$
    \item Das eben definierte $\phi_n$ leistet schon das von einem
      Isomorphismus Gewünschte.

      Es ist noch zu zeigen
      \begin{itemize}
      \item Wenn eine Kante in $L(B_n)$ von $s$ nach $t$ führt, dann
        auch eine Kante in $B_{n+1}$ von $\phi_n(s)$ nach $\phi_n(t)$:

        Im folgenden seien $x,x',y,y',z\in\{0,1\}$ und
        $w,w',v\in\{0,1\}^*$.
        
        Es sei $s=(xw,wy)$ und $t=(x'w',w'y')$. Wenn eine Kante von
        $s$ zu $t$ führt, dann ist $wy=x'w'$.

        \begin{itemize}
        \item Falls $n=1$ und daher $w=w'=\eps$ ist, ist auch $y=x'$
          und alles vereinfacht sich zu $s=(x,y)$ und $t=(y,y')$. Dann
          ist $\phi_n(s)=xy$ nach $\phi_n(t)=yy'$ und es gibt nach
          Definition von $E_2$ eine Kante von $\phi_1(s)$ nach
          $\phi_1(t)$.
        \item Falls $n\geq 2$ ist, gibt es ein $v$ mit
          $wy = x'vy = x'w'$ und es ist $s=(xx'v,x'vy)$ und
          $t=(x'vy,vyy')$. Man erhält $\phi_n(s)=xx'vy$ nach
          $\phi_n(t)=x'vyy'$ und nach Definition von $E_{n+1}$
          existiert eine Kante von $\phi_1(s)$ nach $\phi_1(t)$.
        \end{itemize}
      \item Wenn eine Kante in $B_{n+1}$ von $\phi_n(s)$ nach
        $\phi_n(t)$ führt, dann auch eine Kante in $L(B_n)$ von $s$
        nach $t$:

        \begin{itemize}
        \item Falls $n=1$, ist $\phi_n(s)=xy$ nach $\phi_n(t)=yy'$.
          Also ist der Endpunkt der Kante $(x,y)$ der Anfangspunkt der
          Kante $(y,y')$ und folglich existiert in $L(B_1)$ eine Kante
          von $s=(x,y)$ zu $t=(y,y')$.
        \item Falls $n\geq 2$, ist $\phi_n(s)=xwy$ und
          $\phi_n(t)=x'w'y'$. Wenn es eine Kante von $\phi_n(s)$ zu
          $\phi_n(t)$ gibt, dann ist $wy=x'w'$ und es muss ein $v$
          existieren mit $wy=x'vy=x'w'$. Folglich existiert in
          $L(B_n)$ eine Kante von $(xw,wy)$ zu $(x'w',w'y')$.
        \end{itemize}
      \end{itemize}
    \item für $k\in\{2,3\}$
    \end{itemize}
  \end{enumerate}
\end{loesung}

% -----------------------------------------------------------------------------

\end{document}
%%%
%%% Local Variables:
%%% fill-column: 70
%%% mode: latex
%%% TeX-command-default: "XPDFLaTeX"
%%% TeX-master: "korrektur.tex"
%%% End:
