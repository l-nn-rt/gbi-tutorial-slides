\documentclass[12pt]{article}

\input{preamble-aufgaben}

\usetikzlibrary{positioning}
\usetikzlibrary{arrows,shapes}

% =======================================================
%\newcounter{blattnr}
\setcounter{blattnr}{10}
\newcommand{\ausgabetermin}{13.~Januar 2015}
\newcommand{\abgabetermin}{22.~Januar 2015}
%\newcommand{\punkteblatt}{13} % Blatt 1
%\newcommand{\punkteblattphysik}{13} % Blatt 1
%\newcommand{\punkteblatt}{17} % Blatt 2
%\newcommand{\punkteblattphysik}{14} % Blatt 2
%\newcommand{\punkteblatt}{18} % Blatt 3
%\newcommand{\punkteblattphysik}{18} % Blatt 3
%\newcommand{\punkteblatt}{18} % Blatt 4
%\newcommand{\punkteblattphysik}{18} % Blatt 4
%\newcommand{\punkteblatt}{18} % Blatt 5
%\newcommand{\punkteblattphysik}{18} % Blatt 5
%\newcommand{\punkteblatt}{20} % Blatt 6
%\newcommand{\punkteblattphysik}{20} % Blatt 6
%\newcommand{\punkteblatt}{20} % Blatt 7
%\newcommand{\punkteblattphysik}{0} % Blatt 7
%\newcommand{\punkteblatt}{18} % Blatt 8
%\newcommand{\punkteblattphysik}{18} % Blatt 8
%\newcommand{\punkteblatt}{17} % Blatt 9
%\newcommand{\punkteblattphysik}{17} % Blatt 9
\newcommand{\punkteblatt}{16} % Blatt 9
\newcommand{\punktetotal}{175} %
\newcommand{\punkteblattphysik}{16} % Blatt 9
\newcommand{\punktetotalphysik}{152} %
% =======================================================

\begin{document}

%\noindent
%Mit \textbf{[nicht Physik]} gekennzeichnete Aufgaben müssen von
%Studenten der Physik nicht bearbeitet werden.\\

% -----------------------------------------------------------------------------

\begin{aufgabe}[4]
  Es sei $T = (V, E)$ ein gerichteter Baum und es sei $r$ die Wurzel von $T$. Beweisen Sie, dass es keine Zyklen in $T$ gibt.
\end{aufgabe}

\begin{loesung}
  Angenommen es gibt einen Zyklus $z$ in $T$. Bezeichne den Start- und Zielknoten von $z$ mit $v$. Da $T$ ein Baum ist, gibt es einen Pfad $p$ von $r$ nach $v$. Der Pfad $p'$, der zunächst entlang $p$ von $r$ nach $v$ läuft und dann entlang $z$ von $v$ nach $v$, ist ein Pfad von $r$ nach $v$. Da Zyklen mindestens die Länge $1$ haben, ist $p'$ länger als $p$, also insbesondere verschieden von $p$. Somit gibt es zwei Pfade von $r$ nach $v$, im Widerspruch zur Definition von Bäumen.
\end{loesung}

% -----------------------------------------------------------------------------

\begin{aufgabe}[4]
  Es sei $G = (V, E)$ ein gerichteter Graph und es sei $G' = (V, E^*)$, wobei $E^*$ die reflexiv-transitive Hülle von $E$ ist. Beweisen Sie durch vollständige Induktion, dass für jedes $n \in \N_0$, jeden Knoten $v \in V$ und jeden Knoten $w \in V$ genau dann $(v, w) \in E^n$ gilt, wenn es in $G$ einen Pfad der Länge $n$ von $v$ nach $w$ gibt. % Beweisen Sie, dass für jeden Knoten $v \in V$ und jeden Knoten $w \in V$ genau dann $(v, w) \in E^*$ gilt, wenn es in $G$ einen Pfad von $v$ nach $w$ gibt, indem Sie durch vollständige Induktion beweisen, dass für jedes $n \in \N_0$, jeden Knoten $v \in V$ und jeden Knoten $w \in V$ genau dann $(v, w) \in E^n$ gilt, wenn es in $G$ einen Pfad der Länge $n$ von $v$ nach $w$ gibt.
%  \begin{enumerate} % Bei den auskommentierten Aussagen und deren Beweisen muss man noch Wiederholungsfreiheit oder ähnliches verwenden und sicherstellen, damit die Argumentation korrekt ist.
%    \item
%    \item Beweisen Sie, dass es für jeden Knoten $v \in V$ und jeden Knoten $w \in V$ genau dann einen Pfad von $v$ nach $w$ in $G'$ gibt, wenn es einen mindestens so langen Pfad von $v$ nach $w$ in $G$ gibt.
%    \item Beweisen Sie: Wenn es keine Zyklen in $G$ gibt, dann gibt es keine Zyklen der Mindestlänge $2$ in $G'$ gibt.
%  \end{enumerate}
\end{aufgabe}

\begin{loesung}
  \begin{description}
    \item[Induktionsanfang.] Es seien $v$, $w \in V$. Es gilt $(v, w) \in E^0$ genau dann, wenn $v = w$ gilt. Und es gilt $v = w$ genau dann, wenn es in $G$ einen Pfad der Länge $0$ von $v$ nach $w$ gibt.
    \item[Induktionsschritt.] Es sei $n \in \N_0$ derart, dass für jedes $v \in V$ und jedes $w' \in V$ genau dann $(v, w') \in E^n$ gilt, wenn es in $G$ einen Pfad der Länge $n$ von $v$ nach $w'$ gibt (Induktionsvoraussetzung). 

Weiter seien $v$, $w \in V$. Es gilt $(v, w) \in E^{n + 1} = E^n \circ E$ genau dann, wenn es ein $w' \in V$ gibt so, dass $(v, w') \in E^n$ und $(w', w) \in E$. 

Die rechte Seite gilt nach Induktionsvoraussetzung genau dann, wenn es ein $w' \in V$ gibt so, dass es in $G$ einen Pfad der Länge $n$ von $v$ nach $w'$ gibt und $(w', w) \in E$ gilt. Die rechte Seite gilt genau dann, wenn es in $G$ einen Pfad der Länge $n + 1$ von $v$ nach $w$ gibt.
  \end{description}

  \begin{korrektur}
    Ind.anfang und Ind.vorauss.~jeweils 1 Punkt, restliche Argumentation 2 Punkte.

    Man beachte, dass man "`genau dann, wenn"' beweisen muss! Wer nur
    eine Richtung macht, bekommt jeweils nur die halbe Punktzahl.
  \end{korrektur}
\end{loesung}

% -----------------------------------------------------------------------------

% taken from http://www.texample.net/tikz/examples/prims-algorithm/
\pgfdeclarelayer{background}
\pgfsetlayers{background,main}

\tikzstyle{vertex}=[circle, draw=black, minimum size=20pt,inner sep=0pt]
\tikzstyle{selected vertex} = [vertex, fill=white]
\tikzstyle{edge} = [draw,thick,-]
\tikzstyle{weight} = [font=\small]
\tikzstyle{selected edge} = [draw,line width=5pt,-,red!50]
\tikzstyle{ignored edge} = [draw]

\begin{aufgabe}[4]
  Es sei $G = (V, E)$ ein kantenmarkierter, zusammenhängender und ungerichteter Graph mit Markierungsabbildung $m_E \colon E \to \N_0$. Ein Teilgraph $G' = (V', E')$ von $G$ heißt genau dann \emph{Spannbaum von $G$}, wenn $G'$ ein ungerichteter Baum ist und $V' = V$ gilt. Ein Spannbaum $G' = (V', E')$ von $G$ heißt genau dann \emph{minimal}, wenn für jeden anderen Spannbaum $G'' = (V'', E'')$ von $G$ gilt:
  \begin{equation*}
    \sum_{e'' \in E''} m_E(e'') \geq \sum_{e' \in E'} m_E(e').
  \end{equation*}
%  Die Knoten $v_i$, Mengen $V_i$ und Mengen $E_i$, für $i \in \Z_{|V|}$, und Kanten $e_i$, für $i \in \Z_{|V|} \smallsetminus \{0\}$, seien induktiv gewählt durch
%  \begin{align*}
%    K_0 &= V,\\
%    v_0 &\in K_0,\\
%    V_0 &= \{v_0\},\\
%    E_0 &= \{\},\\
%    \forall i \in \Z_{|V| - 1} \colon
%    K_{i + 1} &= \{v \in V \smallsetminus V_i \mid \exists w \in V_i \colon \{v, w\} \in E\},
%    v_{i + 1} &\in \{v \in K_{i + 1} \mid \forall v' \in K_{i + 1}\; \forall y' \in V_i \text{ mit } \{v', y'\} \in E \colon m_E(\{v', y'\}) \geq m_E(\{v, y\})\},\\
%    V_{i + 1} &= V_i \cup \{v_{i + 1}\},\\
%    e_{i + 1} &\in \{\{v_{i + 1}, y\} \in E \mid y \in V_i \land \forall y' \in V_i \text{ mit } \{x', y'\} \in E \colon m_E(\{v_{i + 1}, y'\}) \geq m_E(\{v_{i + 1}, y\})\},\\
%    E_{i + 1} &= E_i \cup \{e_{i + 1}\}
%  \end{align*}
  Der folgende "`Algorithmus"' berechnet einen minimalen Spannbaum $G' = (V', E')$ von $G$:
  \begin{align*}
    & \text{wähle } x_0 \in V\\
    & V' \gets \{x_0\}\\
    & E' \gets \{\}\\
    & i \gets 1\\
    & \kw{while } V \smallsetminus V' \neq \{\} \kw{ do}\\
    & \quad \text{wähle } x_i \in V' \text{ und } y_i \in V \smallsetminus V' \text{ derart, dass } \{x_i, y_i\} \in E \text{ und}\\
    & \quad \quad \quad \quad \forall x \in V'\; \forall y \in V \smallsetminus V' \colon \bigl(\{x, y\} \in E \rightarrow m_E(\{x, y\}) \geq m_E(\{x_i, y_i\})\bigr)\\
    & \quad V' \gets V' \cup \{y_i\}\\
    & \quad E' \gets E' \cup \{\{x_i, y_i\}\}\\
    & \quad i \gets i + 1\\
    & \kw{od}
  \end{align*}
  \emph{Hinweis:} An den beiden Stellen, an denen \emph{"`wähle"'} steht, ist es für
  die Korrektheit des Verfahrens nicht wichtig, welche Wahl man
  trifft, sofern sie den angegebenen Bedingungen entspricht.

  Geben Sie die Werte der Variablen $x_k$ und $y_k$, für $k \in \Z_n \smallsetminus \{0\}$, nach der Berechnung des minimalen Spannbaums des Graphen

  \begin{tikzpicture}[x={(16mm,0mm)},y={(0,12mm)}, auto,swap]
    \foreach \pos/\name in {{(0,2)/a}, {(2,2)/b}, {(4,2)/c}, {(0,0)/d}, {(2,0)/e}, {(2,-2)/f}, {(4,-2)/g}}
        \node[vertex] (\name) at \pos {$\#{\name}$};
    \foreach \source/ \dest /\weight in {b/a/7, c/b/8,d/a/5,d/b/9,
                                         e/b/7, e/c/5,e/d/15,
                                         f/d/6,f/e/8,
                                         g/e/9,g/f/11}
        \path[edge] (\source) -- node[weight] {$\weight$} (\dest);
  \end{tikzpicture}\\
  mit dem obigen Algorithmus an, wobei im allerersten Schritt als $x_0$ der Knoten $\#a$ gewählt werde.
\end{aufgabe}

\begin{loesung}
  $x_1 = \#a$, $y_1 = \#d$, \\
  $x_2 = \#d$, $y_2 = \#f$, \\
  $x_3 = \#a$, $y_3 = \#b$, \\
  $x_4 = \#b$, $y_4 = \#e$, \\
  $x_5 = \#e$, $y_5 = \#c$, \\
  $x_6 = \#e$, $y_6 = \#g$

  In der folgenden Darstellung bilden alle Knoten zusammen mit den hellgrau hinterlegten Kanten den minimalen Spannbaum, den der Algorithmus mit der Wahl $x_0 = \#a$ berechnet:

  \tikzstyle{selected edge} = [draw,line width=5pt,-,black!20]

  \begin{tikzpicture}[x={(16mm,0mm)},y={(0,12mm)}, auto,swap]
    \foreach \pos/\name in {{(0,2)/a}, {(2,2)/b}, {(4,2)/c}, {(0,0)/d}, {(2,0)/e}, {(2,-2)/f}, {(4,-2)/g}}
        \node[vertex] (\name) at \pos {$\#{\name}$};
    \foreach \source/ \dest /\weight in {b/a/7, c/b/8,d/a/5,d/b/9,
                                         e/b/7, e/c/5,e/d/15,
                                         f/d/6,f/e/8,
                                         g/e/9,g/f/11}
        \path[edge] (\source) -- node[weight] {$\weight$} (\dest);
    \foreach \vertex / \fr in {d/1,a/2,f/3,b/4,e/5,c/6,g/7}
            \path node[selected vertex] at (\vertex) {$\vertex$};
    \begin{pgfonlayer}{background}
        \foreach \source / \dest in {d/a,d/f,a/b,b/e,e/c,e/g}
            \path[selected edge] (\source.center) -- (\dest.center);
        \foreach \source / \dest / \fr in {d/b/4,d/e/5,e/f/5,b/c/6,f/g/7}
            \path[ignored edge] (\source.center) -- (\dest.center);
    \end{pgfonlayer}
  \end{tikzpicture}

  \begin{korrektur}
    Die Zahlen wurden so gewählt, dass der minimale Spannbaum
    eindeutig ist. Da nur die Angabe der $(x_k,y_k)$ verlangt war (und
    kein Bild) ist die Korrektur hoffentlich schnell möglich.

    Problematisch wird es, wenn jemand schon bei der ersten oder
    zweiten Kante einen Fehler gemacht hat. Dann sollten Sie bitte
    gucken, ob der Rest "`nur"' Folgefehler sind. Für letztere nichts
    abziehen (sofern die Aufgabe nicht signifikant leichter wurde).
  \end{korrektur}
\end{loesung}

% -----------------------------------------------------------------------------

\begin{aufgabe}[4]
  Es sei $G = (\Z_n, E)$ ein gerichteter Graph und es sei $A$ die Adjazenzmatrix von $G$. Wir nummerieren die Zeilen und Spalten von $A$ mit $0$ beginnend durch und bezeichnen für jedes $i \in \Z_n$ und jedes $j \in \Z_n$ mit $A_{i,j}$ den Eintrag von $A$ in der $i$-ten Zeile und $j$-ten Spalte. Die Matrix $A$ heißt genau dann
%  Es sei $n$ eine positive ganze Zahl und es sei $A$ eine Matrix mit $n$ Zeilen, $n$ Spalten und ganzzahligen Einträgen. Wir nummerieren die Zeilen und Spalten von $A$ mit $0$ beginnend durch und bezeichnen für jedes $i \in \Z_n$ und jedes $j \in \Z_n$ mit $A_{i,j}$ den Eintrag von $A$ in der $i$-ten Zeile und $j$-ten Spalte. Die Matrix $A$ heißt genau dann
  \begin{itemize}
    \item \emph{hohl}, wenn % siehe https://en.wikipedia.org/wiki/Hollow_matrix
          \begin{equation*}
            \forall i \in \Z_n \colon A_{i,i} = 0;
          \end{equation*}
    \item \emph{symmetrisch}, wenn
          \begin{equation*}
            \forall i \in \Z_n\; \forall j \in \Z_n \colon A_{i,j} = A_{j,i};
          \end{equation*}
    \item \emph{strikte obere Dreiecksmatrix}, wenn
          \begin{equation*}
            \forall i \in \Z_n\; \forall j \in \Z_{i + 1} \colon A_{i,j} = 0;
          \end{equation*}
    \item \emph{hohle $(2 \times 2)$-Blockmatrix}, wenn
          \begin{equation*}
            \exists k \in \Z_n \colon
              (\forall (i,j) \in \Z_k^2 \colon A_{i,j} = 0) \land
              (\forall (i,j) \in (\Z_n \smallsetminus \Z_k)^2 \colon A_{i,j} = 0).
          \end{equation*}
  \end{itemize}
%  Es sei $G = (V, E)$ ein gerichteter Graph. Der Graph $G$ heißt genau dann
  Der Graph $G$ heißt genau dann
  \begin{itemize}
    \item \emph{bipartit}, wenn es zwei Teilmengen $A$ und $B$ von $V$ gibt so, dass $A \cap B = \{\}$ und $E \subseteq (A \times B) \cup (B \times A)$;
    \item \emph{zyklenfrei} oder \emph{DAG}, wenn es keine Zyklen in $G$ gibt;
    \item \emph{schlingenfrei}, wenn es keine Schlingen in $G$ gibt;
    \item \emph{richtungslos}, wenn für jedes $(v, w) \in E$ gilt $(w, v) \in E$. % eigentlich 'symmetrisch' aber ...
  \end{itemize}
%  Fortan sei $G = (\Z_n, E)$ ein gerichteter Graph und es sei $A$ die Adjazenzmatrix von $G$.
  Beantworten Sie mit jeweils einem Wort die folgenden Fragen:
  \begin{enumerate}
    \item Falls $A$ eine strikte obere Dreiecksmatrix ist, welche der oben genannten Eigenschaften hat $G$?
    \item Falls $A$ eine hohle $(2 \times 2)$-Blockmatrix ist, welche der oben genannten Eigenschaften hat $G$?
    \item Falls $G$ schlingenfrei ist, welche der oben genannten Eigenschaften hat $A$?
    \item Falls $G$ richtungslos ist, welche der oben genannten Eigenschaften hat $A$?
  \end{enumerate}
\end{aufgabe}

\begin{loesung}
  \begin{enumerate*}[itemjoin={{\qquad}}]
    \item zyklenfrei
    \item bipartit
    \item hohl
    \item symmetrisch
  \end{enumerate*}
\end{loesung}

% -----------------------------------------------------------------------------

\end{document}
%%%
%%% Local Variables:
%%% fill-column: 70
%%% mode: latex
%%% TeX-command-default: "XPDFLaTeX"
%%% TeX-master: "korrektur.tex"
%%% End:
