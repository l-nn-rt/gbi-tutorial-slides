\documentclass[12pt]{article}

\input{preamble-aufgaben}
% =======================================================

% =======================================================
%\newcounter{blattnr}
\setcounter{blattnr}{6}
\newcommand{\ausgabetermin}{2.~Dezember 2015}
\newcommand{\abgabetermin}{11.~Dezember 2015}
%\newcommand{\punkteblatt}{13} % Blatt 1
%\newcommand{\punkteblattphysik}{13} % Blatt 1
%\newcommand{\punkteblatt}{17} % Blatt 2
%\newcommand{\punkteblattphysik}{14} % Blatt 2
%\newcommand{\punkteblatt}{18} % Blatt 3
%\newcommand{\punkteblattphysik}{18} % Blatt 3
%\newcommand{\punkteblatt}{18} % Blatt 4
%\newcommand{\punkteblattphysik}{18} % Blatt 4
%\newcommand{\punkteblatt}{18} % Blatt 5
%\newcommand{\punkteblattphysik}{18} % Blatt 5
\newcommand{\punkteblatt}{20} % Blatt 6
\newcommand{\punktetotal}{104} %
\newcommand{\punkteblattphysik}{20} % Blatt 6
\newcommand{\punktetotalphysik}{101} %
% =======================================================

\begin{document}

%\noindent
%Mit \textbf{[nicht Physik]} gekennzeichnete Aufgaben werden von
%Studenten der Physik bitte nicht bearbeitet.\\

% -----------------------------------------------------------------------------

\begin{aufgabe}[2 + 2 + (2 + 1 + 1 + 2) + 2 + 2 = 14]
  Es sei $A$ ein Alphabet; es sei $\mathcal{L}$ die Menge aller formalen Sprachen über $A$, das heißt, $\mathcal{L} = \{L \mid L \subseteq A^*\}$; es sei $f \colon \mathcal{L} \to \mathcal{L}$ eine Abbildung derart, dass für jede formale Sprache $S \in \mathcal{L}$ und jede formale Sprache $T \in \mathcal{L}$ mit $S \subseteq T$ gilt: $f(S) \subseteq f(T)$; es seien die formalen Sprachen $L_n$, $n \in \N_0$, induktiv definiert durch
  \begin{align*}
    L_0 &= \{\},\\
    \text{für jedes } n \in \N_0 \colon L_{n + 1} &= f(L_n);
  \end{align*}
  und es sei $L_\infty$ die formale Sprache $\bigcup_{n \in \N_0} L_n$.
  \begin{enumerate}
    \item Beweisen Sie durch vollständige Induktion, dass für jedes $n \in \N_0$ gilt: $L_n \subseteq L_{n + 1}$.
    \item Beweisen Sie, dass $f(L_\infty) = L_\infty$ gilt. Eine formale Sprache mit dieser Eigenschaft nennt man \emph{Fixpunkt von $f$}.

          \emph{Hinweis:} Für jede Menge $I$ und alle formalen Sprachen $S_i \subseteq A^*$, $i \in I$, gilt $f(\bigcup_{i \in I} S_i) = \bigcup_{i \in I} f(S_i)$.
    \item In dieser Teilaufgabe sei $A = \{\#0,\#1\}$ und es sei
          \begin{align*}
            f \colon \mathcal{L} &\to     \mathcal{L},\\
                               L &\mapsto \{\#0,\#1\} \cup (\{\#0,\#1\} \cdot L).
          \end{align*}
          \begin{enumerate}[(i)]
            \item Geben Sie $L_1$, $L_2 \setminus L_1$ und $L_3 \setminus L_2$ so explizit wie möglich in der Form $\{ \dotsc \}$ an.
            \item Geben Sie einen arithmetischen Ausdruck $E$, in dem das Symbol $n$ vorkommt und die Sprachen $L_n$, $n \in \N_0$, nicht vorkommen, so an, dass für jedes $n \in \N_0$ gilt: $|L_{n + 1} \setminus L_n| = E$.
            \item Geben Sie $L_\infty$ ohne Bezug auf die formalen Sprachen $L_n$, $n \in \N_0$, an.
            \item Geben Sie eine kontextfreie Grammatik $G$ so an, dass die von ihr erzeugte formale Sprache $L(G)$ gleich $L_\infty$ ist.
          \end{enumerate}
    \item In dieser Teilaufgabe sei $A = \{\#0,\#1,\#;\}$ und es sei
          \begin{align*}
            f \colon \mathcal{L} &\to     \mathcal{L},\\
                               L &\mapsto \{\#0,\#1\}^+ \cup (\{\#0,\#1\}^+ \cdot \{\#;\} \cdot L).
          \end{align*}
          Geben Sie eine kontextfreie Grammatik $G$ so an, dass $L(G) = L_\infty$ gilt.
    \item In dieser Teilaufgabe sei $A = \{ \#(, \#) \}$ und es sei $G = (N, T, \#S, P)$ die Grammatik mit den Nichtterminalsymbolen $N = \{ \#S \}$, den Terminalsymbolen $T = \{ \#(, \#) \}$ und den Produktionen
          \begin{equation*}
            P = \{ \#S \rightarrow \varepsilon \mid \#{S(S)} \}.
          \end{equation*}
          Geben Sie eine Abbildung $f \colon \mathcal{L} \to \mathcal{L}$ so an, dass $L_\infty = L(G)$ gilt.
  \end{enumerate}
\end{aufgabe}

\begin{loesung}
  \emph{Nebenbei:} Jede Abbildung $g \colon A^* \to A^*$ induziert eine Abbildung $f \colon \mathcal{L} \to \mathcal{L}$ vermöge $L \mapsto g(L)$ mit der gewünschten Eigenschaft. Und für jedes $n \in \N_0$ gilt $L_n = f^n(\{\})$. Und $L_\infty = \bigcup_{n \in \N_0} f^n(\{\})$. Und $f(L_\infty) = \bigcup_{n \in \N_0} f^{n+1}(\{\})$.
  \begin{enumerate}
    \item \emph{Induktionsanfang:} Die leere Menge ist Teilmenge jeder Menge. Insbesondere gilt $L_0 = \{\} \subseteq f(\{\}) = L_1$.

          \emph{Induktionsschritt:} Es sei $n \in \N_0$ derart, dass $L_n \subseteq L_{n + 1}$. Dann gilt $L_{n + 1} = f(L_n) \subseteq f(L_{n + 1}) = L_{(n + 1) + 1}$.

          \emph{Schlussworte:} Nach dem Prinzip der vollständigen Induktion gilt die Behauptung.
          \begin{korrektur}
            0,5 Punkte für den Induktionsanfang; 1,5 Punkte für den Induktionsschritt und davon 0,5 Punkte für die Induktionsvoraussetzung.
          \end{korrektur}
    \item Es gilt
          \begin{multline*}
            f(L_\infty) = f(\bigcup_{n \in \N_0} L_n)
                        = \bigcup_{n \in \N_0} f(L_n)
                        = \bigcup_{n \in \N_0} L_{n + 1}
                        = \bigcup_{k \in \N_+} L_k\\
                        = \{\} \cup \bigcup_{k \in \N_+} L_k
                        = L_0 \cup \bigcup_{k \in \N_+} L_k
                        = \bigcup_{k \in \N_0} L_k
                        = L_\infty.
          \end{multline*}
    \item \begin{enumerate}
            \item $L_1 \setminus L_0 = \{\#0,\#1\}$, $L_2 \setminus L_1 = \{\#{00},\#{01},\#{10},\#{11}\}$, $L_3 \setminus L_2 = \{\#{000},\#{001},\#{010},\#{011},\#{100},\#{101},\#{110},\#{111}\}$.
                  \begin{korrektur}
                    Jeweils 0,5 Punkte für $L_1$ und $L_2$; 1 Punkt für $L_3$.
                  \end{korrektur}
            \item $E = 2^{n + 1}$.
            \item $L_\infty = \{\#0,\#1\}^+$ oder $L_\infty = \{\#0,\#1\} \cdot \{\#0,\#1\}^*$ oder $L_\infty = \{w \in \{\#0,\#1\}^* \mid |w| \geq 1\}$ oder ...
            \item Die Grammatik $G = (N, T, \#B, P)$ mit den Nichtterminalsymbolen $\{ \#B \}$, den Terminalsymbolen $\{ \#0, \#1 \}$ und den Produktionen $\{ \#B \rightarrow \#0 \mid \#1 \mid \#{0B} \mid \#{1B} \}$ leistet das Gewünschte.

                  \emph{Nebenbei:} Für jedes $L \in \mathcal{L}$ gilt $f(L) = \{\#0\} \cup \{\#1\} \cup \{\#0\} \cdot L \cup \{\#1\} \cdot L$. Entfernt man aus diesem Ausdruck die Symbole $\{$, $\}$ und $\cdot$, ersetzt $\cup$ durch $\mid$, $L$ durch $\#B$, ${=}$ durch $\rightarrow$ und $f(L)$ durch $\#B$, so erhält man die Produktionen in $P$.
          \end{enumerate}
    \item Die Grammatik $G = (N, T, \#S, P)$ mit den Nichtterminalsymbolen $\{ \#S, \#B \}$, den Terminalsymbolen $\{ \#0, \#1, \#; \}$ und den Produktionen
          \begin{align*}
            \{
              \#S &\rightarrow \#B \mid \#{B;S},\\
              \#B &\rightarrow \#0 \mid \#1 \mid \#{0B} \mid \#{1B}
            \}
          \end{align*}
          leistet das Gewünschte.
    \item Die Abbildung
          \begin{align*}
            f \colon \mathcal{L} &\to     \mathcal{L},\\
                               L &\mapsto \{\varepsilon\} \cup L \cdot \{\#(\} \cdot L \cdot \{\#)\},
          \end{align*}
          leistet das Gewünschte.
  \end{enumerate}
\end{loesung}

% -----------------------------------------------------------------------------

\newpage
\begin{aufgabe}[2 + 4 = 6]
  Es sei $G = (N, T, \#S, P)$ die Grammatik mit den Nichtterminalsymbolen $N = \{ \#S, \#U, \#X, \#Q \}$, den Terminalsymbolen $T = \{ \#a \}$ und den Produktionen
  \begin{align*}
    P = \{
          \#S &\rightarrow \#{aU} \mid \#{aXa} \mid \#{Qaa},\\
          \#U &\rightarrow \#{aaU} \mid \varepsilon,\\
          \#X &\rightarrow \#{Qaaa} \mid \#a,\\
          \#Q &\rightarrow \#{aXa} \mid \#a
        \}
  \end{align*}
  \begin{enumerate}
    \item Leiten Sie aus dem Startsymbol das Wort $\#{a}^7$ ab. Geben Sie dabei jeden Ableitungsschritt an.
    \item Zeichnen Sie den Ableitungsbaum für das Wort $\#{a}^{16}$. %$\#{aa}^6 \cdot \#a$.
  \end{enumerate}
\end{aufgabe}

\begin{loesung}
  Aus dem Nichtterminalsymbol $\#U$ sind alle Wörter über $T$ gerader Länge ableitbar. Somit sind aus $\#{aU}$ alle Wörter über $T$ ungerader Länge ableitbar. Aus den Nichtterminalsymbolen $\#X$ und $\#Q$ sind alle Wörter über $T$ der Längen $x_n$ beziehungsweise $q_n$, $n \in \N_0$, ableitbar, wobei die nicht-negativen ganzen Zahlen $x_n$ und $q_n$, $n \in \N_0$, wechselseitig induktiv definiert sind durch
  \begin{align*}
    &\qquad x_0 = 1,\\
    &\qquad q_0 = 1,\\
    \text{für jedes } n \in \N_0 \colon &\begin{dcases*}
                                          x_{n + 1} = q_n + 3,\\
                                          q_{n + 1} = 1 + x_n + 1.
                                        \end{dcases*}
  \end{align*}
  Somit sind aus $\#{aXa}$ und $\#{Qaa}$ alle Wörter über $T$ der Längen $1 + x_n + 1$ beziehungsweise $q_n + 2$, $n \in \N_0$, ableitbar. Es gelten $x_0 = 1$, $q_0 = 1$, $x_1 = q_0 + 3 = 4$, $q_1 = 1 + x_0 + 1 = 3$, $x_2 = 6$, $q_2 = 6$, $x_3 = 9$, $q_3 = 8$, $x_4 = 11$, $q_4 = 11$, $x_5 = 14$, $q_5 = 13$.
  \begin{enumerate}
    \item Da das Wort ungerade Länge hat kann es über $\#{aU}$ aus $\#S$ wie folgt abgeleitet werden:
          \begin{equation*}
            \#S \Rightarrow \#{aU}
                \Rightarrow \#{aaaU}
                \Rightarrow \#{aaaaaU}
                \Rightarrow \#{aaaaaaaU}
                \Rightarrow \#{aaaaaaa}.
          \end{equation*}
    \item Da das Wort gerade Länge hat, ist es nicht über $\#{aU}$ aus $\#S$ ableitbar. Da $1 + x_5 + 1 = 16$ gilt, ist das Wort über $\#{aXa}$ aus $\#S$ ableitbar.

          \begin{forest}
            [\#S, baseline
              [\#a]
              [\#X
                [\#Q
                  [\#a]
                  [\#X
                    [\#Q
                      [\#a]
                      [\#X
                        [\#Q
                          [\#a]
                        ]
                        [\#a]
                        [\#a]
                        [\#a]
                      ]
                      [\#a]
                    ]
                    [\#a]
                    [\#a]
                    [\#a]
                  ]
                  [\#a]
                ]
                [\#a]
                [\#a]
                [\#a]
              ]
              [\#a]
            ]
          \end{forest}
  \end{enumerate}
  \begin{korrektur}
    Es gibt viele Punkte ($4$) zu verteilen.
    \begin{itemize}
    \item Ziehen Sie für Fehler auch viel ab \texttt{;-)}.
    \item $2$ Punkte dafür, dass irgendwie erkennbar ist, wie man zu
      dem Wort kommen kann
    \item $2$ Punkte dafür, dass wirklich ein Ableitungsbaum da steht
    \end{itemize}
    \noindent
    Standardfehler bei Ableitungsbäumen (oder dem, was Studenten dafür
    halten):

    \begin{itemize}
    \item von einem Nichterminal nach unten nicht drei oder vier oder
      \dots Striche, sondern nur einer, also \zB
          \begin{forest}
            [\#S, baseline
              [\#{aXa}
                [\#{Qaaa}
                  [\#{aXa}
                    [\#{Qaaa}
                      [\#{aXa}
                        [\#{Qaaa}
                          [\#a]
                        ]
                      ]
                    ]
                  ]
                ]
              ]
            ]
          \end{forest}

          oder noch kreativere Varianten davon; \\
          dafür $0/2$ Punkte mehr.
        \item Baum passt gar nicht zu dem Produktionen; \\
          je nicht passendem Ableitungsschritt $0.5$ Punkte abziehen
    \end{itemize}
  \end{korrektur}
\end{loesung}

% -----------------------------------------------------------------------------

\end{document}
%%%
%%% Local Variables:
%%% fill-column: 70
%%% mode: latex
%%% TeX-command-default: "XPDFLaTeX"
%%% TeX-master: "korrektur.tex"
%%% End:
