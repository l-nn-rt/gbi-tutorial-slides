\documentclass[12pt]{article}

\input{preamble-aufgaben}
\allowdisplaybreaks
% =======================================================

% =======================================================
%\newcounter{blattnr}
\setcounter{blattnr}{3}
\newcommand{\ausgabetermin}{11.~November 2015}
\newcommand{\abgabetermin}{20.~November 2015}
%\newcommand{\punkteblatt}{13} % Blatt 1
%\newcommand{\punkteblattphysik}{13} % Blatt 1
%\newcommand{\punkteblatt}{17} % Blatt 2
%\newcommand{\punkteblattphysik}{14} % Blatt 2
\newcommand{\punkteblatt}{18} % Blatt 3
\newcommand{\punktetotal}{48} %
\newcommand{\punkteblattphysik}{18} % Blatt 3
\newcommand{\punktetotalphysik}{45} %
% =======================================================

\begin{document}

%\noindent
%Mit \textbf{[nicht Physik]} gekennzeichnete Aufgaben werden von
%Studenten der Physik bitte nicht bearbeitet.\\

% -----------------------------------------------------------------------------

\begin{aufgabe}[2 + 4 = 6]
  Die Zahlen $x_n$, $n \in \N_0$, seien induktiv definiert durch
  \begin{align*}
    x_0 &= 0,\\
    \text{für jedes } n \in \N_+ \colon x_n &= n - x_{n - 1}.
  \end{align*}
  \begin{enumerate}
    \item Geben Sie die Zahlenwerte von $x_1$, $x_2$, $x_3$ und $x_4$ an.
    \item Beweisen Sie durch vollständige Induktion, dass für jedes
          $n \in \N_0$ gilt:
          \begin{equation*}
            x_n = \begin{dcases*}
                    \frac{n}{2},     &falls $n$ gerade,\\
                    \frac{n + 1}{2}, &falls $n$ ungerade.
                  \end{dcases*}
          \end{equation*}
  \end{enumerate}
\end{aufgabe}

\begin{loesung}
  \begin{enumerate}
    \item $x_1 = 1$, $x_2 = 1$, $x_3 = 2$, $x_4 = 2$.
          \begin{korrektur}
            Je Zahlenwert 0,5 Punkte
          \end{korrektur}
    \item \emph{Induktionsanfang}: $x_0 = 0 = \frac{0}{2}$.

          \emph{Induktionsschritt}: Es sei $n \in \N_0$ so, dass gilt:
          \begin{equation}\tag{Induktionsvoraussetzung}
            x_n = \begin{dcases*}
                    \frac{n}{2},     &falls $n$ gerade,\\
                    \frac{n + 1}{2}, &falls $n$ ungerade.
                  \end{dcases*}
          \end{equation}
          Nach Definition von $x_{n + 1}$ im ersten Schritt, der Induktionsvoraussetzung im zweiten Schritt und elementarer Arithmetik in den folgenden Schritten gilt:
          \begin{align*}
            x_{n + 1} &= (n + 1) - x_n\\
                      &= (n + 1) - \begin{dcases*}
                                     \frac{n}{2},     &falls $n$ gerade,\\
                                     \frac{n + 1}{2}, &falls $n$ ungerade,
                                   \end{dcases*}\\
                      &= \begin{dcases*}
                           (n + 1) - \frac{n}{2},     &falls $n$ gerade,\\
                           (n + 1) - \frac{n + 1}{2}, &falls $n$ ungerade,
                         \end{dcases*}\\
                      &= \begin{dcases*}
                           \frac{n}{2} + 1, &falls $n$ gerade,\\
                           \frac{n + 1}{2}, &falls $n$ ungerade,
                         \end{dcases*}\\
                      &= \begin{dcases*}
                           \frac{n + 2}{2}, &falls $n$ gerade,\\
                           \frac{n + 1}{2}, &falls $n$ ungerade,
                         \end{dcases*}\\
                      &= \begin{dcases*}
                           \frac{(n + 1) + 1}{2}, &falls $n$ gerade,\\
                           \frac{n + 1}{2},       &falls $n$ ungerade,
                         \end{dcases*}\\
                      &= \begin{dcases*}
                           \frac{(n + 1) + 1}{2}, &falls $n + 1$ ungerade,\\
                           \frac{n + 1}{2},       &falls $n + 1$ gerade,
                         \end{dcases*}\\
                      &= \begin{dcases*}
                           \frac{n + 1}{2},       &falls $n + 1$ gerade,\\
                           \frac{(n + 1) + 1}{2}, &falls $n + 1$ ungerade.
                         \end{dcases*}\\
          \end{align*}

          \emph{Schlussworte}: Gemäß des Prinzips der vollständigen Induktion
          gilt zu beweisende Aussage.
          \begin{korrektur}
            1 Punkt für den Induktionsanfang

            3 Punkte für den Induktionsschritt, davon 1 Punkt für die Induktionsvoraussetzung
          \end{korrektur}
  \end{enumerate}
\end{loesung}

% -----------------------------------------------------------------------------

\begin{aufgabe}[1 + 1 + 1 = 3]
  \begin{enumerate}
    \item Es sei $w = \#{10011}$. Geben Sie $u = \fNum_2(w)$ und $v = \fNum_3(w)$ an.
    \item Geben Sie $\mu = \fRepr_3(285)$ und $\nu = \fRepr_9(285)$ an.
    \item Das Wort $\mu$ der vorangegangenen Teilaufgabe hat die Länge $6$. Geben Sie $\xi = \fRepr_9(\fNum_3(\mu(0)\mu(1))) \cdot \fRepr_9(\fNum_3(\mu(2)\mu(3))) \cdot \fRepr_9(\fNum_3(\mu(4)\mu(5)))$ und $\zeta = \fNum_9(\xi)$ an.

          \emph{Erinnerung:} Für jedes $i \in \Z_6$ ist $\mu(i)$ das $i$-te Zeichen des Wortes $\mu$.
  \end{enumerate}
\end{aufgabe}

\begin{loesung}
  \begin{enumerate}
    \item $u = \fNum_2(w) = 1 \cdot 2^0 + 1 \cdot 2^1 + 1 \cdot 2^4 = 1 + 2 + 16 = 19$

          $v = \fNum_3(w) = 1 \cdot 3^0 + 1 \cdot 3^1 + 1 \cdot 3^4 = 1 + 3 + 81 = 85$
    \item $\mu = \#{101120}$

          $\nu = \#{346}$


    \item $\xi = \#{346} = \nu$

          $\zeta = 285$
  \end{enumerate}
  \begin{korrektur}
    Je Wort 0,5 Punkte
  \end{korrektur}
\end{loesung}

% -----------------------------------------------------------------------------

\begin{aufgabe}[2 + 4 + 3 = 9]
  Die Abbildung $I$ sei induktiv definiert durch
  \begin{align*}
    I \colon \{\#0, \#1\}^* &\to     \{\#0, \#1\}^*,\\
                   \epsilon &\mapsto \#1,\\
                w \cdot \#0 &\mapsto w \cdot \#1, \text{ wobei } w \in \{\#0, \#1\}^*,\\
                w \cdot \#1 &\mapsto I(w) \cdot \#0, \text{ wobei } w \in \{\#0, \#1\}^*.
  \end{align*}
  \begin{enumerate}
    \item Berechnen Sie $I(\epsilon)$, $I(I(\epsilon))$, $I(I(I(\epsilon)))$ und $I(I(I(I(\epsilon))))$.
    \item Beweisen Sie durch vollständige Induktion über die Wortlänge, dass für jedes $w \in \{\#0, \#1\}^*$ gilt:
          \begin{equation*}
            \text{Es gibt ein } i \in \Z_{|I(w)|} \text{ so, dass } (I(w))(i) = \#1.
          \end{equation*}
          \emph{Erinnerung:} Für jedes $w \in \{\#0, \#1\}^*$ und jedes $i \in \Z_{|I(w)|}$ ist $(I(w))(i)$ das $i$-te Zeichen des Wortes $I(w)$.
    \item Es sei $E = \{u \in \{\#0, \#1\}^* \mid \text{es gibt ein } i \in \Z_{|u|} \text{ so, dass } u(i) = \#1\}$. Nach der vorangegangenen Teilaufgabe gilt $I(w) \in E$ für jedes $w \in \{\#0, \#1\}^*$. Definieren Sie induktiv eine Abbildung $S \colon E \to \{\#0, \#1\}^*$ so, dass für jedes $w \in \{\#0, \#1\}^*$ gilt: $\fNum_2(S(I(w))) = \fNum_2(w)$.
  \end{enumerate}
\end{aufgabe}

\begin{loesung}
  \begin{enumerate}
    \item $I(\epsilon) = \#1$

          $I(I(\epsilon)) = I(\#1) = I(\epsilon \cdot \#1) = I(\epsilon) \cdot \#0 = \#1 \cdot \#0 = \#{10}$

          $I(I(I(\epsilon))) = I(\#(10)) = I(\#1 \cdot \#0) = \#1 \cdot \#1 = \#{11}$

          $I(I(I(I(\epsilon)))) = I(\#{11}) = I(\#1 \cdot \#1) = I(\#1) \cdot \#0 = \#{10} \cdot \#0 = \#{100}$

          \begin{korrektur}
            Je Wort 0,5 Punkte
          \end{korrektur}
    \item Es ist zu zeigen, dass für jedes $n \in \N_0$ gilt:
          \begin{equation*}
            \text{Für jedes } w \in \{\#0, \#1\}^n \text{ gibt es ein } i \in \Z_{|I(w)|} \text{ so, dass } (I(w))(i) = \#1.
          \end{equation*}

          \emph{Induktionsanfang:} Es sei $w \in \{\#0, \#1\}^0$. Dann ist $w = \epsilon$. Foglich ist $I(w) = \#1$. Also ist $(I(w))(0) = \#1$.

          \emph{Induktionsschritt:} Es sei $n \in \N_0$ so, dass gilt:
          \begin{equation}\label{eq:iv}\tag{I.V.}
            \text{Für jedes } u \in \{\#0, \#1\}^n \text{ gibt es ein } i \in \Z_{|I(u)|} \text{ so, dass } (I(u))(i) = \#1.
          \end{equation}
          Weiter sei $w \in \{\#0, \#1\}^{n + 1}$. Dann gibt es ein $u \in \{\#0, \#1\}^n$ und ein $x \in \{\#0, \#1\}$ so, dass $u \cdot x = w$.
          \begin{description}
            \item[Fall 1:] $x = \#0$. Dann ist $I(w) = I(u \cdot x) = u \cdot \#1$. Also ist $(I(w))(|w| - 1) = \#1$.
            \item[Fall 2:] $x = \#1$. Dann ist $I(w) = I(u \cdot x) = I(u) \cdot \#0$. Nach \eqref{eq:iv} gibt es ein $i \in \Z_{|I(u)|}$ so, dass $(I(u))(i) = \#1$. Also ist $(I(w))(i) = (I(u))(i) = \#1$.
          \end{description}
          In jedem Fall gibt es ein $i \in \Z_{|I(w)|}$ so, dass $(I(w))(i) = \#1$.

          \emph{Schlussworte:} Gemäß des Prinzips der vollständigen Induktion gilt zu beweisende Aussage.

          \begin{korrektur}
            1 Punkt für den Induktionsanfang

            3 Punkte für den Induktionsschritt, davon 1 Punkt für die Induktionsvoraussetzung
          \end{korrektur}
    \item Interpretiert man Wörter in $\{\#0, \#1\}^*$ als Zahlen in Binärdarstellung, wobei man das leere Wort als die Zahl $0$ interpretiert, so ist $I(w)$ die Summe von $w$ und $1$. Unter dieser Interpretation bedeutet $S(I(w)) = w$, dass $S(I(w))$ die Differenz von $I(w)$ und $1$ ist. Die Abbildung $S$ muss die \enquote{Transformationen}, die $I$ vornimmt rückgängig machen, lax gesagt, müssen wir um die Definition von $S$ zu erhalten die Pfeile der Form $\mapsto$ in der Definition von $I$ umdrehen. Eine mögliche induktive Definition von $S$ ist:
          \begin{align*}
            S \colon E &\to     \{\#0, \#1\}^*,\\
                   \#1 &\mapsto \epsilon,\\
           w \cdot \#1 &\mapsto w \cdot \#0, \text{ wobei } w \in \{\#0, \#1\}^+,\\
           w \cdot \#0 &\mapsto S(w) \cdot \#1, \text{ wobei } w \in \{\#0, \#1\}^*.
          \end{align*}
          Dies ist tatsächlich wohldefiniert, da der Definitionsbereich von $S$ nur Wörter enthält in denen mindestens eine $\#1$ vorkommt.

          \begin{korrektur}
            Korrekter Ansatz wichtig

            Randfälle nicht bedacht oder Flüchtigkeitsfehler 0,5 oder 1 Punkt Abzug
          \end{korrektur}
  \end{enumerate}
\end{loesung}

% -----------------------------------------------------------------------------

\end{document}
%%%
%%% Local Variables:
%%% fill-column: 70
%%% mode: latex
%%% TeX-command-default: "XPDFLaTeX"
%%% TeX-master: "korrektur.tex"
%%% End:
