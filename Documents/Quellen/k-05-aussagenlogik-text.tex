%=======================================================================
\Tut\chapter{Aussagenlogik}
\label{k:aussagenlogik}

%-----------------------------------------------------------------------
\Tut\section{Informelle Grundlagen}

Im früheren Kapiteln haben wir deutsche Sätze wie \zB den folgenden
geschrieben:
%
\begin{quote}
  "`Die Abbildung $U: A_U \to \N_0$ ist injektiv."'
\end{quote}
%
Das ist eine Aussage. Sie ist \emph{wahr}.
%
\begin{quote}
  "`Die Abbildung $U: A_U \to \N_0$ ist surjektiv."'
\end{quote}
%
ist auch eine Aussage. Sie ist aber \emph{falsch} und deswegen haben
wir sie auch nicht getroffen.

Aussagen sind Sätze, die "`objektiv"' wahr oder falsch
sind. 
%
Allerdings bedarf es dazu offensichtlich einer Interpretation der
Zeichen, aus denen die zugrunde liegende Nachricht zusammengesetzt
ist.

Der klassischen Aussagenlogik liegen zwei wesentliche Annahmen
zugrunde.
%
Die erste ist:
\begin{itemize}
\item  Jede Aussage ist entweder falsch oder wahr.
\end{itemize}
%
Man spricht auch von der \mdefine[Zweiwertigkeit der
Aussagenlogik]{Zweiwertigkeit}\index{Zweiwertigkeit}%
\index{Aussagenlogik!Zweiwertigkeit} der Aussagenlogik.
%
Eine Aussage kann nicht sowohl falsch als auch wahr sein, und es gibt
auch keine anderen Möglichkeiten.
%
Wenn etwas nicht entweder wahr oder falsch ist, dann ist es keine Aussage.
% 
% Um einzusehen, dass es auch umgangssprachliche Sätze gibt, die nicht
% wahr oder falsch sondern sinnlos sind, mache man sich Gedanken zu
% Folgendem:
% %
% "`Ein Barbier ist ein Mann, der genau alle diejenigen Männer rasiert,
% die sich nicht selbst rasieren."'
% %
% Man frage sich insbesondere, ob sich ein Barbier selbst rasiert \dots
Überlegen Sie sich, ob Ihnen Formulierungen einfallen, die keine Aussagen sind.

Außerdem bauen wir natürlich in dieser Vorlesung ganz massiv darauf,
dass es keine Missverständnisse durch unterschiedliche
Interpretationsmöglichkeiten einer Aussage gibt, damit auch klar ist,
ob sie wahr oder falsch ist.
%
% Das ist durchaus nicht selbstverständlich: 
% %
% Betrachten Sie das Zeichen \literal{$\N$}. 
% %
% Das schreibt man üblicherweise für eine Menge von Zahlen. 
% %
% Aber ist bei dieser Menge die $0$ dabei oder nicht? 
% %
% In der Literatur findet man beide Varianten (und zumindest für den
% Autor dieser Zeilen ist nicht erkennbar, dass eine deutlich häufiger
% vorkäme als die andere).

Häufig setzt man aus einfachen Aussagen, im folgenden kurz mit $P$
und $Q$ bezeichnet, kompliziertere auf eine der folgenden Arten
zusammen:

\begin{itemize}
\item \textbf{Negation:} "`Nicht  $P$"' 
  % \\ Dafür schreiben wir auch kurz $\lnot P $.
\item \textbf{logisches Und:} "`$P$ und $Q$"' 
  % \\   Dafür schreiben wir auch kurz $P\land Q$.
\item \textbf{logisches Oder:} "`$P$ oder $Q$"' 
  % \\   Dafür schreiben wir auch kurz $P\lor Q$.
\item \textbf{logische Folgerung:} "`Wenn $P$, dann $Q$"' 
  % \\   Dafür schreiben wir auch kurz $P\alimpl Q$.\todo{oder $P\rightarrowQ$ ?}
\end{itemize}

\noindent
\begin{samepage}
Die zweite Grundlage der Aussagenlogik ist dies:
\begin{itemize}
\item Der Wahrheitswert einer zusammengesetzten Aussage ist durch die
  Wahrheitswerte der Teilaussagen eindeutig festgelegt.
\end{itemize}
\end{samepage}
%
Ob so eine zusammengesetzte Aussage wahr oder falsch ist, soll also
insbesondere \emph{nicht} vom konkreten Inhalt der Aussagen abhängen!
%
Das betrifft auch die aussagenlogische Folgerung.
%
Sind zum Beispiel die Aussagen
\begin{itemize}
\item[$P:$] \emph{"`Im Jahr 2014 wurden in Japan etwa 4.7
    Millionen PkW neu
    zugelassen."'}\footnote{\url{http://de.statista.com/statistik/daten/studie/279897/umfrage/pkw-neuzulassungen-in-japan/}}
  und
\item[$Q$:] \emph{"`Im Jahr 1999 gab es in Deutschland etwa 11.2
    Millionen
    Internet"=Nutzer"'}\footnote{\url{http://de.statista.com/statistik/daten/studie/36146/umfrage/anzahl-der-internetnutzer-in-deutschland-seit-1997/}}
\end{itemize}
%
gegeben, dann soll "`Wenn $P$, dann $Q$"' 
\begin{itemize}
\item erstens tatsächlich eine Aussage sein, die wahr oder falsch ist,
  obwohl natürlich \emph{kein} kausaler Zusammenhang zwischen den
  genannten Themen existiert %(schon gar nicht in die Vergangenheit)
  und
\item zweitens soll der Wahrheitswert dieser Aussage nur von den
  Wahrheitswerten der Aussagen $P$ und $Q$ abhängen (und wie wir sehen
  werden im vorliegenden Beispiel \emph{wahr} sein).
\end{itemize}

\noindent
Da in der Aussagenlogik nur eine Rolle spielt, welche Wahrheitswerte
die elementaren Aussagen haben, aus denen kompliziertere Aussagen
zusammengesetzt sind, beschränkt und beschäftigt man sich dann in der
Aussagenlogik mit sogenannten
\mdefine[aussagenlogische\\Formeln]{aussagenlogischen
  Formeln}\index{aussagenlogische
  Formel}\index{Formel!aussagenlogische}, die nach Regeln ähnlich den
oben genannten zusammengesetzt sind und bei denen statt elementarer
Aussagen einfach \mdefine{Aussagevariablen}\index{Aussagevariablen}
stehen, die als Werte "`wahr"' und "`falsch"' haben können.

Der Rest dieses Kapitels ist wie folgt aufgebaut:
%
In Abschnitt~\ref{sec:aussagenlogik-syntax} werden wir definieren, wie
aussagenlogische Formeln syntaktisch aufgebaut sind.
%
%Wir werden das aus didaktischen Gründen auf zwei unterschiedlichen
%Wegen definieren.
%%%% nein das machen wir nicht
%
Wir werden das zunächst auf eine Art und Weise tun, für die uns schon
alle Hilfsmittel zur Verfügung stehen.
%
Eine andere, in der Informatik weit verbreitete Vorgehensweise werden
wir im Kapitel über kontextfreie Grammatiken kennenlernen.
%
In Abschnitt~\ref{sec:boolesche-funktionen} führen wir sogenannte
boolesche Funktionen ein.
%
Sie werden dann in Abschnitt~\ref{sec:aussagenlogik-semantik} benutzt,
um die Semantik, \dh die Bedeutung, aussagenlogischer Formeln zu
definieren.

In diesem Kapitel orientieren wir uns stark am Skriptum zur Vorlesung
"`Formale Systeme"' von \textcite{Schmitt_2013_FS_tr}.

%-----------------------------------------------------------------------
\section{Syntax aussagenlogischer Formeln}
\label{sec:aussagenlogik-syntax}

Grundlage für den Aufbau aussagenlogischer Formeln ist ein Alphabet
\[
  \AAL = \{ \alka, \alkz, \alnot, \aland, \alor, \alimpl \} \cup \AALV \;.
\]
%
Die vier Symbole $\alnot$, $\aland$, $\alor$ und $\alimpl$ heißen auch
\mdefine[aussagen-\\logische\\Konnektive]{aussagenlogische Konnektive}%
\index{Konnektive!aussagenlogische}.
%
Und $\AALV$ ist ein Alphabet, dessen Elemente wir
\emph{aussagenlogische Variablen} nennen.
%
Wir notieren sie in der Form $\alP_i$, mit $i\in\N_0$.
%
Weil schon bei kleinen Beispielen die Lesbarkeit durch die Indizes
unnötig beeinträchtigt wird, erlauben wir, als Abkürzungen für
$\alP_0$,
$\alP_1$,
$\alP_2$
und $\alP_3$
einfach $\alP$, $\alv{Q}$, $\alv{R}$ und $\alv{S}$ zu benutzen.


Die Menge der syntaktisch korrekten aussagenlogischen Formeln soll nun
definiert werden als eine formale Sprache $\LAL$ über einem Alphabet
$\AAL$, also als Teilmenge von $\AAL^*$.
%
% Formal ergeben sich für verschiedene Mengen von Aussagevariablen
% auch verschiedene Mengen von Formeln. 
% %
% Es wird aber nicht zu Problemen führen, wenn wir das ignorieren. % Stimmt das?

Eine erste Forderung ist, dass stets $\AALV\subseteq \LAL$ sein soll.
%
Damit ist also jede aussagenlogische Variable eine aussagenlogische
Formel.

Als zweites wollen wir nun definieren, dass man aus "`einfacheren"'
aussagenlogischen Formeln "`kompliziertere"' zusammensetzen kann.
%
Für die verschiedenen Möglichkeiten definieren wir uns vier Funktionen:
%
\begin{alignat*}{3}
  f_{\alnot}  &:& \AAL^*          & \to \AAL^* :& G      &\mapsto \alka\alnot G\alkz \\
  f_{\aland}  &:& \AAL^*\x \AAL^* & \to \AAL^* :& (G, H) &\mapsto \alka G\aland H\alkz \\
  f_{\alor}   &:& \AAL^*\x \AAL^* & \to \AAL^* :& (G, H) &\mapsto \alka G\alor H\alkz \\
  f_{\alimpl} &:& \AAL^*\x \AAL^* & \to \AAL^* :& (G, H) &\mapsto \alka G\alimpl H\alkz 
\end{alignat*}
%
Wie man sieht, sind die Funktionswerte jeweils definiert durch
geeignete Konkatenationen von Argumenten und Symbolen aus dem Alphabet
der Aussagenlogik.
%
Ein sinnvolles Beispiel ist
\[
  f_{\aland}(\; \alka\alv{P}\alimpl\alv{Q}\alkz\;,\;  \alv{R}\;) = \alka\alka\alv{P}\alimpl\alv{Q}\alkz\aland \alv{R}\alkz
\] 
%
Ein für unser Vorhaben sinnloses Beispiel ist
\[
  f_{\aland}(\alnot\alimpl\alkz,  \alv{R}\alor) = \alka \alnot\alimpl\alkz\aland \alv{R}\alor\alkz
\]
%
Wie man sieht, führt jede Anwendung einer der Abbildungen dazu, dass
die Anzahl der Konnektive um eins größer wird als die Summe der
Anzahlen der Konnektive in den Argumenten.

Für die aussagenlogischen Formeln ist es üblich, sie wie folgt zu
lesen:
%
\begin{center}
  \begin{tabular}{>{$}r<{$}>{"`}l<{"'}l}
    \alka\alnot G\alkz & nicht $G$ \\
    \alka G\aland H\alkz & $G$ und $H$ \\
    \alka G\alor H\alkz & $G$ oder $H$ \\
    \alka G\alimpl H\alkz & $G$ impliziert $H$ & (oder "`aus $G$ folgt $H$"')
  \end{tabular}
\end{center}
%
Wir definieren nun induktiv unendlich viele Mengen wie folgt:
%
\begin{align*}
  M_0 &= \AALV \\
  \text{für jedes }n\in\N_0: M_{n+1} &= M_n \cup f_{\alnot}(M_n) \\
  &\mathrel{\hphantom{= M_n}} \cup f_{\aland}(M_n\x M_n) \cup f_{\alor}(M_n\x M_n) \cup f_{\alimpl}(M_n\x M_n) \\
  \LAL &= \bigcup_{i\in\N_0} M_i
\end{align*}
%
Die formale Sprache $\LAL$ ist die \mdefine[Menge der\\
aussagenlogischen\\Formeln]{Menge der aussagenlogischen Formeln}%
\index{Formel!aussagenlogische}\index{aussagenlogische Formel} für die
Variablenmenge $\AALV$.
%
Wie man anhand der Konstruktion in der zweiten Zeile sieht, ist für
jedes $n\in\N_0$ stets $M_n\subseteq M_{n+1}$.
%
Die Mengen werden also "`immer größer"'.
%
Außerdem enthält jedes $M_n$
und damit auch $\LAL$
nur Aussagevariablen sowie Werte der vier involvierten Abbildungen,
die entstehen können, wenn die Argumente aussagenlogische Formeln
sind.
%
Wenn man also für alle aussagenlogischen Formeln etwas definieren
möchte, dann genügt es auch, das nur für Aussagevariablen sowie Werte
der vier involvierten Abbildungen zu tun, die entstehen können, wenn
die Argumente aussagenlogische Formeln sind.

Um die Konstruktion von $\LAL$
noch ein bisschen besser zu verstehen, beginnen wir der Einfachheit
halber einmal mit $\AALV = \{ \alv{P}, \alv{Q} \}$.
%
Dann ist also
%
\[
  M_0 = \{ \alv{P}, \alv{Q} \} \;.
\]
%
Daraus ergeben sich
%
\begin{align*}
  f_{\alnot}(M_0) &= \{  \alka\alnot\alv{P}\alkz, \alka\alnot\alv{Q}\alkz  \} \\
  f_{\aland}(M_0\x M_0) &= \{  \alka\alv{P}\aland\alv{P}\alkz, \alka\alv{P}\aland\alv{Q}\alkz,
                          \alka\alv{Q}\aland\alv{P}\alkz, \alka\alv{Q}\aland\alv{Q}\alkz \} \\
  f_{\alor}(M_0\x M_0) &= \{  \alka\alv{P}\alor\alv{P}\alkz, \alka\alv{P}\alor\alv{Q}\alkz,
                          \alka\alv{Q}\alor\alv{P}\alkz, \alka\alv{Q}\alor\alv{Q}\alkz \} \\
  f_{\alimpl}(M_0\x M_0) &= \{  \alka\alv{P}\alimpl\alv{P}\alkz, \alka\alv{P}\alimpl\alv{Q}\alkz,
                          \alka\alv{Q}\alimpl\alv{P}\alkz, \alka\alv{Q}\alimpl\alv{Q}\alkz \} 
\end{align*}
%
und folglich ist
%
\begin{align*}
  M_1 = \{ &\alv{P}, \alv{Q}, \\
           &\alka\alnot\alv{P}\alkz, \alka\alnot\alv{Q}\alkz,   \\
           &\alka\alv{P}\aland\alv{P}\alkz, \alka\alv{P}\aland\alv{Q}\alkz,
             \alka\alv{Q}\aland\alv{P}\alkz, \alka\alv{Q}\aland\alv{Q}\alkz,  \\
           &\alka\alv{P}\alor\alv{P}\alkz, \alka\alv{P}\alor\alv{Q}\alkz,
             \alka\alv{Q}\alor\alv{P}\alkz, \alka\alv{Q}\alor\alv{Q}\alkz, \\
           &\alka\alv{P}\alimpl\alv{P}\alkz, \alka\alv{P}\alimpl\alv{Q}\alkz,
             \alka\alv{Q}\alimpl\alv{P}\alkz, \alka\alv{Q}\alimpl\alv{Q}\alkz \} 
\end{align*}
%
Für $M_2$ ergeben sich bereits so viele Formeln, dass man sie gar
nicht mehr alle hinschreiben will.
%
Beispiele sind $\alka \alnot \alka\alnot\alv{P}\alkz \alkz$,
$\alka\alka\alv{P}\alimpl\alv{Q}\alkz \aland
\alv{P} \alkz$
oder $\alka \alv{Q} \alor \alka\alnot \alv{Q}\alkz \alkz$.

Weil es nützlich ist, führen wir noch eine Abkürzung ein.
%
Sind $G$
und $H$
zwei aussagenlogische Formeln, so stehe $\alka G \aleqv H \alkz$
für
$\alka \alka G \alimpl H \alkz \aland \alka H \alimpl G \alkz \alkz$.

So, wie wir aussagenlogische Formeln definiert haben, ist zwar
$\alka\alv{P}\alimpl\alv{Q}\alkz$ eine, aber $\alv{P}\alimpl\alv{Q}$
nicht.
%
(Wie könnte man das leicht beweisen?)
%
Und bei größeren Formeln verliert man wegen der vielen Klammern leicht
den Überblick.
%
Deswegen erlauben wir folgende Abkürzungen bei der Notation
aussagenlogischer Formeln.
%
(Ihre "`offizielle"' Syntax bleibt die gleiche!)
%
\begin{itemize}
\item Die äußerten umschließenden Klammern darf man immer weglassen.
  %
  Zum Beispiel ist $\alv{P}\alimpl\alv{Q}$ die Kurzform von
  $\alka\alv{P}\alimpl\alv{Q}\alkz$.
\item Wenn ohne jede Klammern zwischen mehrere Aussagevariablen immer
  das gleiche Konnektiv steht, dann bedeute das "`implizite
  Linksklammerung"'.
  %
  Zum Beispiel ist $\alv{P}\aland\alv{Q}\aland\alv{R}$ die Kurzform von
  $\alka\alka\alv{P}\aland\alv{Q}\alkz\aland\alv{R}\alkz$.
\item Wenn ohne jede Klammern zwischen mehrere Aussagevariablen
  verschiedene Konnektive stehen, dann ist von folgenden
  "`Bindungsstärken"' der Konnektive auszugehen:
  \begin{enumerate}[a)]
  \item $\alnot$ bindet am stärksten
  \item $\aland$ bindet am zweitstärksten
  \item $\alor$ bindet am drittstärksten
  \item $\alimpl$ bindet am viertstärksten
  \item $\aleqv$ bindet am schwächsten
  \end{enumerate}
  % 
  Zum Beispiel ist
  $\alv{P}\alor\alv{R}\alimpl\alnot\alv{Q}\aland\alv{R}$ die Kurzform
  von
  $\alka \alka\alv{P}\alor\alv{R}\alkz \alimpl %
         \alka \alka\alnot\alv{Q}\alkz \aland\alv{R}\alkz 
   \alkz$.
\end{itemize}
%
\begin{tutorium}
  \paragraph{Klammersparregeln bei aussagenlogischen Formeln}

  Beispiele
  \begin{itemize}
  \item $\alv{P}\alor\alv{Q}\aland\alv{R}$ steht für %
    $\alka \alv{P}\alor \alka\alv{Q}\aland\alv{R}\alkz \alkz$
  \end{itemize}
\end{tutorium}

%-----------------------------------------------------------------------
\Tut\section{Boolesche Funktionen}
\label{sec:boolesche-funktionen}

\begin{window}[0,r,\includegraphics[width=30mm]{../k-05-aussagenlogik/George_Boole_color},{}]
  \noindent
  Der britische Mathematiker George Boole (1815--1864) hat mit seinem
  heutzutage \href{http://www.gutenberg.org/ebooks/15114}{online}
  verfügbaren Buch
  "`\citetitle{Boole_1854_LT_bk}"' \parencite{Boole_1854_LT_bk} die
  Grundlagen für die sogenannte \emph{algebraische Logik} oder auch
  \emph{boolesche Algebra} gelegt.
  %
  Dabei griff er die weiter vorne erwähnte Grundlage der Aussagenlogik
  auf, dass der Wahrheitswert einer Aussage nur von den
  Wahrheitswerten seiner Teilformeln abhängt.
  %
  Was Boole neu hinzunahm, war die Idee, in Anlehnung an arithmetische
  Ausdrücke auch "`boolesche Ausdrücke"' einschließlich "`Variablen"'
  zu erlauben und solche Ausdrücke umzuformen analog zu den gewohnten
  Umformungen bei arithmetischen Ausdrücken.
\end{window}

Im weiteren Verlauf schreiben wir für die Wahrheitswerte "`wahr"' und
"`falsch"' kurz \W und \F und bezeichnen die Menge mit diesen beiden
Werten als $\Bool=\{\W,\F\}$.

Eine \mdefine{boolesche Funktion}\index{Funktion!boolesche}%
\index{boolesche Funktion} ist eine Abbildung der Form
$f: \Bool^n \to \Bool$.
%
In der nachfolgenden Tabelle sind einige "`typische"' boolesche
Funktionen aufgeführt, die wir zunächst einmal mit $b_{\alnot}$,
$b_{\aland}$, $b_{\alor}$ und $b_{\alimpl}$ bezeichnen:

\begin{center}
  \begin{tabular}{*{2}{>{$}c<{$}}|*{4}{>{$}c<{$}}}
    \toprule
    x_1 & x_2 & b_{\alnot}(x_1) & b_{\aland}(x_1,x_2) & b_{\alor}(x_1,x_2) & b_{\alimpl}(x_1,x_2) \\
    \midrule
    \F & \F & \W & \F & \F & \W \\
    \F & \W & \W & \F & \W & \W \\
    \W & \F & \F & \F & \W & \F \\
    \W & \W & \F & \W & \W & \W \\
    \bottomrule
  \end{tabular}
\end{center}
%
Üblicher ist es allerdings, die Abbildungen $b_{\aland}$
und $b_{\alor}$
mit einem Infixoperator zu schreiben und  $b_{\alnot}$
wird auch anders notiert.
%
Dabei findet man (mindestens) zwei Varianten.
%
In der nachfolgenden Tabelle sind die Möglichkeiten nebeneinander
gestellt und auch gleich noch übliche Namen \bzw Sprechweisen mit
angegeben:
%
\begin{center}
  \begin{tabular}{*{3}{>{$}c<{$}}@{\quad}l}
    b_{\alnot}(x) & \lnot x & \bar{x} & Negation \bzw Nicht\\
    b_{\aland}(x,y) &  x\land y & x\cdot y & Und \\
    b_{\alor}(x,y) &  x\lor y & x+y & Oder \\
    b_{\alimpl}(x,y) &   &  & Implikation 
  \end{tabular}
\end{center}
%
Die an Arithmetik erinnernde Notationsmöglichkeit geht oft einher mit
der Benutzung von $0$ statt $\F$ und $1$ statt $\W$.

Das meiste, was durch die Definitionen dieser booleschen Abbildungen
ausgedrückt wird, ist aus dem alltäglichen Leben vertraut. 
%
Nur auf wenige Punkte wollen wir explizit eingehen:
%
\begin{itemize}
\item Das "`Oder"' ist "`inklusiv"' (und nicht "`exklusiv"'): Auch
  wenn $x$ und $y$ \emph{beide} wahr sind, ist $x\alor y$ wahr.
\item Bei der Implikationsabbildung ist $b_{\alimpl}(x,y)$
  auf jeden Fall wahr, wenn $x=\F$
  ist, unabhängig vom Wert von $y$,
  insbesondere auch dann, wenn $y=\F$
  ist.
\item Für jedes $x\in\Bool$
  haben beiden Abbildungen $b_{\aland}$
  und $b_{\alor}$
  die Eigenschaft, dass $b_{\aland}(x,x)=b_{\alor}(x,x)=x$
    ist, und für $b_{\alimpl}$ gilt $b_{\alimpl}(x,x)=\W$.
\end{itemize}
%
Natürlich gibt es zum Beispiel insgesamt $\cramped{2^{2\cdot 2}}=16$
zweistellige boolesche Funktionen.
%
Die obigen sind aber ausreichend, um auch jede der anderen durch
Hintereinanderausführung der genannten zu realisieren.
%
Zum Beispiel ist $\Bool^2\to \Bool: (x,y) \mapsto b_{\alor}(b_{\alnot}(x), x)$
die Abbilung, die konstant $\W$ ist.

%Und es ist auch für jedes $x,y\in\Bool$:
%$\lnot((\lnot x \land (\lnot y))= x\lor y$.
%
Für die Abbildung $b_{\alimpl}$
hat sich übrigens keine Operatorschreibweise durchgesetzt.
%
Man kann sich aber klar machen, dass für jedes $x,y\in\Bool$ gilt:
$b_{\alimpl}(x,y) = b_{\alor}(b_{\alnot}(x), y)$ ist.
%
Eine Möglichkeit besteht darin, in einer Tabelle für alle möglichen
Kombinationen von Werten für $x$
und $y$ die booleschen Ausdrücke auszuwerten:
%
\begin{center}
  \begin{tabular}{*{5}{>{$}c<{$}}}
    \toprule
    x  & y  & b_{\alnot}(x) & b_{\alor}(b_{\alnot}(x), y) & b_{\alimpl}(x,y) \\
    \midrule
    \F & \F & \W      & \W & \W \\
    \F & \W & \W      & \W & \W \\
    \W & \F & \F      & \F & \F \\
    \W & \W & \F      & \W & \W \\
    \bottomrule
  \end{tabular}
\end{center}

%-----------------------------------------------------------------------
\Tut\section{Semantik aussagenlogischer Formeln}
\label{sec:aussagenlogik-semantik}

Unser Ziel ist es nun, jeder aussagenlogischen Formel im wesentlichen
eine boolsche Funktion als Bedeutung zuzuordnen.

Eine \mdefine{Interpretation}\index{Interpretation} einer Menge $V$
von Aussagevariablen ist eine Abbildung $I: V \to \Bool$.
%
Die Menge aller Interpretationen einer Variablenmenge $V$ ist also
$\Bool^V$.
%
Das kann man sich \zB vorstellen als eine Tabelle mit je einer Spalte
für jede Variable $X\in V$ und je einer Zeile für jede Interpretation
$I$, wobei der Eintrag in "`Zeile $I$"' und "`Spalte $X$"' gerade
$I(X)$ ist. Abbildung~\ref{fig:tab-alle-interpretationen} zeigt
beispielhaft den Fall $V=\{\alP_1, \alP_2, \alP_3\}$.
%
Machen Sie sich klar, dass es für jede Variablenmenge mit $k\in\N_+$
Aussagevariablen gerade $2^k$ Interpretationen gibt.
%
\begin{figure}[ht]
  \centering
  \begin{tabular}{*{3}{c}}
    \toprule
    $\alP_1$ & $\alP_2$ & $\alP_3$ \\
    \midrule
    \F & \F & \F \\
    \F & \F & \W \\
    \F & \W & \F \\
    \F & \W & \W \\
    \W & \F & \F \\
    \W & \F & \W \\
    \W & \W & \F \\
    \W & \W & \W \\
    \bottomrule
  \end{tabular}
  \caption{Alle Interpretationen der Menge $V=\{\alP_1, \alP_2, \alP_3\}$ von Aussagevariablen.}
  \label{fig:tab-alle-interpretationen}
\end{figure}
%
\begin{tutorium}
  \paragraph{Interpretationen von Mengen von Aussagevariablen}

  klar machen, dass es für jede Variablenmenge mit $k\in\N_+$
  Aussagevariablen gerade $2^k$ Interpretationen gibt.
  %
  \begin{itemize}
  \item Fälle $k=1,2,3$ betrachten
  \item Wieviele Interpretationen gibt es bei $k+1$ Variablen im
    Vergleich zu $k$ Variablen?
  \end{itemize}
\end{tutorium}

Jede Interpretation $I$ legt eine Auswertung $\val_I(F)$ jeder
aussagenlogischen Formel $F\in\LAL$ fest: 
%
Für jedes $X\in\AALV$
und für jede aussagenlogische Formel $G$ und $H$ gelte:
%
\begin{align*}
  \val_I(X)         &= I(X) \\
  \val_I(\alnot G)   &= b_{\alnot}(\val_I(G)) \\
  \val_I(G \aland H) &= b_{\aland}(\val_I(G), \val_I(H)) \\
  \val_I(G \alor H)  &= b_{\alor}(\val_I(G)  \val_I(H)) \\
  \val_I(G \alimpl H)&= b_{\alimpl}(\val_I(G), \val_I(H))
\end{align*}
%
\begin{tutorium}
  \paragraph{Auswertung von Formeln:}
  meistens macht man das gleich für alle Interpretationen, wobei man
  sich nur die Aussagevariablen hinschreibt, die auch in der Formel
  vorkommen.
  \begin{itemize}
  \item Wenn man größere Formeln "`auswerten"' will, dann kann man
    Wahrheitswerte unter die Konnektive schreiben:
    \begin{enumerate}
    \item Wahrheitswerte für die Variablen:

      \begin{tabular}[t]{ccccc}
        \toprule
        $\alka\alv{P}$ & $\aland$ & $\alv{Q}\alkz$ & $\alor$ & $\alv{P}$\\
        \midrule
        \F && \F && \F \\
        \F && \W && \F\\
        \W && \F && \W\\
        \W && \W && \W\\
        \bottomrule
      \end{tabular}

    \item Wahrheitswerte für die Teilformel $(G\aland H)$:

      \begin{tabular}[t]{ccccc}
        \toprule
        $\alka\alv{P}$ & $\aland$ & $\alv{Q}\alkz$ & $\alor$ & $\alv{P}$\\
        \midrule
        \F &\F& \F && \F \\
        \F &\F& \W && \F\\
        \W &\F& \F && \W\\
        \W &\W& \W && \W\\
        \bottomrule
      \end{tabular}

    \item Wahrheitswerte für die ganze Formel

      \begin{tabular}[t]{ccccc}
        \toprule
        $\alka\alv{P}$ & $\aland$ & $\alv{Q}\alkz$ & $\alor$ & $\alv{P}$\\
        \midrule
        \F &\F& \F &\F& \F \\
        \F &\F& \W &\F& \F\\
        \W &\F& \F &\W& \W\\
        \W &\W& \W &\W& \W\\
        \bottomrule
      \end{tabular}

    \item Man sehe die Äquivalenz von  $\alka\alv{P}\aland \alv{Q}\alkz\alor \alv{P}$ und $\alv{P}$.
    \end{enumerate}
  \end{itemize}
\end{tutorium}
%
Das meiste, was hier zum Ausdruck gebracht wird, ist aus dem
alltäglichen Leben vertraut. 
%
Nur auf wenige Punkte wollen wir explizit eingehen:
%
\begin{itemize}
\item Man kann für komplizierte Aussagen anhand der obigen Definition
  "`ausrechnen"', ob sie für eine Interpretation wahr oder falsch ist,
  weil jede aussagenlogische Formel nur durch Anwendung von genau
  einer erlaubten Abbildungen entstehen kann.
  %
  Für die Interpretation $I$
  mit $I(\alv{P})=\W$
  und $I(\alv{Q})=\F$
  ergäbe die Anwendung der obigen Definition von $\val_I$
  auf die Formel $\alnot\alka\alv{P}\aland\alv{Q}\alkz$ \zB schrittweise:
  \begin{align*}
    \val_I(\alnot\alka\alv{P}\aland\alv{Q}\alkz) 
    &= \lnot ( \val_I(\alv{P}\aland\alv{Q}) ) \\
    &= \lnot ( \val_I(\alv{P}) \land \val_I(\alv{Q}) )  \\
    &= \lnot ( I(\alv{P}) \land I(\alv{Q}) ) \\
    &= \lnot (\W \land \F)  \\
    &= \lnot (\F)  \\
    &= \W  \\
  \end{align*}


  Oft interessiert man sich dafür, was für \emph{jede} Interpretation
  der Variablen einer Formel passiert.
  %
  Dann ergänzt man die Tabelle aller Interpretationen um weitere Spalten
  mit den Werten von $\val_I(G)$.
  %
  Hier ist ein Beispiel:\\

  \begin{tabular}{*{7}{c}}
    \toprule
    $\alv{P}$ & $\alv{Q}$ & $\alnot\alv{P}$ & $\alnot\alv{Q}$ 
    & $\alnot\alv{P}\alor\alnot\alv{Q}$ & $\alv{P}\aland\alv{Q}$ & $\alnot\alka\alv{P}\aland\alv{Q}\alkz$ \\
    \midrule
    \F & \F & \W & \W & \W & \F & \W \\
    \F & \W & \W & \F & \W & \F & \W \\
    \W & \F & \F & \W & \W & \F & \W \\
    \W & \W & \F & \F & \F & \W & \F \\
    \bottomrule
  \end{tabular}\\
  

\item Die Tabelle zeigt, dass die Aussagen
  $\alnot\alka\alv{P}\aland \alv{Q}\alkz$
  und $\alnot \alv{P} \alor \alnot \alv{Q} $
  \emph{für alle Interpretationen} denselben Wahrheitswert annehmen.
  % 
  Solche Aussagen nennt man
  \mdefine[äquivalente\\Aussagen]{äquivalent}\index{äquivalente
    Aussagen}\index{Aussagen!äquivalente}.
  %
  Gleiches gilt für $\alnot\;\alnot \alv{P}$
  und $\alv{P}$ und viele weitere Paare von aussagenlogischen Formeln.

  Wenn zwei Formeln $G$
  und $H$ äquivalent sind, dann schreiben wir auch $G\equiv H$.
\item Noch eine Anmerkung zur Implikation.
  %
  So wie $b_{\alimpl}$
  definiert wurde ist eine Aussage $\alv{P}\alimpl \alv{Q}$
  genau dann wahr, wenn $\alv{P}$ falsch ist oder $\alv{Q}$ wahr.
  %
  Warum das sinnvoll ist, wird unter Umständen noch etwas klarer, wenn
  man überlegt, wie man denn die Tatsache umschreiben wollte, dass
  eine Implikation im naiven Sinne, also "`aus $\alv{P}$
  folgt $\alv{Q}$"',
  \emph{nicht} zutrifft. Das ist doch wohl dann der Fall, wenn zwar
  $\alv{P}$
  zutrifft, aber $\alv{Q}$
  nicht. Also sollte $\alv{P}\alimpl \alv{Q}$
  äquivalent zu $\alnot\alka\alv{P}\aland \alnot \alv{Q}\alkz$,
  und das ist nach obigem äquivalent zu
  $\alka\alnot \alv{P}\alkz \alor \alka\alnot\;\alnot \alv{Q}\alkz$
  und das zu $\alka\alnot \alv{P}\alkz \alor \alv{Q}$.
\end{itemize}
%
\begin{tutorium}
  \paragraph{Implikation}

  \begin{itemize}
  \item  ausführlich erklärt; sehen Sie sich bitte
    die Folien noch mal an.
  \item wesentlich: $\alv{P}\alimpl \alv{Q}$
    ist äquivalent zu $\alnot \alv{P} \alor \alv{Q}$
  \item Auswirkung auf Beweis von Aussagen der Form $A\alimpl B$: Man muss nur
    etwas tun, wenn $A$ wahr ist.
    (so etwas wird sehr oft vorkommen)
  \end{itemize}
\end{tutorium}
%
\begin{tutorium}
  \paragraph{Äquivalenz von aussagenlogischen Formeln}
  \begin{itemize}
  \item Man bespreche noch einmal, was äquivalente Aussagen sind.
  \item Beachte: Äquivalente Aussagen enthalten "`meistens"' die
    gleichen Aussagevariablen:
    \begin{itemize}
    \item Die Formeln $\alv{P}$ und $\alv{Q}$ sind nicht äquivalent.
    \item Denn es kann ja $\alv{P}$ wahr sein und $\alv{Q}$ falsch.
    \item Ausnahmen sind so etwas wie \zB $\alv{P}\aland \alnot \alv{P}$ und $\alv{Q} \aland
      \alnot \alv{Q}$
    \end{itemize}
  \end{itemize}    
\end{tutorium}
%
Man kann nun noch einen letzten Schritt tun und mit jeder
aussagenlogische Formel $G$
eine Abbildung assoziieren, die für jede (passende) Interpretation
die ausgewertete Formel als Wert zuweist.
%
Wir gehen in zwei Schritten vor.
%
Als erstes sei $V$ eine Menge von aussagelogischen Variablen, die alle in $G$
vorkommenden Variablen enthält. 
%
Dann kann man die folgende Abbildung definieren:
\[
  \Bool^V \to \Bool : I \mapsto \val_I(G) \;.
\]
%
Das ist ganz genau genommen noch keine boolesche Abbildung, weil der
Definitionsbereich nicht ein kartesisches Produkt $\Bool^k$
ist, sondern eben die Menge aller Abbildungen von $V$ in $\Bool$.

Konsequenz ist auch, dass zum Beispiel die Formeln
$G=\alP_0\aland\alP_0$
und $H= \alP_2\aland\alP_2$ \emph{nicht} äquivalent sind.
%
Das sieht man an einer Interpretation $I$
mit $I(\alP_0)=\W$ und $I(\alP_2)=\F$.
%
Für sie ist $\val_I(G)=\W$, aber $\val_I(H)=\F$.
%
Man kann sich aber durchaus auf den Standpunkt stellen, dass es einen
"`Kern"' gibt, der von den konkreten Namen der Aussagevariablen
unabhängig und bei beiden Formeln gleich ist.

Um diesen "`Kern"' herauszuarbeiten, kann man verschiedene Wege gehen.
%
Einer, dem wir nicht folgen werden, besteht darin, zu verlangen, dass
in jeder Formel $G$
für eine geeignete nichtnegative ganze Zahl $k$
die Nummern der in $G$
vorkommenden Aussagevariablen gerade die Zahlen in $\Z_n$ sein müssen.
%
Falls das nicht der Fall ist, müsste man also immer Variablen
umbenennen.

Wir bevorzugen eine Alternative, bei der man sozusagen auch noch die
Namen der Aussagevariablen "`vergisst"'.
%
Das formalisieren wir so:
%
Da wir vorgesetzt haben, dass alle Aussagevariablen von der Form
$\alv{P}_i$
sind, können wir den Variablen in $V$ auch eine Reihenfolge geben.
%
Dazu seien einfach für $k=|V|\geq 1$
die Zahlen $i_1, \ldots, i_k\in\N_0$
die Indizes der Variablen in $V$,
und zwar seien sie so gewählt, dass für jedes $j\in\P_{k-1}$
gelte: $i_j<i_{j+1}$.
%
Jedem $k$-Tupel
$x=(x_1,\dots,x_k)\in\Bool^k$
entspricht dann in naheliegender Weise die Interpretation
\[
  I_x: \alv{P}_{i_j} \mapsto x_j \;.
\]
%
Damit ist dann die durch eine aussagenlogische Formel $G$
beschriebene Abbildung diejenige mit
$\Bool^k \to \Bool : x \mapsto \val_{I_x}(G)$.

Ist $I$
eine Interpretation für eine aussagenlogische Formel $G$,
dann nennen wir $I$
ein \mdefine[Modell einer Formel]{Modell}\index{Modell!für
  aussagenlogische Formeln} von $G$, wenn $\val_I(G)=\W$ ist.
%
Ist $I$
eine Interpretation für eine Menge $\Gamma$
aussagenlogischer Formeln, dann nennen wir $I$
ein \mdefine[Modell einer Formelmenge]{Modell} von $\Gamma$,
wenn $I$ Modell jeder Formel $G\in \Gamma$ ist.

Ist $\Gamma$
eine Menge aussagenlogischer Formeln und $G$
ebenfalls eine, so schreibt man auch genau dann $\Gamma\models G$,
wenn jedes Modell von $\Gamma$ auch Modell von $G$ ist.
%
Enthält $\Gamma=\{H\}$
nur eine einzige Formel, schreibt man einfach $H\models G$.
%
Ist $\Gamma=\{\}$ die leere Menge, schreibt man einfach $\models G$.
%
Die Bedeutung soll in diesem Fall sein, dass $G$
für \emph{alle} Interpretationen überhaupt wahr ist, \dh dass $G$ eine
Tautologie ist.


Es zeigt sich, dass in verschiedenen Teilen der Informatik zwei Sorten
aussagenlogischer Formeln von Bedeutung sind.
%
Die eine wichtige Klasse von Formeln sind
\mdefine[erfüllbare\\Formel]{erfüllbare Formeln}, \dh Formeln, die für
mindestens eine Interpretation wahr sind.
%
Die Untersuchung aussagenlogischer Formeln auf Erfüllbarkeit spielt in
einigen Anwendungen eine große Rolle.
%
Für manche Formeln ist das ganz einfach, für andere anscheinend nicht.
%
Vergleichen Sie einfach mal die Formeln
$\alka\alv{P}\aland\alnot\alv{Q}\alkz\alor\alka\alnot\alv{P}\aland\alv{R}\alkz$
und
$\alka\alv{P}\alor\alnot\alv{Q}\alkz\aland\alka\alnot\alv{P}\alor\alv{R}\alkz$
in dieser Hinsicht.
%
In der Vorlesung "`Theoretische Grundlagen der Informatik"' werden Sie
mehr zum Thema Erfüllbarkeit erfahren.

Das andere sind \mdefine[Tautologie]{Tautologien}\index{Tautologie};
das sind Formeln, für die \emph{jede} Interpretation ein Modell ist,
die also für jede Interpretation wahr sind.
%
Sie heißen auch \mdefine[allgemeingültige\\Formel]{allgemeingültige Formeln}.%
\index{allgemeingültige Formel}\index{Formel!allgemeingültige}
%
Solche Formeln gibt es, \zB $\alv{P} \alor\alnot\alv{P}$
oder $\alv{P}\alimpl\alka\alv{Q}\alimpl\alv{P}\alkz$.
%
Sie werden auch im nächsten Abschnitt noch eine wichtige Rolle spielen.

Zum Abschluss dieses Abschnitts geht es erst einmal darum, eine ganze
Reihe von Tautologien kennenzulernen und auch Methoden, sich neue zu
beschaffen.

Als erstes betrachten wir eine Formel der Form $G\aleqv H$.
%
Das wurde eingeführt als Abkürzung für $G\alimpl H \aland H\alimpl G$.
%
Für jede Interpretation $I$ ist
\begin{align*}
  \val_I(G\alimpl H \aland H\alimpl G) 
  &= \val_I(G\alimpl H) \land \val_I(H\alimpl G)\\
  &= (\lnot\val_I(G)\lor\val_I(H)) \land (\lnot\val_I(H)\lor\val_I(G))
\end{align*}
%
Falls nun $G$
und $H$
äquivalente Formeln sind, also für jede Interpretation $I$
$\val_I(G)=\val_I(H)$ gilt, ergibt sich weiter
\begin{align*}
  \val_I(G\alimpl H \aland H\alimpl G) 
  &= (\lnot\val_I(G)\lor\val_I(G)) \land (\lnot\val_I(G)\lor\val_I(G))
\end{align*}
%
und gleichgültig, welchen Wert $\val_I(G)$
hat ergibt die Auswertung letzten Endes \emph{immer}
$\val_I(\cdots)=\cdots = \W\land\W=\W$.
%
Also gilt:
%
\begin{lemma}
  Wenn $G$ und $H$
  äquivalente aussagenlogische Formeln sind, dann ist die Formel
  $G\aleqv H$ eine Tautologie.
\end{lemma}
%
Zusammen mit Einsichten in vorangegangenene Abschnitten zeigt das
sofort, dass für beliebige Formeln $G$
und $H$ \zB folgende Formeln Tautologien sind:
%
\begin{itemize}
\item $\alnot \alnot G \aleqv G$
\item $\alka G\alimpl H\alkz \aleqv \alka\alnot G\alor H\alkz$
\item $\alka G\alimpl H\alkz \aleqv \alka \alnot H \alimpl\alnot G\alkz$
\item $\alka G\aland H\alkz \aleqv \alnot\alka \alnot G \alor\alnot H\alkz$ und \\
   $\alka G\alor H\alkz \aleqv \alnot\alka \alnot G \aland\alnot H\alkz$
\item $\alnot\alka G\aland H\alkz \aleqv \alka \alnot G \alor\alnot H\alkz$ und \\
   $\alnot\alka G\alor H\alkz \aleqv \alka \alnot G \aland\alnot H\alkz$
\item $G\aland G \aleqv G$ und\\
  $G\alor G \aleqv G$
\item $G\aland H \aleqv H\aland G$ und \\
  $G\alor H \aleqv H\alor G$
\end{itemize}
%
Es gilt übrigens auch die Umkehrung der obigen Argumentation:
%
\begin{lemma}
  Wenn für zwei aussagenlogische Formeln $G$
  und $H$
  die Formel $G\aleqv H$
  eine Tautologie ist, dann sind $G$ und $H$ äquivalente Formeln.
\end{lemma}
% 
Außerdem überzeugt man sich schnell, dass auch Formeln des folgenden
Aufbaus allgemeingültig sind für beliebige aussagenlogische Formeln
$G$, $H$ und $K$:
\begin{itemize}
\item $G\alimpl G$
\item $\alnot G\alor G$
\item $G\alimpl \alka H\alimpl G\alkz$
\item $\alka G\alimpl \alka H\alimpl K\alkz\alkz
  \alimpl \alka \alka G \alimpl H\alkz\alimpl\alka G\alimpl K\alkz\alkz $
\item $\alka\alka\alnot H\alimpl \alnot G\alkz\alimpl \alka\alka\alnot
  H\alimpl G\alkz\alimpl H\alkz\alkz$
\end{itemize}


%-----------------------------------------------------------------------
\section{Beweisbarkeit}
\label{sec:beweisbarkeit}

Der grundlegende Begriff im Zusammenhang mit Beweisbarkeit sowohl in
der Aussagenlogik als auch in der Prädikatenlogik ist der des Kalküls.
%
Im Fall der Aussagenlogik gehören zu sogenannten
\mdefine{Aussagenkalkül} gehören dazu stets
\begin{itemize}
\item das Alphabet $\AAL$, aus dem die Zeichen für alle Formeln stammen,
\item die Menge $\LAL$,
  der syntaktisch korrekten Formeln über dem Alphabet $\AAL$,
\item eine Menge sogenannter \mdefine{Axiome} $\AxAL\subseteq\AAL$,
\item und sogenannte \mdefine{Schlussregeln} oder \mdefine{logische
    Folgerungsregeln}. 
\end{itemize}
% 
Als Axiome für das Aussagenkalkül wählen wir 
\begin{align*}
  \AxAL &= \bigl\{\alka G\alimpl \alka H\alimpl  G\alkz\alkz
          \bigm| G,H\in\LAL \bigr\} \\
        &\mathrel{\hphantom{=}} \cup \bigl\{\alka G\alimpl \alka H\alimpl  K\alkz\alkz
          \alimpl \alka\alka G\alimpl H\alkz\alimpl \alka G\alimpl  K\alkz\alkz \bigm| G,H,K\in\LAL \bigr\}\\
        &\mathrel{\hphantom{=}} \cup \bigl\{
          \alka\alnot H\alimpl \alnot G\alkz\alimpl \alka\alka\alnot H\alimpl G\alkz\alimpl  H\alkz
          \bigm| G,H \in\LAL 
          \bigr\}
\end{align*}
%
Die Formelmenge in den drei Zeilen wollen kurz mit $\AxAL{}_1$,
$\AxAL{}_2$ und $\AxAL{}_3$ bezeichnen.
%
Für alle drei Mengen haben wir schon erwähnt, dass sie nur Tautologien
enthalten.
%
(Überlegen Sie sich bitte auch selbst.)
%
Die Axiome sind also alle Tautologien.

Im Fall der Aussagenlogik gibt es nur eine Schlussregel, den
sogenannten \mdefine{Modus Ponens}.
% 
Diese Regel kann man formalisieren als Relation
$\MP\subseteq\LAL^3$, nämlich
\[
  \MP = \{ (G\alimpl H, G, H) \mid  G, H \in\LAL \}
\]
%
Bei Kalkülen schreibt man gelegentlich Ableitungsregeln in der
speziellen Form auf.
%
Beim Modus Ponens sieht das so aus:
  
$\MP:$ \quad \begin{tabular}{c}
                $G \alimpl H$ \qquad $G$ \\
                \midrule
                $H$
              \end{tabular}

\noindent
Wie der Modus Ponens anzuwenden ist, ergibt sich bei der Definition
dessen, was wir formal unter einer Ableitung \bzw unter einem Beweis
verstehen wollen.

Dazu sei $\Gamma$
eine Formelmenge sogenannter \mdefine[Hypothese]{Hypothesen} oder
\mdefine[Prämisse]{Prämissen} und $G$
eine Formel. 
%
Eine \mdefine{Ableitung} von $G$
aus $\Gamma$
ist eine endliche Folge $(G_1,\dots,G_n)$
von $n$
Formeln mit der Eigenschaft, dass erstens $G_n=G$
ist und auf jede Formel $G_i$ einer der folgenden Fälle zutrifft:

\begin{itemize}
\item Sie ist ein Axiom: $G_i\in\AxAL$.
\item Oder sie ist eine Prämisse: $G_i\in\Gamma$.
\item Oder es gibt Indizes $i_1$ und $i_2$ echt kleiner $i$, für die gilt:
  $(G{i_1},G_{i_2},G_i)\in\MP$.
\end{itemize}
%
Wir schreiben dann $\Gamma\vdash G$.
%
Ist $\Gamma=\{\}$,
so heißt eine entsprechende Ableitung auch ein \define{Beweis} von $G$
und $G$
ein \mdefine{Theorem} des Kalküls, in Zeichen: $\vdash G$.

Wir wollen als erstes einen Beweis für die Formel
$\alka\alP\alimpl\alP\alkz$
angeben. 
%
Dabei ist zum besseren Verständnis in jeder Zeile in
Kurzform eine Begründung dafür angegeben, weshalb die jeweilige
Formel an dieser Stelle im Beweis aufgeführt werden darf.

\begin{tabular}{rll}
1. & $\alka \alka \alP \alimpl \alka \alka \alP \alimpl  \alP \alkz\alimpl  \alP \alkz\alkz\alimpl 
       \alka \alka \alP \alimpl \alka \alP \alimpl  \alP \alkz\alkz\alimpl \alka \alP \alimpl  \alP \alkz\alkz\alkz$ & $\AxAL{}_2$ \\
2. & $\alka \alP \alimpl \alka \alka \alP \alimpl  \alP \alkz\alimpl  \alP \alkz\alkz$ & $\AxAL{}_1$ \\
3. & $\alka \alka \alP \alimpl \alka \alP \alimpl  \alP \alkz\alkz\alimpl \alka \alP \alimpl  \alP \alkz\alkz$ & $\MP(1,2)$ \\
4. & $\alka \alP \alimpl \alka \alP \alimpl  \alP \alkz\alkz$ & $\AxAL{}_1$ \\
5. & $\alka \alP \alimpl  \alP \alkz$ & $\MP(3,4)$ 
\end{tabular}

Wenn man in diesem Beweis überall statt $\alP$
eine andere aussagenlogische Formel $G$
notieren würde, erhielte man offensichtlich einen Beweis für
$G\alimpl G$.

% \begin{tabular}{rll}
% 1. & \L{\alka \alka \lmeta{A}\alimpl \alka \alka \lmeta{A}\alimpl  \lmeta{A})\alimpl  \lmeta{A}))\alimpl 
%        \alka \alka \lmeta{A}\alimpl \alka \lmeta{A}\alimpl  \lmeta{A}))\alimpl \alka \lmeta{A}\alimpl  \lmeta{A})))} & $\A_2$ \\
% 2. & \L{\alka \lmeta{A}\alimpl \alka \alka \lmeta{A}\alimpl  \lmeta{A})\alimpl  \lmeta{A}))} & $\A_1$ \\
% 3. & \L{\alka \alka \lmeta{A}\alimpl \alka \lmeta{A}\alimpl  \lmeta{A}))\alimpl \alka \lmeta{A}\alimpl  \lmeta{A}))} & $\RMP$,1,2 \\
% 4. & \L{\alka \lmeta{A}\alimpl \alka \lmeta{A}\alimpl  \lmeta{A}))} & $\A_1$ \\
% 5. & \L{\alka \lmeta{A}\alimpl  \lmeta{A})} & $\RMP$,3,4
% \end{tabular}

Man spricht in einem solchen Fall von einem \mdefine{Beweisschema}.
Wir werden uns aber erlauben, im folgenden auch in solchen Fällen
lax von Beweisen zu sprechen.
%
Auch der folgende Beweis ist nicht besonders lang:

\begin{tabular}{rll}
1. & $\alka \alnot\alP \alimpl \alnot\alP \alkz\alimpl 
     \alka \alka \alnot\alP \alimpl \alP \alkz\alimpl \alP \alkz$ & $\AxAL{}_3$ \\
2. & $\alka \alnot\alP \alimpl \alnot\alP \alkz$ & Beispiel von eben \\
3. & $\alka \alnot\alP \alimpl \alP \alkz\alimpl \alP$  & $\MP(1,2)$ \\
\end{tabular}

\noindent
Genaugenommen müsste man allerdings statt der Zeile~2 die fünf Zeilen
von eben übernehmen und überall $\alP$ durch $\alnot\alP$ ersetzen.
%
Aber so wie man Programme strukturiert, indem man größere Teile aus
kleineren zusammensetzt, macht man das bei Beweisen auch.
%
Insbesondere dann, wenn man kleinere Teil mehrfach verwenden kann.

Wir hatten schon darauf hingewiesen, dass alle Formeln in $\AxAL$
Tautologien sind.
%
Sehen wir uns nun noch den Modus Ponens genauer an. 
%
Es gilt
%
\begin{lemma}
  Modus Ponens "`erhält Allgemeingültigkeit"', \dh wenn sowohl
  $G\alimpl H$
  als auch $G$ Tautologien sind, dann ist auch $H$ eine Tautologie.
\end{lemma}

\begin{beweis}
  Wenn $G\alimpl H$
  eine Tautologie ist, dann gilt für jede Bewertung $I$:
  \[
    \W=\val_I(G\alimpl H)= b_{\alimpl}(\val_I(G), \val_I)(H) \;.
  \]
  %
  Da $G$ aber auch Tautologie ist, ist $\val_I(G)=\W$ und folglich
  \[
    b_{\alimpl}(\val_I(G), \val_I(H))= b_{\alimpl}(\W, \val_I(H)) = \val_I(H)
  \]
  Also ist $\W=\val_I(H)$
  für jede Bewertung $I$, also ist $H$ Tautologie.
\end{beweis}

\noindent
Da alle Axiome Tautologien sind und Modus Ponens Allgemeingültigkeit
erhält, ergibt sich
\begin{korollar}
  \label{kor:theoreme-sind-tautologien}
  Alle Theorem des Aussagenkalküls sind Tautologien.
\end{korollar}

\noindent
Schwieriger zu beweisen ist, dass die Umkehrung dieser Aussage auch
gilt:
\begin{lemma}
  \label{lem:tautolgien-sind-theoreme}
  Jede Tautologie ist im Aussagenkalkül beweisbar.
\end{lemma}

\noindent
Zusammen mit Korollar~\ref{kor:theoreme-sind-tautologien} ergibt sich
der folgende Satz.
%
\begin{theorem}
  Für jede Formel $G\in\LAL$
  gilt $\models G$ genau dann, wenn $\vdash G$ gilt.
\end{theorem}

\noindent
Wir werden Lemma~\ref{lem:tautolgien-sind-theoreme} in diesem Kapitel
nicht beweisen.
%
Möglicherweise werden wir aber für ein verwandtes System (mit anderen
Schlussregeln) sehen, dass man sogar Systeme bauen kann, für die man
algorithmisch (\dh also mit Rechner) für jede Formel 
\begin{itemize}
\item herausfinden kann, ob sie Tautologie ist oder nicht und
\item gegebenenfalls dann sogar auch gleich noch einen Beweis im
  Kalkül.
\end{itemize}
%
\emph{Man beachte}, dass das nur in der Aussagenlogik so schön klappt.
%
In der Prädikatenlogik, die wir in einem späteren Kapitel auch noch
ansehen werden, ist das nicht mehr der Fall.

%-----------------------------------------------------------------------
\section*{Zusammenfassung und Ausblick}

In diesem Kapitel wurde zunächst die Syntax aussagenlogischer Formeln
festgelegt.
%
Anschließend haben wir ihre Semantik mit Hilfe boolescher Funktionen
definiert.
%
Und am Ende haben wir gesehen, wie man dem semantischen Begriff der
Allgemeingültigkeit den syntaktischen Begriff der Beweisbarkeit so
gegenüberstellen kann, dass sich die beiden entsprechen.

\printunitbibliography

\cleardoublepage



% Alles was wir bisher in diesem Abschnitt betrachtet haben, gehört zu
% dem Bereich der \emph{Aussagenlogik}. Wir werden sie im beschriebenen
% Sinne naiv verwenden und in Zukunft zum Beispiel Definitionen wie die
% der Injektivität und Surjektivität von Funktionen entsprechend
% kompakter schreiben können.

% Außerdem gibt es die sogenannte \emph{Prädikatenlogik}. (Genauer
% gesagt interessiert uns Prädikatenlogik erster Stufe, ohne dass wir
% die Bedeutung dieser Bezeichnung jetzt genauer erklären wollen oder
% können.)

% Aus der Prädikatenlogik werden wir --- wieder ganz naiv --- die
% sogenannten \mdefine{Quantoren}\index{Quantor} verwenden:
% %
% \begin{center}
%   \begin{tabular}{cc@{\hspace*{30mm}}cc}
%     \define{Allquantor} & $\forall$ & \define{Existenzquantor} & $\exists$ \\
%   \end{tabular}
%   \index{Allquantor}\index{Existenzquantor}
% \end{center}
% %
% In der puren Form hat eine quantifizierte Aussage eine der Formen
% \[
%   \forall x \; A(x) \text{\qquad oder\qquad} \exists x \; A(x) 
% \]
% %
% Dabei soll $A(x)$ eine Aussage sein, die von einer Variablen $x$
% abhängt (oder jedenfalls abhängen kann). $A$ kann weitere Quantoren
% enthalten. 

% Die Formel $\forall x \; A(x)$ hat man zu lesen als: "`Für alle $x$
% gilt: $A(x)$"'.  Und die Formel $\exists x \; A(x)$ hat man zu lesen
% als: "`Es gibt ein $x$ mit: $A(x)$"'.

% Zum Beispiel ist die Formel
% \[
% \forall x \; (x\in \N_0 \alimpl \exists y \; (y\in\N_0 \aland y=x+1))
% \]
% wahr.

% Sehr häufig hat man wie in diesem Beispiel den Fall, dass eine Aussage
% nicht für alle $x$ gilt, sondern nur für $x$ aus einer gewissen
% Teilmenge $M$. Statt
% \[
% \forall x \; (x\in M \alimpl B(x))
% \]
% schreibt man oft kürzer
% \[
% \forall x \in M :\; B(x)
% \]
% wobei der Doppelpunkt nur die Lesbarkeit verbessern soll.  Obiges
% Beispiel wird damit zu
% \[
% \forall x \in \N_0 : \exists y \in\N_0 : y=x+1
% \]


%-----------------------------------------------------------------------
%%%
%%% Local Variables:
%%% fill-column: 70
%%% mode: latex
%%% TeX-master: "../k-05-aussagenlogik/skript.tex"
%%% TeX-command-default: "XPDFLaTeX"
%%% End:
