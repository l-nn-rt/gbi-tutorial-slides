%=======================================================================
\chapter{Dokumente}
\label{k:dokumente}
\strictpagecheck
%-----------------------------------------------------------------------
\section{Dokumente}

Die Idee zu dieser Einheit geht auf eine Vorlesung "`Informatik A"' von Till
Tantau\index{Tantau, Till} aus dem Jahre 2012 zurück.
% URL tot:
% (siehe \url{http://www.tcs.uni-luebeck.de/lehre/2012-ws/info-a/wiki/Vorlesung}, 18.10.2013).

Im alltäglichen Leben hat man mit vielerlei Inschriften zu tun: Briefe,
Kochrezepte, Zeitungsartikel, Vorlesungsskripte, Seiten im WWW, Emails, und so
weiter.  Bei vielem, was wir gerade aufgezählt haben, und bei vielem anderen
kann man drei verschiedene Aspekte unterscheiden, die für den Leser eine Rolle
spielen, nämlich
%
\begin{itemize}
\item den \mdefine{Inhalt}\index{Inhalt} des Textes,
\item seine \mdefine{Struktur}\index{Struktur} und
\item sein \mdefine[Erscheinungsbild,\\
  Form]{Erscheinungsbild},\index{Erscheinungsbild} die (äußere)
  \define{Form}.\index{Form}
\end{itemize}
%
Stünde die obige Aufzählung so auf dem Papier:
%
\begin{itemize}
\item den \uppercase{Inhalt} des Textes,
\item seine \uppercase{Struktur} und
\item sein \uppercase{Erscheinungsbild}, die (äußere) \uppercase{Form}.
\end{itemize}
%
dann hätte man an der Form etwas geändert (Wörter in Großbuchstaben statt
kursiv), aber nicht am Inhalt oder an der Struktur.  Hätten wir dagegen
geschrieben:
\begin{quote}
  "`[\dots] den \define{Inhalt} des Textes, seine \define{Struktur} und sein
  \define{Erscheinungsbild}, die (äußere) \define{Form}."'
\end{quote}
dann wäre die Struktur eine andere (keine Liste mehr, sondern Fließtext), aber
der Inhalt immer noch der gleiche.  Und stünde hier etwas über
\begin{itemize}
\item \emph{Balaenoptera musculus} (Blauwal),
\item \emph{Mesoplodon carlhubbsi} (Hubbs-Schnabelwal) und
\item \emph{Physeter macrocephalus} (Pottwal),
\end{itemize}
dann wäre offensichtlich der Inhalt ein anderer.

Von Ausnahmen abgesehen (welche fallen Ihnen ein?), ist Autoren und Lesern
üblicherweise vor allem am Inhalt gelegen. Die Wahl einer bestimmten Struktur
und Form hat primär zum Ziel, den Leser beim Verstehen des Inhalts zu
unterstützen. Mit dem Begriff \mdefine{Dokumente}\index{Dokument} in der
Überschrift sollen Texte gemeint sein, bei denen man diese drei Aspekte
unterscheiden kann.

Wenn Sie später beginnen, erste nicht mehr ganz kleine Dokumente (\zB
Seminarausarbeitungen oder die Bachelor-Arbeit) selbst zu schreiben, werden
Sie merken, dass die Struktur, oder genauer gesagt das Auf"|finden einer
geeigneten Struktur, auch zu Ihrem, also des Autors, Verständnis beitragen
kann. Deswegen ist es bei solchen Arbeiten unseres Erachtens sehr\graffito{ein
  Rat für Ihr weiteres Studium} ratsam%
\index{früh aufschreiben}\index{aufschreiben!früh}\index{Rat fürs
  Studium}\index{Studium!ein guter
  Rat}\index{Bachelorarbeit}\index{Seminararbeit}\index{Masterarbeit},
\emph{früh damit zu beginnen, etwas aufzuschreiben,} weil man so gezwungen
wird, über die Struktur nachzudenken.

Programme fallen übrigens auch in die Kategorie dessen, was wir hier
Dokumenten nennen.

%-----------------------------------------------------------------------
\section{Struktur von Dokumenten}

Oft ist es so, dass es Einschränkungen bei der erlaubten Struktur von
Dokumenten gibt. Als Beispiel wollen wir uns zwei sogenannte
\mdefine[Auszeichnungssprache]{Auszeichnungssprachen}%
\index{Auszeichnungssprache} (im Englischen \mdefine{markup
language}\index{markup language}) ansehen. Genauer gesagt werden wir uns dafür
interessieren, wie Listen in \LaTeX{} aufgeschrieben werden müssen, und wie
Tabellen in XHTML.

%-----------------------------------------------------------------------
\subsection{\LaTeX}

\LaTeX{}\index{LaTeX@{\LaTeX}}, ausgesprochen "`Latech"', ist (etwas ungenau,
aber für unsere Belange ausreichend formuliert) eine Erweiterung eines
Textsatz"=Programmes (ini\TeX), das von Donald Knuth\index{Knuth, Donald}
entwickelt wurde (\url{http://www-cs-faculty.stanford.edu/~knuth/}, 27.11.2015).

Es wird zum Beispiel in der Informatik sehr häufig für die Verfassung von
wissenschaftlichen Arbeiten verwendet, weil unter anderem der Textsatz
mathematischer Formeln durch \TeX{} von hervorragender Qualität ist, ohne dass
man dafür viel tun muss. Sie ist deutlich besser als alles, was der Autor
dieser Zeilen jemals in Dokumenten gesehen hat, die mit Programmen wie \zB
libreoffice o.\,ä.\ verfasst wurden. Zum Beispiel liefert der Text
\begin{verbatim}
  \[ 2 - \sum_{i=0}^{k} i 2^{-i} = (k+2) 2^{-k} \]
\end{verbatim}
die Ausgabe
  \[ 2 - \sum_{i=0}^{k} i 2^{-i} = (k+2) 2^{-k} \]
%
Auch der vorliegende Text wurde mit \LaTeX{} gemacht. Für den Anfang
dieses Abschnittes wurde \zB geschrieben:

\verb|\section{Struktur von Dokumenten}|\hfill

\noindent
woraus \LaTeX{} die Zeile

{\large 7.2\ \ \ \ \textsc{Struktur von Dokumenten}}

\noindent
am Anfang dieser Seite gemacht hat.  Vor dem Text der Überschrift wurde also
automatisch die passende Nummer eingefügt und der Text wurde in einer
Kapitälchenschrift gesetzt. Man beachte, dass \zB die Auswahl der Schrift
\emph{nicht} in der Eingabe mit vermerkt ist. Diese Angabe findet sich an
anderer Stelle, und zwar an \emph{einer} Stelle, an der das typografische
Aussehen \emph{aller} Abschnittsüberschriften (einheitlich) festgelegt ist.

Ganz grob kann ein \LaTeX-Dokument \zB die folgende Struktur haben:
{\small
\begin{verbatim}
  \documentclass[11pt]{report}
  % Kommentare fangen mit einem %-Zeichen an und gehen bis zum Zeilenende
  % dieser Teil heißt Präambel des Dokumentes
  % die folgende Zeile ist wichtig, kann hier aber nicht erklärt werden
  \usepackage[T1]{fontenc}      
  % Unterstützung für Deutsch, z.B. richtige automatische Trennung
  \usepackage[ngerman]{babel}      
  % Angabe des Zeichensatzes, der für den Text verwendet wird
  \usepackage[utf8]{inputenc}     % zu UTF-8 siehe Kapitel 10
  % für das Einbinden von Grafiken
  \usepackage{graphicx}
  \begin{document}
    % und hier kommt der eigentliche Text .....
  \end{document}
\end{verbatim}
}
%
\noindent
Eine Liste einfacher Punkte sieht in \LaTeX{} so aus:
%
\begin{center}
  \begin{tabular}{l@{\hspace*{20mm}}l}
    \toprule
    Eingabe & Ausgabe \\
    \midrule
    \verb|\begin{itemize}|& \\
      \verb|\item Inhalt|   & \textbullet\ \ Inhalt \\
      \verb|\item Struktur| &\textbullet\ \ Struktur \\
      \verb|\item Form|     &\textbullet\ \ Form \\
      \verb|\end{itemize}|  & \\
    \bottomrule
  \end{tabular}
\end{center}
%
Vor den aufgezählten Wörtern steht jeweils ein dicker Punkt. Auch dieser
Aspekt der äußeren Form ist \emph{nicht} dort festgelegt, wo die Liste steht,
sondern an \emph{einer} Stelle, an der das typografische Aussehen \emph{aller}
solcher Listen (einheitlich) festgelegt ist. Wenn man in seinen Listen lieber
alle Aufzählungspunkte mit einem sogenannten Spiegelstrich "`--"' beginnen
lassen möchte, dann muss man nur an einer Stelle (in der Präambel) die
Definition von \verb|\item| ändern.

Wollte man die formale Sprache $L_{\mathrm{itemize}}$ aller legalen
Texte für Listen in LaTeX{} aufschreiben, dann könnte man \zB zunächst
geeignet die formale Sprache $L_{\mathrm{item}}$ aller Texte
spezifizieren, die hinter einem Aufzählungspunkt vorkommen dürfen.
Dann wäre

\noindent
$L_{\mathrm{itemize}} = \bigl\{$\textcolor{darkblue}{\texttt{\char92
    begin\char123 itemize\char125}}$\bigr\}$
$\Bigl(\bigl\{$\textcolor{darkblue}{\texttt{\char92 item}}$\bigr\}
L_{\mathrm{item}}\Bigr)^+ \bigl\{$\textcolor{darkblue}{\texttt{\char92
    end\char123 itemize\char125}}$\bigr\}$ 

\noindent
Dabei haben wir jetzt vereinfachend so getan, als wäre es kein Problem,
$L_{\mathrm{item}}$ zu definieren. Tatsächlich ist das aber zumindest auf den
ersten Blick eines, denn ein Aufzählungspunkt in \LaTeX{} darf seinerseits
wieder eine Liste enthalten. Bei naivem Vorgehen würde man also genau
umgekehrt für die Definition von $L_{\mathrm{item}}$ auch auf
$L_{\mathrm{itemize}}$ zurückgreifen (wollen). Wir diskutieren das ein kleines
bisschen ausführlicher in Unterabschnitt~\ref{ssub:rekursive-def}.

% -----------------------------------------------------------------------
\subsection{HTML und XHTML}

HTML ist die Auszeichnungssprache, die man benutzt, wenn man eine WWW-Seite
(be)schreibt. Für HTML ist formaler als für \LaTeX{} festgelegt, wie
syntaktisch korrekte solche Seiten aussehen. Das geschieht in einer
sogenannten \emph{document type definition}, kurz \emph{DTD}.

Hier ist ein Auszug aus der DTD für eine (nicht die allerneueste) Version von
XHTML. Sie dürfen sich vereinfachend vorstellen, dass das ist im wesentlichen
eine noch striktere Variante von HTML ist. Das nachfolgenden Fragment
beschreibt teilweise, wie man syntaktisch korrekt eine Tabelle notiert.

{\small
\begin{verbatim}
  <!ELEMENT table (caption?, thead?, tfoot?, (tbody+|tr+))>
  <!ELEMENT caption  %Inline;>
  <!ELEMENT thead    (tr)+>
  <!ELEMENT tfoot    (tr)+>
  <!ELEMENT tbody    (tr)+>
  <!ELEMENT tr       (th|td)+>
  <!ELEMENT th       %Flow;>
  <!ELEMENT td       %Flow;>
\end{verbatim}
} 

\noindent
Wir können hier natürlich nicht auf Details eingehen. Einige einfache
aber wichtige Aspekte sind aber mit unserem Wissen schon
verstehbar. Die Wörter wie \literal{table}, \literal{thead},
\literal{tr}, \usw dürfen wir als bequem notierte Namen für formale
Sprachen auf"|fassen. Welche, das wollen wir nun klären.

Die Bedeutung von \literal{*} und \literal{+} ist genau das, was wir als
Konkatenationsabschluss und $\eps$-freien Konkatenationsabschluss kennen
gelernt haben. Die Bedeutung des Kommas \literal{,} in der ersten Zeile ist
die des Produktes formaler Sprachen. Die Bedeutung des senkrechten Striches
\literal{|} in der sechsten Zeile ist Vereinigung von Mengen.

Das Fragezeichen ist uns neu, hat aber eine ganz banale Bedeutung: In der uns
geläufigen Notation würden wir definieren: $L^? = L^0\cup L^1 = \{\eps\} \cup
L$. Mit anderen Worten: Wenn irgendwo $L^?$ notiert ist, dann kann an dieser
Stelle ein Wort aus $L$ stehen, oder es kann fehlen. Das Auftreten eines
Wortes aus $L$ ist also mit anderen Worten optional.

Nun können Sie zur Kenntnis nehmen, dass \zB die Schreibweise

\qquad \qquad \literal{<!ELEMENT tbody (tr)+ >} 

\noindent
die folgende formale Sprache festlegt:
\[
L_{\mathtt{tbody}} = \{\literal{<tbody>}\}\cdot
L_{\mathtt{tr}}^+\cdot\{\literal{</tbody>}\}
\]
Das heißt, ein Tabellenrumpf (\emph{table body}) beginnt mit der
Zeichenfolge \literal{<tbody>}, endet mit der Zeichenfolge
\literal{</tbody>}, und enthält dazwischen eine beliebige positive
Anzahl von Tabellenzeilen (\emph{table rows}). Und die erste Zeile aus
der DTD besagt
\[
L_{\mathtt{table}} = \{\literal{<table>}\}\cdot
L_{\mathtt{caption}}^?\cdot L_{\mathtt{thead}}^?\cdot
L_{\mathtt{tfo{}ot}}^?\cdot (L_{\mathtt{tbody}}^+ \cup L_{\mathtt{tr}}^+)\cdot
\{\literal{</table>}\}
\]
das heißt, eine Tabelle (\emph{table}) ist von den Zeichenketten
\literal{<table>} und \literal{</table>} umschlossen und enthält
innerhalb in dieser Reihenfolge
%
\begin{itemize}
\item optional eine Überschrift (\emph{caption}),
\item optional einen Tabellenkopf (\emph{table head}),
\item optional einen Tabellenfuß (\emph{table foot}) und
\item eine beliebige positive Anzahl von Tabellenrümpfen (siehe eben) oder
  Tabellenzeilen.
\end{itemize}
%
Insgesamt ergibt sich, dass zum Beispiel
\begin{verbatim}
  <table>
    <tbody>
      <tr>  <td>1</td> <td>a</td>  </tr>
      <tr>  <td>2</td> <td>b</td>  </tr>
    </tbody>
  </table>
\end{verbatim}
eine syntaktisch korrekte Tabelle.

In Wirklichkeit gibt es noch zusätzliche Aspekte, die das Ganze formal
verkomplizieren und bei der Benutzung flexibler machen, aber das ganz
Wesentliche haben wir damit jedenfalls an einem Beispiel beleuchtet.

%%-----------------------------------------------------------------------
%\section{Erstellung von Dokumenten}
%
%Wir haben schon darauf hingewiesen, dass 


% -----------------------------------------------------------------------
\subsection{Eine Grenze unserer bisherigen Vorgehensweise}
\label{ssub:rekursive-def}

Dass wir eben mit Hilfe von Produkt und Konkatenationsabschluss formaler
Sprachen in einigen Fällen präzise Aussagen machen konnten, hing auch mit der
Einfachheit dessen zusammen, was es zu spezifizieren galt.
%
Es wurde, jedenfalls in einem intuitiven Sinne, immer etwas von einer
komplizierteren Art aus Bestandteilen einfacherer Art zusammengesetzt.

Es gibt aber auch den Fall, dass man sozusagen größere Dinge einer Art aus
kleineren Bestandteilen zusammensetzen will, die aber von der gleichen Art
sind. Auf Listen, deren Aufzählungspunkte ihrerseits wieder Listen enthalten
dürfen, hatten wir im Zusammenhang mit \LaTeX{} schon hingewiesen.  

Ein anderes typisches Beispiel sind korrekt geklammerte arithmetische
Ausdrücke. Sehen wir einmal von den Operanden und Operatoren ab und
konzentrieren uns auf die reinen Klammerungen.  Bei einer syntaktisch
korrekten Klammerung gibt es zu jeder Klammer auf "`weiter hinten"' die
"`zugehörige"' Klammer zu. Insbesondere gilt:
%
\begin{itemize}
\item Man kann beliebig viele korrekte Klammerungen konkatenieren und
  erhält wieder eine korrekte Klammerung.
\item Man kann um eine korrekte Klammerung außen herum noch ein
  Klammerpaar schreiben (Klammer auf ganz vorne, Klammer zu ganz
  hinten) und erhält wieder eine korrekte Klammerung.
\end{itemize}
%
Man würde also gerne in irgendeinem Sinne $L_{\mathrm{Klammer}}$ mit
$L_{\mathrm{Klammer}}^*$ und mit
$\{\literal{(}\}\cdot L_{\mathrm{Klammer}}\cdot\{\literal{)}\}$ in Beziehung
setzen. Es ist aber nicht klar wie. Insbesondere würden bei dem
Versuch, eine Art Gleichung hinzuschreiben, sofort die Fragen im Raum
stehen, ob die Gleichung überhaupt lösbar ist, und wenn ja, ob die
Lösung eindeutig ist.
 
%-----------------------------------------------------------------------
\section{Zusammenfassung}

In dieser Einheit haben wir über \emph{Dokumente} gesprochen. Sie
haben einen \emph{Inhalt}, eine \emph{Struktur} und ein
\emph{Erscheinungsbild}.

Formale Sprachen kann man \zB benutzen, um zu spezifizieren, welche
Struktur(en) ein legales, \dh syntaktisch korrektes Dokument haben
darf, sofern die Strukturen hinreichend einfach sind. Was man in
komplizierten Fällen machen kann, werden wir in einer späteren Einheit
kennenlernen.
 
%-----------------------------------------------------------------------
%\section{Ausblick}

%In der Einheit~\ref{k:codierungen} "`Codierungen"' werden wir genauer
%auf Codierungen eingehen.

%Mehr über Schriften findet man zum Beispiel über die WWW-Seite
%\url{http://www.omniglot.com/writing/} (1.10.08).

\cleardoublepage

%-----------------------------------------------------------------------
%%%
%%% Local Variables:
%%% fill-column: 78
%%% mode: latex
%%% TeX-master: "../k-11-dokumente/skript.tex"
%%% TeX-command-default: "XPDFLaTeX"
%%% End:

