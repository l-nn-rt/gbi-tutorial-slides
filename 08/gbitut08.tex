%beamer

% Comment/uncomment this line to toggle handout mode
\newcommand{\handout}{}

\input{../framework/PraeambelTut.tex}

\morescalingdelimiters

\begin{document}
\starttut{8}

\framePrevEpisode

\begin{frame}{Rückblick: Kontextfreie Grammatiken}
	\begin{itemize}[<+->]
		\item Ein Vier-Tupel: $G = (N, T, S, P)$
		\item Produktionen definieren Ersetzungen eines Nichtterminals mit Wörtern über $N \cup T$
		\item Wir wenden Produktionen in Ableitungsschritten an: $v \Rightarrow w, v \Rightarrow^* w$
		\item $L(G)$ sind alle aus $S$ ableitbaren Wörter über $T$ (die also nur aus Terminalsymbolen bestehen)
	\end{itemize}
\end{frame}

\begin{frame}[t]{Wahr oder Falsch?}
	\Socrative
	Sei $G=(\{X,Y\},\, \{\word a, \word b\},\, X,\, P)$ eine kontextfreie Grammatik. \\
	\FalseQuestion{Die Produktion $XY \to \word a$ könnte eine gültige Produktion sein.}
	\FalseQuestion{Die Produktion $\word a \to XY$ könnte eine gültige Produktion sein.}
	\TrueQuestionE{Die Produktion $X \to X\word aX$ könnte eine gültige Produktion sein.}{}
	\TrueQuestionE{Wenn $X \to w$ eine gültige Produktion ist, dann gilt $X \derives^* w$.}{}
	\FalseQuestionE{Wenn $X \derives^* w$ gilt, dann ist $X \to w$ eine Produktion in P.}{Sei $P=\{X \to XX \mid \word a\}$. Dann gilt $X \derives^* X\word a$, aber $X \to X\word a \notin P$.}
\end{frame}

\input{../Bloecke/Praedikatenlogik.tex}

\input{../Bloecke/Praedikatenlogik2}

\input{../Bloecke/DrMetaNordpol.tex}

\appendix
\beginbackup

\section{Zusammenfassung und Ausblick}

\begin{frame}
	\begin{block}{Was ihr nun wissen solltet}
		\begin{itemize}
			\item Wie Prädikatenlogische Formeln aufgebaut sind
			\item Wie man damit präzise Aussagen trifft
			\item Wie man sie auswertet
		\end{itemize}
	\end{block}
	
	\begin{block}{Was nächstes Mal kommt}
		\begin{itemize}
			\item Alles korrekt? –- Beweise mit dem Hoare-Kalkül
		\end{itemize}
	\end{block}
\end{frame}

\begin{frame}[plain]
	\begin{center}
		\large
		Nicht vergessen, Kinder: Nächste Woche findet noch ein Tutorium statt! \smiley
	\end{center}
\mycomment{	\bigskip
	Für alle die nicht kommen: \\
	Frohe Weihnachten und einen guten Start in das neue Jahr!}
\end{frame}

\only<handout:0>{\slideThanks}



 %% Letzte Seite
% \xkcdframe{704}{Danke für eure Aufmerksamkeit! \smiley}{2.5}
\xkcdframevert{835}{Frohe Weihnachten und bis nächstes Jahr! \smiley}{2.2}
%\lastframe{0.50}{0}{xkcd/logic_principle_of_explosion.png}{https://www.xkcd.com/}
%\lastframe{0.50}{0}{xkcd/christmastree.png}{https://www.xkcd.com/835} % Dieses Jahr dürften noch genügend Tutanden beim nächsten Termin kommen

 \only<beamer:0>{\slideThanks}

 \backupend

\end{document}