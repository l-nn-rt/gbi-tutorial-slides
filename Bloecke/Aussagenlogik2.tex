\section{Beweisbarkeit}

\begin{frame}{Modelle}
	\begin{Definition}
		Sei $G$ eine aussagenlogische Formel. \\
		Eine Interpretation $I$ heißt \textbf{Modell} von $G$, wenn gilt: \quad $val_I(G) = \W$. \\
		\pause
		\medskip
		Sei $\Gamma$\thasse{\footnote{Gamma}} eine Formelmenge.
		Eine Interpretation $I$ heißt \textbf{Modell} von $\Gamma$, wenn für alle Formeln $G \in  \Gamma$ gilt: \quad $val_I(G) = \W$.
	\end{Definition}
	\pause
	\begin{block}{Schreibweisen}
		$\Gamma \models G$: jedes Modell von $\Gamma$ auch Modell von $G$ \\
		Beispiel: \quad $\set{A,B,C} \models G$ \quad Jedes Modell von $A, B$ \textbf{und} $C$ ist auch Modell von $G$ \\
		\medskip
		Schreibe $H \models G$ statt $\set{H} \models G$ \\
		\medskip
		Schreibe $\models G$ statt $\set{} \models G$ \\
		\impl G ist \emph{Tautologie} oder \emph{allgemeingültig} 
	\end{block}
\end{frame}

\begin{frame}{Modelle}
	\begin{Definition}
		Eine Formel $G$ heißt \textbf{erfüllbar}, wenn für mindestens ein $I$ wahr. \\
		
	\end{Definition}
% 	\pause
% 	\begin{Beispiel}
% 		Alle Modelle von \mword{(C \boder \bnot C) \bimp \left(\bnot(B \bimp A)\right)}? \\
% 		\pause
% 		\impl $I_1, I_2 \from \set{\word A, \word B, \word C} \functionto \BB, \; 
% 		I_1(v) = 
% 		\caseslr{\F, & v = \word A \\
% 				\W, & v = \word B \\
% 				\W, & v = \word C}, \;
% 		I_2(v) = 
% 		\caseslr{\F, & v = \word A \\
% 				\W, & v = \word B \\
% 				\F, & v = \word C}$. \\
% 		\impl Formel \emph{erfüllbar}. \\
% 		\pause
% 		\medskip
% 		Alle Modelle von \mword{\bnot(C \bimp C)}? \\
% 		\pause
% 		\impl Gibt keine, Formel \emph{unerfüllbar}.
% 	\end{Beispiel}
\end{frame}

\begin{frame}{Tautologie, Äquivalenz}
	\begin{Definition}
		Eine Formel $G$ heißt \textbf{Tautologie}, wenn für alle möglichen $I$ wahr. \\
		\pause
		\medskip
		Kurzschreibweise: Für Formeln $G$ und $H$ ist \\
		$$\bleftBr G \bgdw H \brightBr :\equiv \bleftBr\bleftBr G \bimp H \brightBr \bund \bleftBr H \bimp G\brightBr\brightBr$$ \\
		\medskip
		\alert{\textbf{Bitte aufpassen mit Pfeilen: \quad $\bgdw$ vs. $\gdw$ \quad $\bimp$ vs. $\impl$}}
	\end{Definition}
	\pause
	\begin{block}{Lemma}
		Für Formeln $G$ und $H$ gilt \\
		\[ G \equiv H \quad \text{ genau dann, wenn } \quad G \bgdw H \text{ Tautologie ist.} \]
	\end{block}
\end{frame}

\begin{frame}{Tautologie, Äquivalenz}
	\begin{Beispiel}
	    \text{ }
		\centered{$\bnot(\bnot G) \bgdw G$ ist Tautologie, also $\bnot(\bnot G) \equiv G$}
	\end{Beispiel}
\end{frame}

% \begin{frame}{Der Aussagenkalkül}
% 	\begin{block}{Was ist ein Kalkül?}
% 		\begin{itemize}
% 			\item Ein \emph{Rechensystem}: \; \textbf{Dinge} und was man mit ihnen \textbf{anstellen} darf. \\
% 			Bsp.: \quad $\R$ und $+, -, \*, /$ \qquad Schachbrett, Figuren und Zugregeln
% 		\end{itemize}
% 	\end{block}
% 	\pause
% 	\begin{block}{Aussagenkalkül}
% 		Haben
% 		\begin{itemize}
% 			\item Syntaktisch korrekte Formeln $For_{AL}$ \\
% 			\impl können erfüllbar sein oder nicht 
% 			\item Davon nennen wir einige bestimmte \textbf{Axiome} \\
% 			\impl setzen wir als Tautologien voraus
% 			\item Eine \emph{Schlussregel}: \textbf{Modus Ponens} \\
% 			...um neue wahre Aussagen zu konstruieren
% 		\end{itemize}
% 	\end{block}
% \end{frame}

% \begin{frame}{Axiome, Modus Ponens}
% 	\begin{block}{Axiome}
% 		\includegraphics[width=.90\linewidth]{../figures/Axiome} \\
% 		Wir bestimmen: Das sind unsere „Basis-Tautologien“.
% 	\end{block}
% \end{frame}
\begin{frame}{Axiome, Modus Ponens}
	\begin{block}{Modus Ponens (MP)}
		\begin{columns}[T] 
			\begin{column}[T]{.45\textwidth} 
				\begin{itemize}
					\item<2-> Wenn $G$ gilt
					\item<2-> und $G \bimp H$ gilt \\ \mbox{}
					\implitem<2-> dann gilt auch $H$.
					\item<4-> Schreibweise: \deduction{G \qquad G \bimp H \concludes H} 
				\end{itemize}
			\end{column}
			\hspace{-2\baselineskip}
			\begin{column}[T]{.55\textwidth} 
				\begin{itemize}
					\item<3-> Wir wissen: „Es regnet.“
					\item<3-> Wir erinnern uns: \\ „Wenn es regnet, ist die Straße nass.“
					\implitem<3-> Also wissen wir: „Die Straße ist nass“.
				\end{itemize}
				\hspace{.6\baselineskip} \only<5->{\fbox{\parbox{.9\linewidth}{Mit MP können wir aus bekannten \\ Wahrheiten neue konstruieren!}}}
			\end{column}
		\end{columns}
		
	\end{block}
\end{frame}

% \begin{frame}{Ableitungen}
% 	Haben Formelsammlung $\Gamma$ („Hypothesen“/„Prämissen“), \\
% 	wollen eine Formel $G$ daraus ableiten
% 	\pause
% 	\begin{block}{Ableitung von $G$ aus $\Gamma$}
% 		Eine „Abfolge“ von Formeln, die in $G$ mündet \quad (Schreibweise: \; $\Gamma \vdash G$) \\
% 		Was dürfen wir machen?
% 		\pause
% 		\begin{itemize}
% 			\item<+-> aus syntaktisch korrekten Formeln \emph{Axiome} bilden und hinschreiben
% 			\item<.-> \emph{Prämissen} aus $\Gamma$ hinschreiben
% 			\item<.-> aus zwei vorherigen Formeln mit \emph{Modus Ponens} eine neue konstruieren
% 			\implitem<+-> das machen wir solange, bis wir $G$ konstruiert haben
% 		\end{itemize}
% 	\end{block}
% \end{frame}

% \begin{frame}{Ableitungen}
% 	\begin{exampleblock}{Beispiel}   % geklaut aus 2017 ÜB2 A4
% 		Gegeben sei die Prämissenmenge $\Gamma = \set{\alA \bimp \alB, \, \alA \bimp \bnot \alB}$ und die Formel $F = \bnot \alA$. \\ 
% 		Ableitung $\Gamma \vdash F$: \\
% 		\begin{tabular}{@{}l@{\;\;}l@{\;\;}l}
% 			1. & $\alA \bimp \alB$ & Prämisse 1; ist $\in \Gamma$ \\
% 			2. & $\alA \bimp \bnot \alB$ & Prämisse 2; ist $\in \Gamma$ \\
% 			3. & $\alka \alA \bimp \alB \alkz \bimp \alka \alka \alA \bimp \bnot \alB \alkz \bimp \bnot \alA \alkz$ & Anwendung von $Ax_{AL3}$ {\small (mit $H = \bnot \alA$ und $G = \bnot \alB$)} \\ 
% 			& & ({\small  Schema $Ax_{AL3}$:  $\alka \bnot H \bimp \bnot G \alkz \bimp \alka \alka \bnot H \bimp G \alkz \bimp H \alkz $ }) \\
% 			4. & $\alka \alA \bimp \bnot \alB \alkz \bimp \bnot \alA $ & MP (3, 1) \quad {\small (Modus Ponens mit Formeln 3 und 1)} \\
% 			5. & $\bnot \alA$ & MP (4, 2) \quad {\small (Modus Ponens mit Formeln 4 und 2)} \\
% 		\end{tabular}
% 	\end{exampleblock}
% \end{frame}

% \begin{frame}{Beweisbarkeit}
% 	\begin{block}{Beweis von $G$}
% 		\impl Ableitung von $G$ aus $\Gamma  = \emptyset$ \\
% 		\impl Wir verwenden nur Axiome und MP, es gibt \textbf{keine} Prämissen! \\
% 		Schreibweise: \quad $\vdash G$ \qquad „$G$ ist beweisbar“ \\
% 		Ein solches beweisbares $G$ nennen wir \textbf{Theorem} des Kalküls.
% 	\end{block}
% 	\pause 
% 	\begin{block}{Lemma}
% 		Eine AL-Formel $G$ ist genau dann Tautologie, wenn $G$ ein Theorem des AL-Kalküls ($=$~im Kalkül beweisbar) ist. \\
% 		\smallskip
% 		\centered{--- bzw. ---} 
% 		\smallskip
% 		Für jede AL-Formel $G$ gilt: \qquad $\models G \; \Gdw \; \vdash G$.\\
% 	\end{block}
% 	(Achtung: Es gibt Kalküle / Logiken, für die so etwas nicht gilt!)
% \end{frame}
% \begin{frame}{Beweisbarkeit}
% 	\begin{block}{Lemma}
% 		Für Formeln $G$, $H$ gilt $G \vdash H$ genau dann, wenn $\vdash \bleftBr G \bimp H \brightBr$.
% 	\end{block}
% \end{frame}
