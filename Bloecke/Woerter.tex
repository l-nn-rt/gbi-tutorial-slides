\section{Wörter}

\begin{frame}{Wörter}
	\begin{block}{Definition}
		\begin{itemize}
			\item Ein \textbf{Alphabet} $A$ ist eine endliche, nichtleere Menge von Zeichen. \pause
			\item Ein \textbf{Wort} $w$  über einem Alphabet $A$ ist ein \textbf{endliche Folge von Zeichen} aus A \\ 
		\end{itemize}
	\end{block}	
	
	\pause
	\begin{block}{Definition}
		$A^*$ ist die \textbf{Menge aller Wörter} beliebiger Länge, die nur Zeichen aus $A$ enthalten.
		% brauchen wir das?
		\mycomment{, also:\\
		\pause 
		$A^*$ ist die Menge aller Abbildungen $w \from \Z_n \functionto B$ mit $n \in \N_0$ und $B \subseteq A$. \\}
	\end{block}

	\pause
	\begin{exampleblock}{Beispiel}
		Sei $ A = \{ \word a, \word b \} $ ein Alphabet. 
		Dann sind $ w_1 = \word{aabbabab}$ und $w_2 = \word{ab} $ zwei mögliche Wörter über $A$. \\
		\impl $ w_1 \in A^*, w_2 \in A^*$
	\end{exampleblock}

\end{frame}

\begin{frame}{Wörter -- \thassedaniel{formal betrachtet}{Formalkram}}
	\begin{block}{Formale Definition}
		Ein \textbf{Wort} $w$  über einem Alphabet $A$ ist eine \textit{surjektive} Abbildung $w \from \Z_n \functionto B$ mit $B \subseteq A$. \\ 
		\smallskip
		Zur Erinnerung: \; $ \Z_n = \{i \in \N_0 \mid 0 \leq i < n \} = \set{0, ..., n-1} $ 
	\end{block}
	
	\begin{exampleblock}{Beispiel}
		Sei $ A := \{ \word a, ..., \word z \} $ ein Alphabet und $B := \set{\word a, \word b, \word e, \word n, \word o, \word r, \word s, \word t} \subseteq A$. \\
		Dann ist ein Wort $w$ gegeben durch $ w \from \Z_{12} \functionto B$, \\ 
		\smallskip
		\begin{tabular}{c|c@{\:}c@{\:}c@{\:}c@{\:}c@{\:}c@{\:}c@{\:}c@{\:}c@{\:}c@{\:}c@{\:}c@{\:}c@{\:}c@{\:}c}
			$i$ & \small 0 & \small 1 & \small 2 & \small 3 & \small 4 & \small 5 & \small 6 & \small 7 & \small 8 & \small 9 & \small 10 & \small 11 \\
			\hline
			$w(i)$ & \word a & \word n & \word a & \word n & \word a & \word s & \word s & \word o & \word r & \word b & \word e & \word t 
		\end{tabular} \\
		\bigskip
		Oder einfach wie vorhin: \quad  $w := \word{ananassorbet}$.
	\end{exampleblock}
\end{frame}

\begin{frame}[t]{Das leere Wort}
	\begin{block}{Definition}
		Wir definieren das \textbf{leere Wort} als $$ \varepsilon \from \Z_0  \functionto \set{} \qquad\text{bzw.}\qquad \eps \from \set{} \functionto \set{} $$ \pause

	\end{block}
	
	\begin{block}{Wichtig}
		Das leere Wort ist \textbf{nicht} Nichts, sondern ein echtes Wort! \\
		($0 \in \N_0$ ist ja auch ne Zahl! $\emptyset$ ist auch ne Menge!)
	\end{block}
		%Wichtig: Das leere Wort ist auch ein \enquote{echtes, gleichberechtigtes} Wort. Die Null ist bei den natürlichen Zahlen ja auch nicht einfach \enquote{nichts}. \medskip \pause
		
		\YesQuestionE{Ist $\eps \from \set{} \functionto \set{} $ eine Relation? Und eine Funktion? Ist es surjektiv?}{
			Achtung: Wir müssen fordern, dass Wörter surjektiv sind, sonst ist das leere Wort nicht eindeutig!
		}
	
\end{frame}


\begin{frame}{Konkatenation von Wörtern}
	\begin{block}{Wörter „aneinanderkleben“}
		Seien $w_1 = \word{erd}, w_2 = \word{blumentopf}$. \\
		Dann ist $ w_1 \* w_2 = \word{erdblumentopf} \neq w_2 \* w_1 = \word{blumentopferd}$. \\ 
		\pause
		Konkatenation ist also \textbf{nicht kommutativ!} \\
		\YesQuestionE{Ist sie \textbf{assoziativ}?}{$(w_1 \* w_2) \* w_3 = w_1 \* (w_2 \* w_3)$.} \medskip 
		\only<+->{Außerdem gilt immer $ \eps \* w \* \eps = w $.}
		
	\end{block}
	
	\pause
	
	\begin{block}{Beobachtung}
		Falls $w=w_1\* w_2 $ und $w_1 \in A^* , w_2 \in B^* $, dann ist
		$ w\in (A \cup B)^* $.
	\end{block}
	
\end{frame}
\begin{frame}{Konkatenation von Wörtern}

	\begin{block}{„Wortpotenzen“}
		\begin{align*}
			w^0 &= \eps \\
			w^k &= \underbrace{w \* w  \cdots  w}_{\text{$k$-mal}}
		\end{align*}

	\end{block}

\end{frame}

\begin{frame}{Länge von Wörtern}
	\begin{block}{Definition}
		$\size w$: Die \textbf{Länge} eines Wortes $w$, also die Gesamtanzahl der Zeichen von $w$.
	\end{block}
	
	\begin{block}{Beispiel}
		$$ \size{\word{hallo}} = 5 \qqquad \size \eps = 0$$
	\end{block}

	\pause
	\begin{block}{Lemma}
		$$ \size{a \* b} = \size a + \size b $$ \\
		\pause
		$$ \size{w^k} = k \* \size w $$
	\end{block}

\end{frame}

\begin{frame}{Wörter}
	\begin{block}{Definition}
		$A^n$: \emph{Menge aller Wörter der Länge $n$} über dem Alphabet $A$.\\
		Wie kann man damit $A^*$ ausdrücken? \\
		\pause
		\[ A^* = \bigcup \limits_{i = 0}^\infty A^i \]
	\end{block}
	
	
	\pause
	\begin{block}{Erinnerung}
		$
		\bigcup \limits_{i\in I} M_i = \{ x \mid \text{es gibt ein } i\in I \text{ so, dass } x\in  M_i \}  
		$
	\end{block}
\end{frame}

\begin{frame}{Aufgabe 4}
	\visible<+(-1)>{}
	\begin{itemize}
		\item Welche Wörter lassen sich aus dem Alphabet $A = \{ \word a , \word b \}$ bilden? Was enthält die Menge $A^*$? \\
		\visible<+-|handout:2->{
			\impl $\word a, \word b, \word {aa}, \word {bb}, \word {ab}, \word {ba}, \word {aaa}, \word {bbb}, \dots$ \\
			\impl $A^*$ enthält genau diese Wörter (und auch $\eps$!).
		}
		\item Ist das Wort $w = \word{aabb} \* \word{ba}$ ein Element der Menge $A^5$? \\
		\visible<+-|handout:2->{
			\impl Nein. $w = \word{aabbba}$ ist zwar ein Wort über $A$, aber hat Länge $6 \neq 5$.
		}
		\item Was ist $A^2 \times A^2$? \\
			\visible<+-|handout:2->{
				\impl $ A^2 \times A^2 = \set{(\word{aa},\word {aa}),(\word{aa},\word{bb}),(\word {aa},\word {ab}),(\word {aa},\word {ba}),(\word {bb},\word {aa}), \dots }$
			} \\ \smallskip
			Wir definieren die Abbildung $f \from A^* \times A^* \functionto A^*, \; (w_1, w_2) \mapsto w_1 \cdot w_2$. \\
			Was ist $f(A^2 \times A^2)$? \\
			\uncover<+-|handout:2->{
				\impl $ f(A^2 \times A^2) = \set{\word{aaaa}, \word{aabb}, \word{aaab}, \word{aaba}, \word{bbaa}, \dots } = A^4 $
			}\\
		\bigskip
		Erinnerung: Für $f \from A \functionto B, M \subseteq A$ definieren wir $f(M) = \{f(a) \mid a \in M\}$
	\end{itemize}
\end{frame}

\begin{frame}{Aufgabe 5}
	Es sei $n \in \N_0$. Wir definieren eine Abbildung $\tau: \Pot (\Z_n) \to A^n$, sodass für $M \in \Pot (\Z_n)$ das Wort $w = \tau(M) \in A^n$ mit $A = \set{0, 1}$ eindeutig durch folgende Eigenschaft gegeben ist: Für jedes $i \in \Z_n$ ist $w(i)=1$ genau dann, wenn $i \in M$ ist.


	\visible<+(-1)>{}
	\begin{alist}
		\item Es sei $n=4$. Geben Sie $\tau(M)$ für jedes $M \in \set{\set{1,2,3},\set{0,3},\set{\emptyset}}$ explizit an. \\
		\visible<+-|handout:2->{
			\impl $\tau(\set{\{1,2,3}) = \word{0111}, \tau(\set{0,3}) = \word{1001},\tau(\set{\{\emptyset}) = \word{0000}$
		}
		\item Es sei $n$ wieder beliebig. Zeigen Sie, dass $\tau$ bijektiv ist. \\
		\visible<+-|handout:2->{
			Injektiv: Seien $M_1, M_2 \in \Pot (\Z_n) mit w = \tau(M_1)=\tau(M_2)$ Dann gilt:
			\begin{center}
				$i \in M_1$ gdw. $w(i) = 1$ gdw. $i \in M_2$
			\end{center}
			Also ist $M_1 = M_2$ und damit $\tau$ injektiv. \\
			Surjektiv: Wir beobachten $\size{\Pot (\Z_n)} = 2^{\size{\Z_n}} = 2^n = \size{A}^n = \size{A^n}$. Somit muss $\tau$ auch surjektiv sein.
		}
		\item Es sei $w \in A^n$. Geben Sie eine hinreichende und notwendige Bedingung für $M \in  \Pot (\Z_n)$ an, sodass $\tau(M)=w$ ist. In Ihrer Formulierung darf dabei $\tau$ nirgends vorkommen. \\
		\visible<+-|handout:2->{
			\impl $M = \set{i \in \Z_n \mid w(i) = 1}$
		}

	\end{alist}
\end{frame}


% Übungsaufgabe 4.1 WS22/23
\begin{frame}{Palindrome}
	Sei $A$ ein Alphabet. Wir betrachten die Funktion $\diamond^R: A* \to A*$,
	die jedem Wort $w \in A*$ seine Spiegelung $w^R \in A$ zuordnet.
	Dieser Operator ist festegelegt für jedes $w \in A$ durch
	\begin{align*}
		\size{w^R} &= \size{w} \\
		w^R(i) &= w(\size{w} - i - 1)	\text{für alle } 0 \leq i \leq \size{w}
	\end{align*}
	Ein Wort $w \in A*$ heißt Palindrom, wenn gilt $w^R = w$. $P_A = \set{w \in A* \mid w = w^R}$
	bezeichnet die Menge aller Palindrome über A.

	\visible<+(-1)>{}
	\begin{alist}
		\item Geben Sie alle Palidrome der Länge 4 über $\set{\word a, \word b}$ an. \\
		\visible<+-|handout:2->{
			$\set{\word{aaaa}, \word{abba}, \word{baab}, \word{bbbb}}$
		}
		\item Betrachten Sie nun die $Q_A = \set{ww^R \mid w \in A*}$ \\
			 Zeigen oder widerlegen Sie $P_A \subseteq Q_A$ \\
		\visible<+-|handout:2->{
			$\word{aba} \in P_A$, aber wegen $\size{\word{aba}} = 3$ nicht in $Q_A$.
		}
	\end{alist}
\end{frame}

