\section{Kontextfreie Grammatiken}
\begin{frame}{Kontextfreie Grammatiken}
	\begin{Definition}
		Eine \textbf{kontextfreie Grammatik} ist ein 4-Tupel $G = (N, T, S ,P)$ mit
		\begin{itemize}
			\item[$N$] Alphabet von Nichtterminalsymbolen
			\item[$T$] Alphabet von Terminalsymbolen ($N \cap T = \emptyset$)
			\item[$S$] Startsymbol ($S \in N$)
			\item[$P$] Produktionsmenge ($P \subseteq N \times (N \cup T)^\ast$)
		\end{itemize}
	\end{Definition}
 \end{frame}
 \begin{frame}{Kontextfreie Grammatiken}
        \begin{Beispiel}
		Sei $A$ das deutsche Alphabet (mit Klein-/Großbuchstaben).\\
		$G_{MI} := \left(\{S, M, I, N\}, \, A \cup \N_+, \, S, \, P\right)$ mit
		\begin{align*}
			P = \{S &\to \text{M\word\sp I\word\sp N}, \\
			M &\to \word{Monkey}, \\
			I &\to \word{Island}, \\
			N &\to \word 1 \mid \word 2 \mid \word 3 \}
		\end{align*}
	\end{Beispiel}
\end{frame}

\begin{frame}{Kontextfreie Grammatiken}
	\begin{block}{Produktionen}
		$=$ Menge von Ersetzungsregeln. \\
		Schema: \\
		\qqquad $\underbrace{L}_{\mathclap{\text{\textbf{genau ein} Nichtterminal}}} \to [\text{rechte Seite aus \word{Terminalen} und/oder Nichtterminalen}]$ \\
		\smallskip
		Heißt: Wir können in \textbf{einem} Ersetzungsschritt \textbf{genau ein} solches Nichtterminal $L$ mit der rechten Seite ersetzen. \textbf{Wenn wir wollen}. \\
		\medskip
		\pause
		Mehrere Möglichkeiten zur Auswahl: \\
		\qquad $L \to [\text{rechte Seite}]_1 \mid [\text{rechte Seite}]_2 \mid [\text{rechte Seite}]_3 \mid ...$ \\
		Heißt: $L$ kann durch die rechte Seite 1, 2 oder 3 ersetzt werden. \textbf{Wie wir wollen.} \\
	\end{block}
\end{frame}

\begin{frame}{Kontextfreie Grammatiken}
    \begin{Beispiel}
        $S \to \word aB\word x \mid \word{zz}$ \\
        \impl $S$ kann durch $\word aB\word x$ oder \word{zz} ersetzt werden.
    \end{Beispiel}
\end{frame}

\begin{frame}{Ableitungen von Wörtern}
	\begin{Definition}
		Wort $u = .....X.....$ \textbf{ableitbar} nach $v = ..... w......$ \\
		wenn es eine Produktion $X \to w$ gibt. \; ($X$ ist Nichtterminal.) \\
		Schreibweise: \quad $u \derives v$ \quad oder \quad $u \derives^1 v$ \\
	\end{Definition}
	
	\medskip
	\textbf{\alert{Achtung}}: Aufpassen mit Pfeilen: \quad  $\derives$ vs. $\to$
	
	\pause
	\begin{Beispiel}
		Für $G_{MI}$ gilt: \\
		$S \derives M\word{\sp} I\word{\sp}N$\\
		$ M\word{\sp} I\word{\sp}N \derives \word{Monkey}\word{\sp} I\word{\sp} N \derives \word{Monkey}\word{\sp} I\word{\sp 3}$ \\
		$\word{Monkey}\word{\sp} I\word{\sp} N \derives \word{Monkey}\word{\sp} I\word{\sp 1} \derives \word{Monkey}\word{\sp}\word{Island}\word{\sp 1}$ \\
		Es gibt kein Wort, das aus \word{Monkey\sp Island\sp 1} abgeleitet werden kann.
	\end{Beispiel}
	
\end{frame}

\begin{frame}{Ableitung}	
	\begin{block}{Schreibweisen}
		$u \derives^k v$ \quad heißt: $u$ in $k$ Schritten ableitbar zu $v$ \\
		$u \derives^0 v$ \quad heißt also: in 0 Schritten ableitbar ($u=v$) \\
		\smallskip
		$u \derives^* v$ \quad heißt: $u$ in \emph{beliebig vielen} Schritten ableitbar zu $v$
	\end{block}
	
	\mycomment{
		\pause
		\begin{block}{Beobachtung}
			Die Definitionen stimmen mit den Potenzen der Relation $\derives$ überein.\\
			$\derives^\ast$ ist die reflexiv-transitive Hülle von $\derives$.
		\end{block}
	}
	
	\pause
	\begin{Beispiel}
		Für $G_{MI}$ gilt: $S \derives^3 \word{Monkey\sp Island\sp}N \derives^1 \word{Monkey\sp Island\sp3}$\\
		$S \derives^* \word{Monkey\sp Island\sp2}$
	\end{Beispiel}
\end{frame}


\begin{frame}{Erzeugte Sprache einer Grammatik}
	\begin{Definition}
		Sei $G = (N, T, S, P)$ eine kontextfreie Grammatik. Wir nennen die Sprache $$L(G) := \{w \in T^\ast \mid S \Rightarrow^\ast w \} \subseteq T^*$$ die von $G$ \textbf{erzeugte Sprache}.
	\end{Definition} \pause
	Das sind also alle Wörter aus \emph{Terminalsymbolen}, die vom Startsymbol aus ableitbar sind.\\
	\bigskip
	Achtung: Die erzeugte Sprache kann auch \textbf{leer} sein. \\
	\§{Beispiele:} \pause $L\left(\left(\{X\},\, \{\word a, \word b\},\, X\, ,\{X\to X\}\right)\right) = \emptyset$ \pause \quad oder \\
	\. $L\left(\left(\{X\},\, \{\word a, \word b\},\, X\, ,\emptyset \right)\right) = \emptyset$
\end{frame}

\begin{frame}{Erzeugte Sprache}
	\begin{Beispiel}
		$L(G_{MI}) = \{\word{Monkey\sp Island\sp1}, \word{Monkey\sp Island\sp 2}, \word{Monkey\sp Island\sp 3}\}$\\
		$M\wordsp I\wordsp N \notin L(G_{MI})$ \quad (enthält nämlich  Nichtterminale!)
	\end{Beispiel}
	
	\pause
	\begin{Definition}
		Eine Sprache $L$, für die irgendeine kontextfreie Grammatik G mit $L(G) = L$ existiert, heißt \textbf{kontextfrei}.
	\end{Definition}
	\medskip
	
	Viele Programmiersprachen sind kontextfrei. \\  
	% ACHTUNG, streng genommen falsch. Jede Sprache, in der nur zuvor deklarierte Bezeichner als solche gültig sind, ist eigentlich kontext-sensitiv. Aber viele solcher Sprachen werden beim Parsen mittels einer kontextfreien Grammatik zerlegt; Deklariertheit von Bezeichnern wird dann später überprüft. Daher die allgemein für solche Sprachen gängige Bezeichnung „kontextfrei“.
	% Fun fact: Einrücksensitive Sprachen (wie Python – oder Haskell <3) sind dies streng genommen nicht! Das Lexing kann dann nicht von einem Regex-Automaten gemacht werden, der („übermächtige“ Stack-Automaten-)Lexer muss dann den Input vorverarbeiten, der daraus entstehende Tokenstream kann dann kontextfrei geparst werden.  Source: http://trevorjim.com/python-is-not-context-free/
	Eine vereinfachte Variante der englischen Sprache auch. \smiley
% Grob falsch:
%	Viele \enquote{natürlich vorkommende} Sprachen sind kontextfrei.
		
\end{frame}

\subsection{Beispiele}
\begin{frame}{Beispiel}
	$$ G = (\{X\}, \, \{\word a, \word  b\}, \, X, \, \{X \to \word aX\word b \mid \eps\}) $$
	\delimitershortfall=1pt
	\begin{itemize}
		\item Gilt $X \derives \word aX\word b$, \; $X \derives \word{aa}X\word{bb}$, \; $XX \derives \word{a}X\word{ba}X\word{b}$? \\
			  \visible<2-|handout:2->{Ja, Nein, Nein.}
		\item Welche Wörter über $\{\word a, \word  b\}$ lassen sich aus $\word{aa}X\word{bb}$ ableiten? Und aus $XX$? \\
			  \visible<3-|handout:2->{
				  Aus $\word{aa}X\word{bb}$: \quad $\set{\word a^k\word b^k \mid k \geq 2}$ \\
				  Aus $XX$: \quad $\set{\word a^k\word b^k \word a^\ell \word b^\ell \mid k,\ell \in \N_0}$
			  }
		\item Gebt $L(G)$ an! \\
		      \visible<4-|handout:2->{ $L(G) = \set{\word a^k\word b^k\mid k \in \N_0}$ }
	\end{itemize}
	
\end{frame}

\begin{frame}{Klammerausdrücke}
	Gegeben sei die Grammatik $$G = (\{X\}, \{\word{(}, \word)\}, X, \{X \to XX \mid \word(X\word) \mid \eps\})$$
	\begin{itemize}
		\item Wie leitet man \word{(())} ab?
		\item Wie leitet man \word{()()} ab?
		\item Kann man \word{(()(} ableiten? \pause \impl Nein!
		\item Und wie leitet man \word{(())()()} ab?\\ \pause
			$X \derives XX \derives XXX \derives \word(X\word)XX \derives \word(X\word)X\word(X\word) \derives \word(X\word)\word(X\word)\word(X\word)$ \\
			$\quad \derives \word(X\word{)()(}X\word) \derives \word{((}X\word{))()(}X\word) \derives \word{((}X\word{))()()} \derives \word{(())()()}$\\
			Geht das auch übersichtlicher? \impl Ableitungsbäume
	\end{itemize}
\end{frame}


\begin{frame}{Ableitungsbäume}
	\centering
	$G = (\{X\}, \{\word{(}, \word)\}, X, \{X \to XX \mid \word(X\word) \mid \eps\})$
	\medskip
	
	\begin{tikzpicture}
	[level 1/.style={sibling distance=50mm},
	level 2/.style={sibling distance=30mm},
	level 3/.style={sibling distance=10mm}]
	\node {$X$}
	child { node {$X$}
		[level 2/.style={sibling distance=15mm}]
		child {node {$\literal{(}$} }
		child {node {$X$} 
			child {node {$\literal{(}$} }
			child {node {$X$}
				child {node {$\varepsilon$} }
			}
			child {node {$\literal{)}$} }
		}
		child {node {$\literal{)}$} }
	} 
	child { node {$X$} 
		child { node {$X$} 
			child {node {$\literal{(}$} }
			child {node {$X$}
				child {node {$\varepsilon$} }
			}
			child {node {$\literal{)}$} }
		}
		child { node {$X$} 
			child {node {$\literal{(}$} }
			child {node {$X$}
				child {node {$\varepsilon$} }
			}
			child {node {$\literal{)}$} }
		}
	} ;
	\end{tikzpicture}
\end{frame}


\begin{frame}{Klammerausdrücke}
	Gegeben sei die Grammatik $$G = (\{X\}, \{\word{(}, \word)\}, X, \{X \to XX \mid \word(X\word) \mid \eps\})$$
	Was ist $L(G)$? Was kann man also aus $X$ ableiten?\\ \pause 
	Alle \enquote{\textit{wohlgeformten Klammerausdrücke}} \quad ($=$ alle, die Sinn machen).\\[1em]
	
	Was bedeutet wohlgeformt in diesem Kontext?\\ \pause
	$$\forall w \in L(G): N_{\word(}(w) = N_{\word)}(w)$$ 
	Reicht das? \pause \impl Notwendig, aber nicht hinreichend! 

\end{frame}

\begin{frame}{Klammerausdrücke}
	Wir dürfen eine Klammer erst schließen, \textit{nachdem} wir sie geöffnet haben.\\
	Also: Anzahl der schließenden Klammern darf nie größer als Anzahl der öffnenden Klammern sein! \pause \\[1em]
	\impl Für jedes Präfix $v$ von einem Wort $w \in L(G)$ gilt $$N_{\word(} (v) \geq N_{\word)} (v)$$ \pause
	
	\textbf{Achtung}: Grammatiken sind \textbf{nicht eindeutig}! Wir können zur gleichen Sprache mehrere verschiedene erzeugende Grammatiken finden. \\
	Alternative Grammatik für wohlgeformte Klammerausdrücke: \pause $$G = (\{X\}, \, \{\word (, \word )\}, \, X, \, \{X \to \word(X\word)X \mid \varepsilon\})$$
\end{frame}

\begin{frame}{Und jetzt ihr...}
	Gebt jeweils eine Grammatik über dem Alphabet $T = \{\word a, \word b\}$ an, die folgende Sprache erzeugt:
	\begin{itemize}
		\item Alle Wörter, in denen irgendwo das Teilwort $\word{baa}$ vorkommt.\\
		\visible<2-|handout:2>{
			$(\{X,Y\},T,X,P)$ mit $P=\{X \to Y\word{baa}Y, Y \to \word aY \mid \word bY \mid \varepsilon\}$
		}
		
		\item Alle Wörter, in denen $\word{ab}$ als Teilwort vorkommt oder kein $\word a$ enthalten ist. \\
		\visible<3-|handout:2>{
			$G = (\{X, Y\}, T, X, P)$ mit $P = \{X \to \word bX \mid Y\word{ab}Y \mid \varepsilon, Y \to \word aY \mid \word bY \mid \varepsilon\}$
		}
		
		\item Die Menge aller Wörter $w\in T^*$ mit der Eigenschaft, dass
		für alle Präfixe $v$ von $w$ gilt: $\setsize{N_{\word{a}}(v) - N_{\word{b}}(v)} \leq
		1$.\\
		\emph{Tipp}: Was für eine Struktur haben Wörter der Länge $2$, $4$, \dots? \\
		% $\{ab, ba\}^*$
		\visible<4-|handout:2>{
			$(\{X,Y\},T,X,P)$ mit $P=\{X \to \word{ab}X \mid \word{ba}X \mid \word a \mid \word b \mid \varepsilon\}$
		}
		
	\end{itemize}
\end{frame}


\begin{frame}{Aufgabe: Palindrome}
	\begin{itemize}
		\item Geben Sie eine kontextfreie Grammatik $$G = (N, \{\word a, \word b\}, S, P )$$ an, für die $L(G)$ die Menge aller Palindrome über dem Alphabet $\{\word a, \word b\}$ ist.
		\item Geben Sie eine Ableitung der Wörter \word{baaab} und \word{abaaaba} aus dem Startsymbol Ihrer Grammatik an.
		\item Beweisen Sie, dass Ihre Grammatik jedes Palindrom über dem Alphabet $\{\word a, \word b\}$ erzeugt.\\
		(\emph{Tipp}: Induktion: Wenn's für $n$ und $n+1$ gilt, dann gilt's auch für $n+2$.)
	\end{itemize}
\end{frame}

\begin{frame}{Lösung}
	Die Grammatik $$G = (\{S\}, \{\word a, \word b\}, S, P = \{S \to \word aS\word a \ | \ \word bS\word b \ | \ \word a \ | \ \word b \ | \ \varepsilon \})$$ erzeugt gerade die Menge der Palindrome. \pause Die Ableitungen der Wörter mit dieser Grammatik sind 
	$$S \derives \word bS\word b \derives \word{ba}S\word{ab} \derives \word{baaab}$$
	$$S \derives \word aS\word a \derives \word{ab}S\word{ba} \derives \word{aba}S\word{aba} \derives \word{abaaaba}$$
\end{frame}

\begin{frame}{Lösung}
	Sei $w$ ein Palindrom über $\{\word a,\word b\}$. Wir zeigen durch Induktion über $n = \size{w}$, dass alle Palindrome aus $S$ abgeleitet werden können. \pause
	\begin{block}{Induktionsanfang} \pause
		Für $n = 0$ ist das leere Wort $\varepsilon$ in einem Schritt aus $S$ ableitbar. \\
		Für $n=1$: Die einzigen Wörter aus $\{\word a,\word b\}^\ast$ der Länge 1 sind $\word a$ und $\word b$. Auch diese sind offensichtlich aus $S$ ableitbar.
	\end{block}
 	\pause
	\begin{block}{Induktionsvoraussetzung} \pause
		Für ein festes, aber beliebiges $n \in \N_0$ gilt, dass alle Palindrome der Länge $n$ und alle Palindrome der Länge $n + 1$ aus S abgeleitet werden können.
	\end{block}
\end{frame}

\begin{frame}{Lösung}
		\begin{block}{Induktionsschritt}
			Sei $w$ ein Palindrom der Länge $n + 2$. Das erste (und damit auch das letzte) Zeichen sei oBdA. ein $\word a$. Dann gibt es ein $w' \in \{\word a, \word b\}^\ast$, so dass $w = \word aw'\word a$ ist. Da $w$ ein Palindrom ist, muss auch $w'$ ein Palindrom sein. Weiterhin gilt $\size{w'} = n$. \pause Nach IV gibt es somit eine Ableitung $S \derives^\ast w'$. Somit gibt es die Ableitung $$S \derives \word aS\word a \overset{IV}{\derives^\ast} \word aw'\word a = w$$ und $w \in L(G)$ folgt. \pause Entsprechendes gilt, wenn das erste Zeichen von w ein \word b ist. \\
			Mit der IV haben wir also gezeigt, dass auch Palindrome der Länge $n+1$ und $n+2$ aus $S$ ableitbar sind.
	\end{block}

\end{frame}

\begin{frame}{Aufgabe Grammatiken}
	\begin{block}{Aufgabe}
		Es sei $G = (\set{S, X, Y}, \set{\word{a}, \word{b}}, S, P)$ mit $P = \set{S \to SS \mid XX, X \to \word a X \word b \mid \word b X \word b \mid Y, Y \to \word{aa} \mid \word{bb} \mid \word a \mid \word b}$

		
		\begin{itemize}
			\item[a)] Zeichnen Sie einen Ableitungsbaum zu dem Wort $\word{abbaabbbabbba} \in L(G)$.
			\item[b)] Geben Sie Wörter $u, v, x \in \set{S, X, Y, \word a, \word b}^*$ mit $S \derives^2 u \derives v \derives^* x \derives \word{abbaabbbabbba}$ an.
			\item[c)] Geben Sie die Sprache $L(G)$ an, ohne auf $G$ bezug zu nehmen.\\
			\textbf{Hinweis:} Eine Möglichkeit, das zu tun, ist es die Menge $M = \set{w \in \set{\word a, \word b}^* \mid X \derives^* w} \subseteq A^*$ der aus dem Nichtterminal $X$ ableitbaren Worte anzugeben und mit Hilfe von $M$ einen Ausdruck in Mengenschreibweise für $L(G)$ anzugeben.
		\end{itemize}
	\end{block}
\end{frame}

\begin{frame}{Lösung}
	\begin{itemize}
		\item[b)] $u = SXX, v=SXY, w = \word{abbaabbbab}Y\word{ba}$
		\item[c)] $M = \set{w \in \set{\word{a}, \word{b}}^* \mid w = w^R \land \size{w} \ge 0}, L(G) = (MM)^+$
	\end{itemize}
\end{frame}



\begin{frame}{Gibt es noch mehr?}
	Viele Sprachen in der Informatik sind kontextfrei.\\[1em]
	Was ist mit der Sprache $L_{vv} = \{v\word cv \mid v \in \{\word a,\word b\}^*\}$\\
	\smallskip
	\pause
	In der Vorlesung: Es gibt keine kontextfreie Grammatik, die $L_{vv}$ erzeugt.\\
	\medskip
	\pause
	Können wir die Sprache trotzdem irgendwie \enquote{verarbeiten}?
	
	\begin{block}{}
		\Large
		\centering
		Soon...\\[1em]
	\end{block}
\end{frame}