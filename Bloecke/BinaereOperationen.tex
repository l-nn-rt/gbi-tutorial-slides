\section{Binäre Operatoren}
\begin{frame}
	\begin{block}{Binäre Operatoren}
		Eine binäre Operation auf einer Menge $M$ ist eine Abbildung:
		\[ \diamond: M \times M \to M \]
		Üblicherweise benutzt man (wie bei manchen Relationen) die Infixschreibweise.
	\end{block} \pause
	\begin{block}{Eigenschaften}
		Die Operation $\diamond$ heißt kommutativ, wenn gilt:
		\[ \forall x \in M \ \forall y \in M: x \diamond y = y \diamond x \]
		
		Die Operation $\diamond$ heißt assoziativ, wenn gilt:
		\[ \forall x \in M \ \forall y \in M \ \forall z \in M : (x \diamond y) \diamond z = x \diamond (y \diamond z) \]
	\end{block} \pause


\end{frame}