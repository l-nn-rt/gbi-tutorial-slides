\section{Automaten}

\begin{frame}{Endliche Automaten}
	Ein deterministischer endlicher Automat...
	\begin{itemize}
		\item besteht aus endlich vielen Zuständen
		\item ist zu jedem Zeitpunkt in \emph{genau einem} dieser Zustände
		\item wechselt bei Eingabe \emph{genau eines} Zeichens den Zustand in \emph{genau einen, definierten} Folgezustand
		\item und gibt dabei ein Wort als Ausgabe aus
	\end{itemize}

	Die gültigen \enquote{Zustandswechsel} sind als Zustandsübergänge definiert.
\end{frame}

\begin{frame}{Anwendungen}
	\begin{itemize}
		\item Getränkeautomaten
		\item Textsuche
		\item Textersetzung
	\end{itemize}

	Für Beispiele zur Textsuche und Textersetzung: Übung~12, WS~15/16
\end{frame}

\mycomment{ % As a source for copy/pasting:
\begin{tikzpicture}[->,>=stealth,shorten >=1pt,auto,node distance=2.8cm,
semithick,initial text={}]
\tikzstyle{every state}=[fill=red,draw=none,text=white]

\node[initial,state] (A)                    {$q_a$};
\node[state]         (B) [above right of=A] {$q_b$};
\node[state]         (D) [below right of=A] {$q_d$};
\node[state]         (C) [below right of=B] {$q_c$};
\node[state]         (E) [below of=D]       {$q_e$};

\path (A) edge              node {0,1,L} (B)
		  edge              node {1,1,R} (C)
(B) edge [loop above] node {1,1,L} (B)
	edge              node {0,1,L} (C)
(C) edge              node {0,1,L} (D)
	edge [bend left]  node {1,0,R} (E)
(D) edge [loop below] node {1,1,R} (D)
	edge              node {0,1,R} (A)
(E) edge [bend left]  node {1,0,R} (A);
\end{tikzpicture}	
}

% \subsection{Mealy-Automaten}
% \begin{frame}{Ein einfacher Automat}
% 	\begin{center}
% 		\begin{tikzpicture}[->,>=stealth,shorten >=1pt,auto,node distance=2.8cm,
% 			semithick,initial text={}]
% 			\tikzstyle{every state}=[]
			
% 			\node[initial,state] (A)                    {$a$};
% 			\node[state]         (M) [right of=A] 	    {$m$};
			
% 			\path (A) edge [loop above] node {$\word 1\io\word 0$} (A)
% 			          edge [loop below]  node {$\word 0\io\word 1$} (A)
% 			          edge 					  node {$\word{2}\io\word{X}$} (M)
% 			      (M) edge [loop right] node {$\stackedtight{\word 0\io\word X \\ \word 1\io\word X \\ \word 2\io\word X}$} (M);
% 			      %(M) edge [loop right] node {\word 1|\word X} (M)
% 			      %(M) edge [loop below] node {\word 2|\word X} (M);
% 		\end{tikzpicture}
% 	\end{center}
% 	\pause
% 	Eingabe: \word{011101} \?> Ausgabe: \word{100010} \\
% 	Eingabe: \word{011102} \?> Ausgabe: \word{10001X} \\
% 	Eingabe: \word{012101} \?> Ausgabe: \word{10XXXX} \\ \pause
% 	\smallskip
% 	\impl Der Automat negiert das Eingabewort bitweise. \\
% 	(Frisst er ne \word 2, so isser beleidigt. \smiley)
% \end{frame}

% \begin{frame}{Mealy-Automat}
% 	Bei einem Mealy-Automaten findet die Ausgabe \textbf{bei den Zustandsübergängen} statt.
% 	\begin{Definition}
% 		Ein \textbf{Mealy-Automat} $ A = (Z,z_0,X,f,Y,g)$ besteht aus...
% 		\begin{itemize}
% 			\item einer endlichen Zustandsmenge $Z$ 
% 			\item einem Startzustand $z_0$
% 			\item einem Eingabealphabet $X$ 
% 			\item einer Zustandsübergangsfunktion $f \from Z\times X \functionto Z $
% 			\item einem Ausgabealphabet $Y$
% 			\item einer Ausgabefunktion $g \from Z\times X \functionto Y^* $ 
% 		\end{itemize}						
% 	\end{Definition}
% \end{frame}


% \begin{frame}{}
% 	\begin{block}{Graphische Darstellung}
% 		\begin{itemize}
% 			\item Oft malt man Automaten hin. (Achtung: Die formale Schreibweise wird trotzdem auch im Studium verwendet! Also auswendig lernen!)
% 			\item Zustände sind \textbf{Knoten} und Übergänge sind \textbf{gerichtete Kanten} in einem Graphen. \\
% 			Kanten\textbf{beschriftung}: „$\langle\text{Eingabe}\rangle | \langle\text{Ausgabe}\rangle$“
% 			\item \textbf{Der Startzustand wird mit einem Pfeil \enquote{aus dem Nichts} gekennzeichnet!}  \pause
% 			\implitem \textbf{NICHT VERGESSEN!} Kostet sonst Punkte! \textbf{Immer!}
% 		\end{itemize}
		
% 	\end{block}
% \end{frame}

% % \begin{frame}{Beispiel: Automatengraphen}
% % 	%\vspace{-1\baselineskip} 
% % 	%\hspace{-\baselineskip}
% % 	\includegraphics[page=3,trim=18px 47px 18px 14px,clip, scale=0.9]{U12.pdf}	
% % \end{frame}

% %% Übung: Beispiel
% %\setbeamercolor{background canvas}{bg=}


% \begin{frame}{Eingabe eines Zeichens}
% 	Was ist die Zustandsübergangsfunktion $f$? \\ \pause
% 	\medskip
% 	Haben
% 	\begin{itemize}
% 		\item aktuellen Zustand $z \in Z$
% 		\item eingelesenes Zeichen $x \in X$
% 	\end{itemize}
% 	\impl $f$ liefert den Folgezustand $z_\text{neu}$, in welchem der Automat danach ist.\\
% 	\medskip
% 	Formell: $$ f(z,x) = z_\text{neu} $$ 
% \end{frame}

% \begin{frame}{Eingabe eines Wortes}
% 	Was machen wir mit ganzen Wörtern? \\ \pause
% 	Wir definieren uns ganz analog
% 	\begin{block}{Zustandsübergangsfunktion extended}
% 	\begin{align*}
% 		 	f_*(z,\eps) &:= z \\
% 		 	\forall w\in X^* , \, x\in X : \quad  f_*(z,wx) &:= f\left(f_*(z,w),x\right) 
% 	\end{align*}
% 	\end{block}
% 	\pause Was macht diese Funktion? \\ \pause
% 	\medskip
% 	Haben
% 	\begin{itemize}
% 		\item aktuellen Zustand $z \in Z$
% 		\item eingelesenes Wort $w \in X^*$
% 	\end{itemize}
% 	\impl $f_*$ liefert den \textbf{Endzustand} $z_\text{ende}$, in welchem der Automat \textbf{nach dem Wort} dann ist.
% 	Formell: $$ f_*(z,w) = z_\text{ende} $$
% 	%Bei Eingabe eines Wortes $w$ und Anfangszustand $z$ gibt sie den Zustand $z' = f_*(z,w)$ aus, in dem der Automat enden wird. 
% \end{frame}
	
% \begin{frame}{Eingabe eines Wortes}
% 	Definieren wir nun weiter 
% 	\begin{block}{Zustandsübergangsfunktion extended extended}
% 		\begin{align*}
% 			f_{**}(z, \eps) &:= z \\
% 			\forall w\in X^* , \, x\in X : \quad  f_{**}(z, xw)   &:= z \cdot f_{**}\!\left(f(z,x),w\right) \\
% 		\end{align*}
% 	\end{block}
% 	\pause Was macht diese Funktion? \\ \pause
% 	Diese Funktion gibt die Reihe aller \textbf{Zustände als Gesamtwort} aus, die der Automat bei Eingabe des Wortes $w$ im Startzustand $z$ durchläuft. Also: 
% 	$$ f_{**}(z, w) = z \* z_1 \* z_2 \cdots z_\text{ende} $$
% \end{frame}

% \begin{frame}{Eingabe eines Wortes}		
% 	Alternative Definition:
% 	\begin{block}{Zustandsübergangsfunktion extended extended}
% 		\begin{align*}
% 			f_{**}(z,\varepsilon) &= z \\
% 			\forall w \in X^* \ \forall x \in X : f_{**}(z,wx) &= f_{**}(z,w)\cdot f(f_*(z,w),x)	 
% 		\end{align*}
% 	\end{block}
% \end{frame}

% \begin{frame}{Beispiel $f, f_*, f_{**}$}
% 	\begin{center}
% 			\begin{tikzpicture}[->,>=stealth,shorten >=1pt,auto,node distance=2.8cm,
% 		semithick,initial text={}]
% 		\tikzstyle{every state}=[]
		
% 		\node[initial,state] (A)                    {$a$};
% 		\node[state]         (M) [right of=A] 	    {$m$};
		
% 		\path (A) edge [loop above] node {$\word 1\io\word 0$} (A)
% 		edge [loop below]  node {$\word 0\io\word 1$} (A)
% 		edge 					  node {$\word{2}\io\word{X}$} (M)
% 		(M) edge [loop right] node {$\stackedtight{\word 0\io\word X \\ \word 1\io\word X \\ \word 2\io\word X}$} (M);
% 		\end{tikzpicture}
% 	\end{center}
%     \begin{columns}
%         \begin{column}{0.5\textwidth}
%             \begin{align*}
%         		f(a, \word 2) &= m \\
%         		f(a, \word 0) &= a \\ 
%         	\end{align*}
%         	\pause  \vspace{-2.5\baselineskip}
%         	\begin{align*}
%         		f_*(a, \word{101}) &= \hphantom{aaa}a \\
%         		f_{**}(a, \word{101}) &= aaaa \\ 
%         	\end{align*}
%         \end{column}
%         \begin{column}{0.5\textwidth}
%             \pause  \vspace{-5\baselineskip}
%         	\begin{align*}
%         		f_*(a, \word{1021}) &= \hphantom{aaam}m \\
%         		f_{**}(a, \word{1021}) &= aaamm
%         	\end{align*}
%         \end{column}
%     \end{columns}
	
	
% \end{frame}

% \begin{frame}{Ausgaben}
% 	Automaten können ja was ausgeben. \\
% 	\smallskip
% 	Erinnerung: Die Kanten sind beschriftet mit $x \io y$ , wobei $x\in X$ und $y\in Y^* $. \\
% 	 \impl Für Eingabe $x$ wird das Wort $y$ ausgegeben. \\ 
% 	Formal: $$g(z,x) = y$$
% \end{frame}

% \begin{frame}{Ausgabefunktionen}
% 	\begin{block}{Ausgabefunktion extended}
% 		Wir können also analog zu $f_*$ und $f_{**}$ definieren
% 		\begin{threealign}
% 		\text{Für die letze Ausgabe: } \qquad g_* : Z\times X^* &\functionto& Y^* \\
% 		g_*(z,\varepsilon) &=& \varepsilon \\
% 		\forall w\in X^* \; \forall x\in X : g_*(z,wx) &=& g(f_*(z,w),x) \\ \\
% 		\text{Für das ganze Ausgabewort: } \qquad g_{**} : Z\times X^* &\functionto& Y^* \\ 
% 		g_{**}(z,\varepsilon) &=& \varepsilon \\
% 		\forall w\in X^* \; \forall x\in X : g_{**}(z,wx) &=& g_{**}(z,w)\cdot g_*(z,wx) 			
% 		\end{threealign} 
% 	\end{block}
	
% 	%\pause Dies gibt nun nicht die durchlaufenen Zustände bzw. den Abschlusszustand an, sondern die letzte Ausgabe bzw. alle durchlaufenen Ausgaben. 

% \end{frame}

% \begin{frame}{Beispiel $g, g_*, g_{**}$}
% 	\begin{center}
% 		\begin{tikzpicture}[->,>=stealth,shorten >=1pt,auto,node distance=2.8cm,
% 		semithick,initial text={}]
% 		\tikzstyle{every state}=[]
		
% 		\node[initial,state] (A)                    {$a$};
% 		\node[state]         (M) [right of=A] 	    {$m$};
		
% 		\path (A) edge [loop above] node {$\word 1\io\word 0$} (A)
% 		edge [loop below]  node {$\word 0\io\word 1$} (A)
% 		edge 					  node {$\word{2}\io\word{X}$} (M)
% 		(M) edge [loop right] node {$\stackedtight{\word 0\io\word X \\ \word 1\io\word X \\ \word 2\io\word X}$} (M);
% 		\end{tikzpicture}
% 	\end{center}
% 	% \vspace{-\baselineskip}
%  \begin{columns}
%      \begin{column}{.5\textwidth}
%         \begin{align*}
%         	g(a, \word 2) &= \word X \\
%         	g(a, \word 0) &= \word 1 \\ 
%     	\end{align*}
%     	\pause  \vspace{-2.5\baselineskip}
%     	\begin{align*}
%         	g_*(a, \word{101}) &= \hphantom{\word{01}}\word 0 \\
%         	g_{**}(a, \word{101}) &= \word{010} \\ 
%     	\end{align*} 
%      \end{column}
%      \begin{column}{.5\textwidth}
%         \pause  \vspace{-3.5\baselineskip}
%         \begin{align*}
%             g_*(a, \word{1021}) &= \hphantom{\word{01X}}\word X \\
%             g_{**}(a, \word{1021}) &= \word{01XX} \\ 	\visible<+->{}
%             \visible<+->{g_*(m, \word{0110}) &= }\visible<+->{\word X}
%         \end{align*}
%      \end{column}
%  \end{columns}
	
	
% \end{frame}


\subsection{Moore-Automat}
% \begin{frame}{Moore-Automat}
% 	Moore-Automaten sind fast genauso aufgebaut wie Mealy-Automaten.\\ \smallskip
% 	\textbf{Unterschied}: Ausgabe erfolgt „\textbf{im Zustand}“, nicht beim Zustandsübergang. \\ \pause 
% 	\begin{tabular}{cc}
% 		Mealy: & Moore: \\
% 		\begin{tikzpicture}[->,>=stealth,shorten >=1pt,auto,node distance=2.8cm,
% 		semithick,initial text={}]
% 		\tikzstyle{every state}=[]
		
% 		\node[state,white] (X)                    {\hphantom{XXX}};
% 		\node[state]         (A) [right of=X] 	    {$A$};
		
% 		\path (X) edge [bend left=9]	 node {$\word a\io\word 0$} (A)
% 				  edge [bend right=9]	 node [below] {$\word b\io\word 1$} (A);
% 		\end{tikzpicture}
% 		&
% 		\begin{tikzpicture}[->,>=stealth,shorten >=1pt,auto,node distance=2.8cm,
% 		semithick,initial text={}]
% 		\tikzstyle{every state}=[]
		
% 		\node[state,white] (X)                    {\hphantom{XXX}};
% 		\node[state]         (A) [right of=X] 	    {$A\io\word{XY}$};
		
% 		\path (X) edge [bend left=9]	 node {\word a } (A)
% 				  edge [bend right=9]	 node [below] {\word b } (A);
% 		\end{tikzpicture}
% 	\end{tabular} \\
% 	\pause
% 	Ausgabefunktion heißt jetzt also auch: $$h\from Z\functionto Y^*$$ \impl Ausgabe hängt \textbf{nicht} von der Eingabe ab!\\
% \end{frame}

\begin{frame}{Ein einfacher Automat}
	\begin{columns}
	\begin{column}{.5\textwidth}
	\begin{tikzpicture}[->,>=stealth,shorten >=1pt,auto,node distance=2.1cm,
		semithick,initial text={}]
		\tikzstyle{every state}=[]
		
		\node[initial,state] (A)                    {$a\io\eps$};
		\node[state]         (B) [above right of=A] 	    {$b\io\word 0$};
		\node[state]         (C) [below right of=A] 	    {$c\io\word 1$};
		\node[state]         (M) [below right of=B] 	    {$m\io\word X$};
		
		\path 
			(A) edge 				node {\word 1} (B)
			(A) edge 				node [below] {\word 0} (C)
			(B) edge [bend left=8]	node {\word 0} (C)
			(B) edge [loop right]   node {\word 1} (B)
			(C) edge [bend left=8]	node {\word 1} (B)
			(C) edge [loop right]   node {\word 0} (C)
			(B) edge 				node {\word 2} (M)
			(C) edge 				node [below] {\word 2} (M)
			(M) edge [loop right]	node {\word 0, \word 1, \word 2} (M)
		;
		\draw[->] (A) to[out=-90,in=-76, looseness=1.94] (M) node [at start, below right = 48pt and 2pt] {\word 2}; %Fuck you, LaTeX. "midway" node option doesn't work at all. Beware: `below=48pt, right=2pt` doesn't seem to work anymore?! WTF!?
		\end{tikzpicture}
	\end{column}
	\begin{column}{.5\textwidth}
		\pause
		Eingabe: \word{011101} \?> Ausgabe: \word{100010} \\
		Eingabe: \word{011102} \?> Ausgabe: \word{10001X} \\
		Eingabe: \word{012101} \?> Ausgabe: \word{10XXXX} \\ \pause
		\smallskip
		\impl Der Automat negiert das Eingabewort bitweise. \\
		(Frisst er ne \word 2, so isser beleidigt. \smiley)
	\end{column}
	\end{columns}
\end{frame}

\begin{frame}{Moore-Automat}
	Bei einem Moore-Automaten findet die Ausgabe \textbf{im Zustand} statt.
	\begin{Definition}
		Ein \textbf{Moore-Automat} $ A = (Z,z_0,X,f,Y,h)$ besteht aus...
		\begin{itemize}
			\item einer endlichen Zustandsmenge $Z$ 
			\item einem Startzustand $z_0$
			\item einem Eingabealphabet $X$ 
			\item einer Zustandsübergangsfunktion $f \from Z\times X \functionto Z $
			\item einem Ausgabealphabet $Y$
			\item einer Ausgabefunktion $h \from Z \functionto Y^* $ 
		\end{itemize}						
	\end{Definition}
\end{frame}


% \begin{frame}{Moore-Automat – Ausgabe}
% 	Ausgabefunktion: $$h\from Z\functionto Y^*$$ \impl Ausgabe hängt \textbf{nicht} von der Eingabe ab!\\ 
% 	Dann def. wir $$ g_* (z,w) := h(f_*(z,w)) $$
% 	\pause
% 	Für $g_{**}$ gilt dann mit $h^{**}\from Z^*\functionto Y^*$ \quad (der durch $h$ induzierte Hom.!)  $$ g_{**}(z,w) = h^{**} (f_{**}(z,w)) $$
% 	Wir wenden also $h$ auf jeden durchlaufenen Zustand an.
% \end{frame}


% \begin{frame}{Beispiel Moore-Automat}
% 	\includegraphics[page=11,trim=18px 47px 18px 13px,clip, scale=.9]{U12.pdf}	
% \end{frame}


% \begin{frame}{Umwandlung von Automaten: Moore $\leftrightsquigarrow$ Mealy}
% 	\begin{block}{Von Moore zu Mealy}
% 		Relativ straightforward.\\
% 		Ausgaben \enquote{aus den Knoten auf die Übergänge ziehen}\\
% 		\medskip
% 		\textbf{Beachte}: Für die Eingabe $\eps$ kann ein Moore-Automat eine Ausgabe $g_{**}(s, \eps) \neq \eps$ erzeugen, ein Mealy-Automat \emph{jedoch nicht}.
% 	\end{block}

% 	\begin{block}{Von Mealy zu Moore}
% 		Komplizierter.
% 	\end{block}
% \end{frame}

% \begin{frame}{Beispiel: Umwandlung von Mealy nach Moore}
%     \begin{columns}
%         \begin{column}{.5\textwidth}
%             %\vspace{-2\baselineskip}
%         	\begin{tikzpicture}[->,>=stealth,shorten >=1pt,auto,node distance=2.8cm,
%         	semithick,initial text={}]
%         	\tikzstyle{every state}=[]
        	
%         	\node[initial,state] (A)                    {$a$};
%         	\node[state]         (M) [right of=A] 	    {$m$};
        	
%         	\path (A) edge [loop above] node {$\word 1\io\word 0$} (A)
%         	edge [loop below]  node {$\word 0\io\word 1$} (A)
%         	edge 					  node {$\word{2}\io\word{X}$} (M)
%         	(M) edge [loop right] node {$\stackedtight{\word 0\io\word X \\ \word 1\io\word X \\ \word 2\io\word X}$} (M);
%         	\end{tikzpicture}
%         \end{column}
%         \pause  
%         \begin{column}{.5\textwidth}
%             % \vspace{-1\baselineskip}
%         	%\mbox{}\hspace{.6\textwidth} 
%         	\begin{tikzpicture}[->,>=stealth,shorten >=1pt,auto,node distance=2.1cm,
%         	semithick,initial text={}]
%         	\tikzstyle{every state}=[]
        	
%         	\node[initial,state] (A)                    {$a\io\eps$};
%         	\node[state]         (B) [above right of=A] 	    {$b\io\word 0$};
%         	\node[state]         (C) [below right of=A] 	    {$c\io\word 1$};
%         	\node[state]         (M) [below right of=B] 	    {$m\io\word X$};
        	
%         	\path 
%         		(A) edge 				node {\word 1} (B)
%         		(A) edge 				node [below] {\word 0} (C)
%         		(B) edge [bend left=8]	node {\word 0} (C)
%         		(B) edge [loop right]   node {\word 1} (B)
%         		(C) edge [bend left=8]	node {\word 1} (B)
%         		(C) edge [loop right]   node {\word 0} (C)
%         		(B) edge 				node {\word 2} (M)
%         		(C) edge 				node [below] {\word 2} (M)
%         		(M) edge [loop right]	node {\word 0, \word 1, \word 2} (M)
%         	;
%         	\draw[->] (A) to[out=-90,in=-76, looseness=1.94] (M) node [at start, below right = 48pt and 2pt] {\word 2}; %Fuck you, LaTeX. "midway" node option doesn't work at all. Beware: `below=48pt, right=2pt` doesn't seem to work anymore?! WTF!?
%         	\end{tikzpicture}
%         \end{column}
%     \end{columns}
% \end{frame}

\begin{frame}{Aufgabe: Textersetzung mit Automaten}
	\begin{block}{Aufgabe}
		Geg.: \quad $w \in \set{\word a, \word b}^*$, wobei $w$ nicht auf ungerade viele \word{a} endet. \\
		Wollen \word{aa} durch \word{ccc} ersetzen. \\
		Malt einen Mealy-Automaten, der das leistet. \\
		\pause
		\bigskip
	\end{block}
	%\vspace{-2\baselineskip}
	% \begin{tikzpicture}[->,>=stealth,shorten >=1pt,auto,node distance=2.8cm,
	% semithick,initial text={}]
	% \tikzstyle{every state}=[]
	
	% \node[initial,state] (1)                    {$1$};
	% \node[state]         (2) [right of=A] 	    {$2$};
	
	% \path (1) edge [loop above] node {$\word b\io\word b$} (1)
	% % edge [loop below]  node {$\word 0\io\word 1$} (A)
	% edge 	[bend left=12]				  node {$\word{a}\io\eps$} (2)
	% (2) edge [bend left=12] node  {$\stackedtight{\word b\io\word{ab} \\ \word a\io\word{ccc}}$} (1);
	% \end{tikzpicture} \\
	% Zustand 2 „merkt sich“, dass ein einzelnes \word a gelesen wurde. \\
	% \pause
	% \medskip
	% \Impl Beispiel:  $g_{**}(1, \word{babaabaa}) = \word{babcccbccc}$.
	% Einschränkungen von oben mit ba und so nötig, weil sonst das letzte a nicht ausgegeben wird (möglicher Fix: Ein End-of-Input-Zeichen, das am Ende „aufräumt“)
\end{frame}