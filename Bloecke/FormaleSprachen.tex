\section{Formale Sprachen}

\begin{frame}{Rückblick}
	Sei $A = \{\word A, \word B, ..., \word Z, \word a, \word b, ..., \word z, \word\sp\}$ ein Alphabet.\\
	\pause
	Dann enthält $A^*$ alle Wörter, die man mit Zeichen aus $A$ bilden kann. Aber nicht jedes dieser Wörter ist auch sinnvoll.\\[1em]
	
	\enquote{\word{\thassedaniel{egnarts\sp si\sp efiL}{retsinnaL\sp nosrO}}} ist kein sinnvolles Wort. \pause Oder? \\[1em]
	\pause
	\impl Hängt vom Kontext ab!\\
	
\end{frame}

\begin{frame}{Formale Sprachen}
		\begin{Definition}
			Sei $A$ ein Alphabet.\\
			Eine \textbf{formale Sprache} $L$ ist eine Teilmenge von $A^*$.
		\end{Definition}
		\pause
		\vspace{10pt}
		Durch eine formale Sprache geben wir an, welche der möglichen Wörter über $A$ wir in einem bestimmten Kontext als „wünschenswert“/„interessant“ ansehen.\\
		\pause \smallskip
		Formale Sprachen werden häufig nicht direkt, sondern über Bildungsvorschriften angegeben.
		
		\pause
		\begin{Beispiel}
			$A = \{\word 0, \word 1\}$ \\
			$L = \{ w \in A^* \mid w \text{ endet auf } \word{10} \}  = \{\word{10}, \word{010}, \word{110}, \word{0010}, ...\}$
		\end{Beispiel}
\end{frame}







\begin{frame}{Produkt}
		\begin{Definition}
			Seien $L_1$ und $L_2$ zwei formale Sprachen. Dann bezeichnet
				$$L_1 \cdot L_2 := \{w_1 w_2 \mid w_1 \in L_1 \text{ und } w_2 \in L_2 \}$$
				das \textbf{Produkt} der Sprachen $L_1$ und $L_2$.
		\end{Definition}
		\pause
		In $L_1 \cdot L_2$ sind also alle Wörter enthalten, deren erster Teil aus $L_1$ und deren zweiter Teil aus $L_2$ ist.
	
\end{frame}

\begin{frame}{Produkt}
	\begin{Beispiele}
		$\{\word a, \word b\} \cdot \{\word c, \word d\} = \{\word{ac}, \word{ad}, \word{bc}, \word{bd}\}$\\[0.3em]
		\pause
		$ L = \{w_1w_2 \mid w_1\in \{\literal{b}\}^* ,  w_2\in \{\literal{a}\}^* \} 
		= \{\literal{b}\}^* \cdot \{\literal{a}\}^* $\\[1em]
		\pause
		Für alle formalen Sprachen $L$ gilt:\\
		$$ L \cdot \{\varepsilon\} = L  \qquad L \cdot \emptyset = \emptyset$$
	\end{Beispiele}
\end{frame}



\begin{frame}{Potenzen}
	\begin{Definition}
	Potenzen formaler Sprachen definieren wir induktiv so: 
	$$L^0 := \{\varepsilon \}$$
	$$L^{i+1} := L^i \cdot L$$
	\end{Definition} \pause
	$L^i$ enthält also alle Kombinationen von $i$ (nicht unbedingt verschiedenen) Wörtern aus $L$.
\end{frame}

\begin{frame}{Konkatenationsabschluss}
	\begin{Definition}
		Der \textbf{Konkatenationsabschluss} einer Sprache $L$ ist \quad $L^\ast = \bigcup \limits_{i=0}^\infty L^i$. \\
		$$ \set{}^* = \set{ \eps } \qquad \set{\word{abc}, \word{xy}}^* = \set{\eps, \word{abc}, \word{xy}, \word{abcabc}, \word{abcabcxy}, \word{xyabc}, ...} $$ 
		\medskip
		\pause
		Der \textbf{$\eps$-freie Konkatenationsabschluss} ist \quad $L^+ = \bigcup \limits_{i=1}^\infty L^i$. \\
		$$ \set{}^+ = \set{} \qquad \set{\word{abc}, \word{xy}}^+ = \set{\word{abc}, \word{xy}, \word{abcabc}, \word{abcabcxy}, \word{xyabc}, ...} $$
	\end{Definition} \pause
	\textbf{Achtung}: Der $\eps$-freie Konkatenationsabschluss muss nicht „$\eps$-frei“ sein! \\ \pause
	$$ \set{\eps}^+ = \underbrace{\set{\eps}}_{\mathclap{\text{enthält ein $\eps$!}}} $$
	
\end{frame}

\begin{frame}{Beispiele aus dem Leben}
	\begin{itemize}
		\item Sprache der korrekten IP4-Adressen: \pause
		\begin{align*}
		\set{\word{192.168.178.1}, \quad 
			\word{76.147.112.6}, ... }
		\end{align*} 
		\pause Enthält aber nicht: $\word{000.999.123.666}$ \pause
		\item Formale Sprache der Schlüsselwörter in Java: $$L = \{ \word{class}, \word{int} , \word{if}, \dots \}$$ \pause
		\item Formale Sprache der ganzen Zahlen: Mit $A = \{\word 0, ..., \word 9\}$ bauen wir
		\pause $$A \cdot A^* = \{\word{12849}, \word{1001}, \word{42}, ...\}$$ 
		\pause Und minus? Also besser $ \{ \mword-, \varepsilon \} \cdot A \cdot A^*$
	\end{itemize}
\end{frame}

\begin{frame}{Formale Sprachen}
	
	\begin{Beispiel}
		Sei $L$ formale Sprache aller Wörter über $A=\{\literal{a},\literal{b}\}$, in denen nirgends das Teilwort $\literal{ab}$ vorkommt.
		\begin{itemize}
			\pause
			\implitem $L=\{\literal{a},\literal{b}\}^*
			\setminus \set{w_1 \cdot \literal{ab} \cdot w_2 \mid w_1,w_2\in
			\{\literal{a},\literal{b}\}^* }$
			
			\pause
			\item Erst ein beliebiges Wort (evtl. $\varepsilon$) nur aus $\literal{b}$,\\
			danach ein beliebiges Wort (evtl. $\varepsilon$) nur aus $\literal{a}$.
			
			\pause
			\implitem $L=\set{w_1w_2 \mid w_1\in 
			\{\literal{b}\}^*  \text{ und }  w_2\in \{\literal{a}\}^* }$
		\end{itemize}
	\end{Beispiel}
	
	\begin{block}{Bemerkungen}
		\begin{itemize}
			\item Die Beschreibung einer formalen Sprache ist \textbf{nicht} eindeutig (siehe oben).
			\pause
			\item Immer auf den Unterschied achten: $\{\literal{abc} \} \neq \literal{abc} $
		\end{itemize}
	\end{block}
\end{frame}


\begin{frame}{Aufgabe}
	Es sei $A = \{\word a, \word b\}$. Beschreibt die folgenden formalen Sprachen mit den Symbolen $\word\{, \word\}, \word a, \word b, \textcolor{blue}{\eps}, 
	\mword{\cup}, 
	\mword{{}^*}, \mword{{}^+},
	\word, , \mword{\cdot} , \word( \text{ und } 
	\word)
	$:
	\begin{itemize}
		\item die Menge aller Wörter über $A$, die das Teilwort \word{ab} enthalten. \\
		\uncover<2-|handout:2>{$\{\word a, \word b\}^\ast \cdot \{\word{ab}\} \cdot \{\word a, \word b\}^\ast$}
		\item die Menge aller Wörter über $A$, deren vorletztes Zeichen ein \word b ist. \\
		\uncover<3-|handout:2>{$\{\word a, \word b\}^\ast \cdot \{\word b\} \cdot \{\word a, \word b\}$}
		% $\{\word a, \word b\}^\ast \cdot \{\word b\} \cdot \{\word a, \word b\}$
		\item die Menge aller Wörter über $A$, in denen nirgends zwei \word bs unmittelbar hintereinander vorkommen. \\
		\uncover<4-|handout:2>{$\{\word a, \word{ba}\}^\ast \cdot \{ \word b, \eps \}$}
	\end{itemize}
\end{frame}


