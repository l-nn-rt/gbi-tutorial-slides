%beamer

% TODO:
% - VL-Folien einarbeiten
% - Weniger f_* etc. Theorie, mehr Aufgaben zwischendrin und zu Akzeptoren
% - Bessere Hinleitung zu Akzeptoren

% Comment/uncomment this line to toggle handout mode
% \newcommand{\handout}{}

\input{../framework/PraeambelTut.tex}

\morescalingdelimiters

\begin{document}
\starttut{12}


\framePrevEpisode

\begin{frame}{Rückblick: Laufzeitabschätzungen}
	\begin{itemize}[<+->]
		\item Asymptotisches Wachstum
		\item $O, \Theta, \Omega$
		\item Beweisverfahren
		\item Rechenregeln im O-Kalkül
	\end{itemize}

	\pause
	\begin{block}{Typische Laufzeiten}
		Lineare Suche: $\Th{n}$ \\
		Binäre Suche:  $\Th{\log n}$ \\
		Potenzmenge berechnen: $\Th{2^n}$
	\end{block}
\end{frame}

\begin{frame}{Rückblick}
	\centering
	\includegraphics[scale=0.5]{laufzeit/polyVsExp}
\end{frame}

\begin{frame}[t]{Wahr oder Falsch?}
	\TrueQuestionE{$x^4 \in \Oh{{(x^3)}^3}$}{ $ {(x^3)}^3 = x^9$}
	\FalseQuestionE{$\sqrt{n} \in \Om{2^n}$}{}
	\TrueQuestionE{$log_{5000} n \in \Th{\log_2{n^4}}$}{ $ \log_2{n^4} = 4 \cdot \log_2{n}$}
	%
	% ACHTUNG: Sehr aufwendig zu erklären. Besser weglassen, um Zeit zu sparen!
	\FalseQuestionE{$O(f_1) + f_2 = O(f_1 + f_2)$}{Z.~B.: \\ $\Oh{n} + 4 = \set{f(n) + 4 \Mid f(n) \in \Oh{n}} \neq \set{f(n) \Mid f(n) \in \Oh{n}} = \Oh{n+4}$. \\ \textbf{Aber} es gilt: $O(f_1) + O(f_2) = O(f_1 + f_2)$}
\end{frame}

\begin{frame}[t]{Wahr oder Falsch?}
	\FalseQuestion{Für zwei Funktionen $f, g$ gilt immer $f \preceq g$ oder $f \succeq g$.}
	\medskip
	
	\visible<2>{
		Es gibt unvergleichbare Funktionen! Beispiel:
		\begin{align*}
		f(n) &=
		\begin{cases}
		1, & \text{ falls $n$ gerade} \\
		n, & \text{ falls $n$ ungerade} \\
		\end{cases} \\
		g(n) &=
		\begin{cases}
		n, & \text{ falls $n$ gerade} \\
		1, & \text{ falls $n$ ungerade} \\
		\end{cases} \\
		\end{align*}
		Es gilt \textbf{nicht} $g\preceq f$, es gilt \textbf{nicht} $f\preceq
		g$ und es gilt \textbf{erst recht nicht} $f\asymp g$.
	}
\end{frame}

\begin{frame}[t]{Wahr oder Falsch?}
	\FalseQuestionE{Das (komplizierte) Master-Theorem kann man immer anwenden.}{ Nur bei rekursiven Algorithmen, bei denen das Problem in gleich große Teilprobleme aufgeteilt wird.}
	\FalseQuestionE{Jeder Moore-Automat kann in einen Mealy-Automaten umgewandelt werden, der für jedes Wort die gleiche Ausgabe produziert.}{ Für das leere Wort kann ein Mealy-Automat niemals eine Ausgabe produzieren.}
	\TrueQuestionE{Endliche Akzeptoren sind Moore-Automaten mit dem Ausgabealphabet $\{\word 0,\word 1\}$.}{}
	\FalseQuestionE{Mit endlichen Automaten kann jede beliebige Sprache erkannt \\ werden.}{Tatsächlich ist die Menge der akzeptierbaren Sprachen sogar sehr eingeschränkt.}	
\end{frame}


\input{../Bloecke/RechtslineareGrammatiken}

\input{../Bloecke/Turing}

\input{../Bloecke/TMKomplexitaet}

\begin{frame}{Turingmaschinen: Klausur}
	Noch ein Hinweis zum Schluss:\\
	Bisher kam in {\tiny fast} \textbf{jeder} Klausur eine Aufgabe zu Turingmaschinen dran.\\
	Diese gibt meist relativ viele Punkte.\\
	
	\bigskip
	Zu 99 \% wird auch dieses Mal wieder eine TM-Aufgabe drankommen.\\
	\textbf{Also übt das!} Hier zählt vor allem Geschwindigkeit (und Präzision).
\end{frame}

\appendix
\beginbackup

\section{Zusammenfassung und Ausblick}

\begin{frame}
	\begin{block}{Was ihr nun wissen solltet}
		\begin{itemize}
			\item Wie man rekursive Laufzeiten mastert
			\item Endlich: Automaten!
			\item ... und wie man damit Wörter akzeptiert/ablehnt
		\end{itemize}
	\end{block}
	
	\begin{block}{Was die nächsten Male kommt}
		\begin{itemize}
			\item Nicht immer so vulgär: Reguläre Ausdrücke
			\item Rechtslineare Grammatiken
			\item Turingmaschinen
			%\item Turingmaschinen -- mächtiger wird es nicht mehr!
		\end{itemize}
	\end{block}
\end{frame}






\only<handout:0>{\slideThanks}



%% Letzte Seite
\xkcdframevert{1319}{Danke für eure Aufmerksamkeit! \smiley}{2.5}
%\lastframe{0.6}{30}{xkcd/automation_1319.png}{http://www.xkcd.com/1319}
%\xkcdframe{0.5}{30}{xkcd/houston_1438.png}{http://www.xkcd.com/1438}{Oh, hi mom. No, nothing important, just work.}

\only<beamer:0>{\slideThanks}

\backupend
\end{document}